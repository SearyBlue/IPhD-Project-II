\documentclass[11pt,a4paper]{article}
\usepackage[utf8]{inputenc}
\usepackage[scale=0.9]{geometry}
\usepackage{amsmath}
\usepackage{amsfonts}
\usepackage{amssymb}
\author{Anirban Mukherjee, NS Vidhyadhiraja, Siddhartha Lal}
\title{Entanglement features of the Kondo cloud}
\begin{document}
\begin{center}\begin{Large}\textbf{Entanglement features of the Kondo Cloud}\end{Large}\end{center}
\subsubsection*{Motivation}
The key feature of the Kondo screening cloud is the entanglement content between a magnetic impurity and conduction electrons in the vicinity of the Fermi surface. A recent work Phys. Rev. Lett. 120, 146801 (2018) shows that the entanglement content is related to electronic conductivity. In many body systems entanglement entropy scaling shows distinct features for gapped as against gapless phases Phys. Rev. Lett. (2007), arxiv:2003.06118 (2020).  Are there any observable entanglement RG scaling features with regards to formation of the Kondo cloud? Our  URG procedure mitigates fermion exchange signatures , i.e. it functions as a decoder circuit comprising a error correcting code leading to an emergent subspace where an electronic cloud entangles with the Kondo spin. In this work we want to study the interplay fermion exchange signatures, many particle entanglement, and quantum transport observables like conductivity, shot noise, spectral function etc.\\
\par\noindent
\textbf{List of things we can do}
\par\noindent
\begin{enumerate}
\item[1.]We can start with the Heisenberg Kondo Hamiltonian with isotropic Fermi surface of the Fermi liquid and obtain the URG flow in the space of Hamiltonians arranged from UV to IR. From the IR fixed points obtained in the antiferromagnetic side of the Kondo model we can compute the effective Hamiltonian and the eigenstates.
\item[2.]Our experience suggests that the effective Hamiltonian in the strong coupling regime on the antiferromagnetic side will be of the pseudospin kind. By reversing the RG flow we can tomographically create the many body states at UV, by re-entangling the high energy electronic states with their IR counterparts. This allows realization of an entanglement renormalization group and altogether comprise the construction of the EHM tensor network.
\item[3.]In the construction of the entanglement RG flow we can study the effect of fermion exchange signatures in the entanglement entropy, mutual information (MI) flow. We can also study the Ryu-Takayanagi entropy bound, emergent holographic spacetime generated from MI. Can this entanglement features witness the entanglement phase transition between the ferromagnetic and antiferromagnetic side of the Kondo model?
\item[4.]We can extract the reduced density matrix comprising the HIlbert space associated with the Fermi surface (FS) and the Kondo Impurity (KI). We can study the RG dynamics of MI content between the FS and the KI, does this show the formation of the Kondo cloud? Does the fermion exchange signs have observable effects in the entanglement scaling flow towards the Kondo cloud?
\item[5.]In the case when the Kondo cloud is formed can we confirm Martin’s sum rule, i.e. the reduction in Luttinger’s sum by the no. of electronic states added to the KI. This implies that the ferromagnetic to antiferromagnetic transition is a topological transition. How does this coincide with our understanding of the entanglement phase transition?
\item[6.]Finally we can study the holographic renormalization of the quantum geometric tensor for the Fermi surface and KI HIlbert space, this will surely be a witness to the formation of the Kondo cloud.
\item[7.]Show quantum advantage in the kondo cloud for error correction. But it requires a bit more study of the current plan. Especially the entanglement scaling features fermion sign issues. Then we can understand how they get error corrected upon scaling, and eventually form the cloud, i.e., How fermion sign issues are resolved resulting in the formation of Kondo cloud. This could be used in a proposal for quantum error correction and an equivalent machine learning protocol.
\item[8.]Can we perform a gauge theoretic construction of the local quantum liquid generated by isolating the Kondo impurity via tracing out its degree of freedom?
\end{enumerate}
\maketitle
\begin{abstract}
Loading...
\end{abstract}
%The Kondo Hamiltonian in 2D momentum space,
\begin{eqnarray}
\hat{H} = \sum_{\mathbf{k}\sigma}\epsilon_{\mathbf{k}}\hat{n}_{\mathbf{k}\sigma}+\frac{J}{2}\sum_{\mathbf{k},\mathbf{k}'}\mathbf{S}\cdot c^{\dagger}_{\mathbf{k}\alpha}\boldsymbol{\sigma}_{\alpha\beta}c_{\mathbf{k}'\beta}
\end{eqnarray}

\begin{eqnarray}
\mathbf{k}_{\Lambda\hat{s}}=\mathbf{k}_{F}(\hat{s})+\Lambda\hat{s}, \hat{s}=\frac{\nabla\epsilon_{\mathbf{k}}}{|\nabla\epsilon_{\mathbf{k}}|}|_{\epsilon_{\mathbf{k}}=E_{F}}
\end{eqnarray}
\begin{eqnarray}
|j,l,\sigma\rangle = |\mathbf{k}_{\Lambda_{j}\hat{s}},\sigma\rangle, l:=(\hat{s}_{m},\sigma)
\end{eqnarray}
If there are $n_{j}$ normal directions then $l$ ranges from $1$ to $2n_{j}$ as follows: $1=(\hat{s}_{1},\uparrow)$, $2=(\hat{s}_{1},\downarrow)$, $3=(\hat{s}_{2},\uparrow)$, $\ldots$. 
\begin{eqnarray}
H_{(j-1)}=U_{(j)}H_{(j)}U^{\dagger}_{(j)}
\end{eqnarray}
\begin{eqnarray}
U_{(j)}&=&\prod_{l}U_{j,l}U_{j,l}, \nonumber\\ 
U_{j,l}&=&\frac{1}{\sqrt{2}}[1+\eta_{j,l}-\eta^{\dagger}_{j,l}],\nonumber\\
\eta_{j,l}&=&Tr_{j,l}(c^{\dagger}_{j,l}H_{j,l})c_{j,l}\frac{1}{\hat{\omega}_{j,l}-Tr_{j,l}(H_{j,l}^{D}\hat{n}_{j,l})\hat{n}_{j,l}},\nonumber\\
\hat{\omega}_{j,l}=H^{D}_{j,l}+H^{X}_{j,l}-H^{X}_{j,l-1}
H_{j,l}&=&\prod_{m=1}^{l}U_{j,m}H_{(j)}[\prod_{m=1}^{l}U_{j,m,\sigma}]^{\dagger}
\end{eqnarray}
Note $H_{j,2n_{j}+1}=H_{(j-1)}$.
\begin{eqnarray}
H_{j,l+1}=Tr_{j,l}(H_{(j,l)})+\lbrace c^{\dagger}_{j,l}Tr_{j,l}(H_{(j,l)}c_{j,l}),\eta_{j,l}\rbrace\tau_{j,l}
\end{eqnarray}
Ignoring the higher order correlated tangential scattering processes on a given high energy shell 
(constituting states 
being 
\begin{eqnarray}
\hat{H}_{(j-1)}(\omega) = \sum_{j,l,\sigma}\epsilon_{j,l}\hat{n}_{j,l}+\frac{J^{(j-1)}(\omega)}{2}\sum_{\substack{j_{1},j_{2}<j-1,\\ \hat{s}_{1},\hat{s}_{2}}}\mathbf{S}\cdot c^{\dagger}_{j_{1},\hat{s}_{1},\alpha}\boldsymbol{\sigma}_{\alpha\beta}c_{j_{2},\hat{s}_{2},\beta}(1+
\sum^{j}_{l=1}\tau_{l}+\sum^{j}_{l,l'=1}\tau_{l}\tau_{l'}+\ldots)~.
\end{eqnarray}
\begin{eqnarray}
J^{(j-1)}-J^{(j)}=\sum_{m=1}^{n_{j}}\left[\frac{2(J^{(j)})^{2}\tau_{j,\hat{s}_{m},\uparrow}}{\omega-\epsilon_{j,m}\tau_{j,\hat{s}_{m},\uparrow}-\frac{J^{(j)}}{2}S^{z}(\tau_{j,\hat{s}_{m},\uparrow}-\tau_{j,\hat{s}_{m},\downarrow})}+\frac{2(J^{(j)})^{2}\tau_{j,\hat{s}_{m},\downarrow}}{\omega-\epsilon_{j,m}\tau_{j,\hat{s}_{m},\downarrow}-\frac{J^{(j)}}{2}S^{z}(\tau_{j,\hat{s}_{m},\uparrow}-\tau_{j,\hat{s}_{m},\downarrow})}\right]
\end{eqnarray}

\begin{eqnarray}
\frac{1}{a}\left(1-\frac{b}{a}S^{z}\right)^{-1}=\frac{1}{a}\left(1+\frac{b}{a}S^{z}+\frac{b^{2}}{4a^{2}}+\frac{b^{3}}{4a^{3}}S^{z}+\ldots\right)\nonumber\\
=\frac{1}{a}\left[\frac{1}{1-\frac{b^{2}}{4a^{2}}}+\ldots\right]
\end{eqnarray}
%consists of a quantum impurity spin and a Fermi liquid(FL) bath is hybridized with the single quantum impurity via electrons scattering off the impurity spin with or without spin exchange. $\epsilon_{\mathbf{k}_{\Lambda\hat{s}}}$ is the electronic dispersion. In order to proceed we write the momentum vectors in polar coordinates $\mathbf{k}_{\Lambda\hat{s}}=\mathbf{k}_{F\hat{s}}+\Lambda\hat{s}$ where $\hat{s}=\frac{\nabla_{\mathbf{k}}\epsilon_{\mathbf{k}}}{|\nabla_{\mathbf{k}}\epsilon_{\mathbf{k}}|}|_{\epsilon_{\mathbf{k}}=E_{F}}$ are unit vectors normal to the Fermi surface.  The Hamiltonian can then be equivalently written as,
%\begin{eqnarray}
%\hat{H} = \sum_{\Lambda,\hat{s},\sigma}\epsilon_{\Lambda}\hat{n}_{\Lambda\sigma}(\hat{\mathbf{s}})+\frac{J}{2\text{Vol}}\sum_{\substack{\Lambda,\Lambda',\\ \hat{\mathbf{s}},\hat{\mathbf{s}}'}}\mathbf{S}\cdot c^{\dagger}_{\Lambda\alpha}(\hat{\mathbf{s}})\boldsymbol{\sigma}_{\alpha\beta}c_{\Lambda'\beta}(\hat{\mathbf{s}}')~.
%\end{eqnarray}
%Let us associate a unitary transformation with the disentanglement of 
%a single electronic degree of freedom belonging to the FL bath. We choose a scheme for doing this: electronic state labels are arranged with respect to the single electron energy arranged from high to low energies. Then on that highest energy 1d shell all states are arranged each marked by a angular coordinate($\phi$). Firstly all states on the highest energy shell are disentangled via successive application of the unitary operation. The product of such unitary transformation marks one high energy shell disentanglement. Successive application of the unitary shell disentanglement operations on the Hamiltonian and the associated eigenbasis renormalization leads to simultaneous renormalization of the Hamiltonian and its eigenstates. The $\Lambda$'s are arranged in ascending order from $\Lambda_{N}>\Lambda_{N-1}>\ldots>0$ where $\epsilon_{\Lambda_{N}}=D$ represents the highest energy contour corresponding to the largest distance from the Fermi surface(FS).
%The unitary operator that disentangles the state farthest from FS is given by,
%\begin{eqnarray}
%U_{\Lambda_{N},\sigma}(\hat{\mathbf{R}}_{1}) = \frac{1}{\sqrt{2}}\left[1+\eta_{\Lambda_{N},\sigma}(\hat{\mathbf{R}}_{1})-\eta^{\dagger}_{\Lambda_{N},\sigma}(\hat{\mathbf{R}}_{1})\right]
%\end{eqnarray}
%where $\eta_{\Lambda_{N},\uparrow}(\hat{\mathbf{R}}_{1})$ is given by($\hat{\tau}_{\Lambda\downarrow}(\mathbf{\hat{r}})=\hat{n}_{\Lambda\downarrow}(\mathbf{\hat{r}})-\frac{1}{2}$),
%\begin{eqnarray}
%\eta_{\Lambda_{N},\uparrow}(\mathbf{\hat{R}}_{1}) = \frac{J}{2\text{Vol}}\mathbf{S}\cdot\sum_{\Lambda,\hat{\mathbf{r}},\sigma}c^{\dagger}_{\Lambda_{N}\uparrow}(\hat{\mathbf{R}}_{1})\boldsymbol{\sigma}_{\uparrow\sigma}c_{\Lambda\sigma}(\hat{\mathbf{r}})\frac{1}{\hat{\omega} - \sum_{\Lambda',\mathbf{\hat{r}}',\sigma}\epsilon_{\Lambda'}\hat{\tau}_{\Lambda'\sigma}(\hat{\mathbf{r}'})-\sum_{\Lambda',\hat{r}'}\frac{J}{2\text{Vol}}S^{z}(\hat{\tau}_{\Lambda'\uparrow}(\mathbf{\hat{r}}')-\hat{\tau}_{\Lambda'\downarrow}(\mathbf{\hat{r}}'))}~.
%\end{eqnarray}
%The unitarily rotated Hamiltonian obtained upon disentangling one state $\Lambda_{N},\hat{\mathbf{R}}_{1},\uparrow$ from the outermost shell at $\Lambda_{N}$ is  $\hat{H}^{(N),\uparrow,\hat{\mathbf{R}}_{1}} = U_{\Lambda_{N},\uparrow}(\hat{\mathbf{R}}_{1})\hat{H}U^{\dagger}_{\Lambda_{N},\uparrow}(\hat{\mathbf{R}}_{1})$~. This when projected along a given eigen-direction of the $\hat{\omega}$ operator (where $\hat{O}^{(N),\uparrow,\hat{\mathbf{R}}_{1}}(\omega)$ is the projector) leads to ,
%\begin{eqnarray}
%H^{(N),\uparrow,\hat{\mathbf{R}}_{1}}(\omega) &=& \hat{O}^{(N),\uparrow,\hat{\mathbf{R}}_{1}}(\omega)\left[\sum_{\Lambda,\hat{r},\sigma}\epsilon_{\Lambda}\hat{n}_{\Lambda\sigma}(\hat{\mathbf{r}})+\frac{J}{2\text{Vol}}\sum_{\substack{\Lambda,\Lambda',\\ \hat{\mathbf{r}},\hat{\mathbf{r}}'\neq \hat{\mathbf{R}}_{1}}}\mathbf{S}\cdot c^{\dagger}_{\Lambda\alpha}(\hat{\mathbf{r}})\boldsymbol{\sigma}_{\alpha\beta}c_{\Lambda'\beta}(\hat{\mathbf{r}}')\right]\nonumber\\
%&+&\hat{O}^{(N),\uparrow,\hat{\mathbf{R}}_{1}}(\omega)\sum_{\substack{\Lambda'\Lambda<\Lambda_{N}, \\ \hat{\mathbf{r}}\hat{\mathbf{r}}'\neq \hat{\mathbf{R}}_{1}}}\frac{J^{2}\tau_{\Lambda_{N}\uparrow}(\hat{\mathbf{R}}_{1})}{8\text{Vol}^{2}(\omega - \epsilon_{\Lambda_{N}}\tau_{\Lambda_{N}\uparrow}(\hat{\mathbf{R}}_{1})-\frac{1}{2}(\epsilon_{\Lambda}-\epsilon_{\Lambda'})-\frac{J}{4\text{Vol}})}\bigg[\sum_{\sigma}\left(c^{\dagger}_{\Lambda\sigma}(\mathbf{\hat{r}})c_{\Lambda'\sigma}(\mathbf{\hat{r}}')+h.c.\right)\nonumber\\
%&+& 4\mathbf{S}\cdot c^{\dagger}_{\Lambda\alpha}(\hat{\mathbf{r}})\boldsymbol{\sigma}_{\alpha\beta}c_{\Lambda'\beta}(\hat{\mathbf{r}}')\bigg]~.
%\end{eqnarray}
%For the complete disentanglement of a shell the following product of disentangling unitary operation needs to be performed with respect to all the states on a given shell i.e.
%\begin{eqnarray}
%U_{(N)} = \prod_{i=1}^{l}U_{\Lambda_{N},\uparrow}(\hat{\mathbf{R}}_{i})U_{\Lambda_{N},\downarrow}(\hat{\mathbf{R}}_{i})~.
%\end{eqnarray}
%In the new Hamiltonian $H^{(N-1)} = U_{(N)}H^{(N)}U^{\dagger}_{(N)}$ the updated Hamiltonian projected along a subspace with a given energy scale $\omega$ looks like,
%\begin{eqnarray}
%H^{(N-1)}(\omega) &=& \hat{O}^{(N)}(\omega)\left[\sum_{\Lambda,\hat{r},\sigma}\epsilon_{\Lambda}\hat{n}_{\Lambda\sigma}(\hat{\mathbf{r}})+\frac{J}{2\text{Vol}}\sum_{\substack{\Lambda,\Lambda',\\ \hat{\mathbf{r}},\hat{\mathbf{r}}'\neq \hat{\mathbf{R}}_{i}'s}}\mathbf{S}\cdot c^{\dagger}_{\Lambda\alpha}(\hat{\mathbf{r}})\boldsymbol{\sigma}_{\alpha\beta}c_{\Lambda'\beta}(\hat{\mathbf{r}}')\right]\nonumber\\
%&+&\sum_{\substack{\Lambda'\Lambda<\Lambda_{N}, \\ \hat{\mathbf{r}}\hat{\mathbf{r}}'\neq \hat{\mathbf{R}}_{i}'s}}\frac{J^{2}\sum_{i=1,\sigma}^{l}\tau_{\Lambda_{N}\sigma}(\hat{\mathbf{R}}_{i})}{8\text{Vol}^{2}(\omega - \epsilon_{\Lambda_{N}}\tau_{\Lambda_{N}\uparrow}(\hat{\mathbf{R}}_{1})-\frac{1}{2}(\epsilon_{\Lambda}-\epsilon_{\Lambda'})-\frac{J}{4\text{Vol}})}\bigg[\sum_{\sigma}\left(c^{\dagger}_{\Lambda\sigma}(\mathbf{\hat{r}})c_{\Lambda'\sigma}(\mathbf{\hat{r}}')+h.c.\right)\nonumber\\
%&+& 4\mathbf{S}\cdot c^{\dagger}_{\Lambda\alpha}(\hat{\mathbf{r}})\boldsymbol{\sigma}_{\alpha\beta}c_{\Lambda'\beta}(\hat{\mathbf{r}}')\bigg]\nonumber\\
%&+&\sum_{\substack{\Lambda'\Lambda<\Lambda_{N}, \\ \hat{\mathbf{r}}\hat{\mathbf{r}}'\neq \hat{\mathbf{R}}_{i}'s}}\frac{J^{3}\sum_{i,j=1,\sigma\sigma'}^{l}\tau_{\Lambda_{N}\sigma}(\hat{\mathbf{R}}_{i})\tau_{\Lambda_{N}\sigma'}(\hat{\mathbf{R}}_{j})}{16\text{Vol}^{3}(\omega - \epsilon_{\Lambda_{N}}\tau_{\Lambda_{N}\uparrow}(\hat{\mathbf{R}}_{1})-\frac{1}{2}(\epsilon_{\Lambda}-\epsilon_{\Lambda'})-\frac{J}{4\text{Vol}})^{2}}\bigg[\sum_{\sigma}\left(c^{\dagger}_{\Lambda\sigma}(\mathbf{\hat{r}})c_{\Lambda'\sigma}(\mathbf{\hat{r}}')+h.c.\right)\nonumber\\
%&+& 4\mathbf{S}\cdot c^{\dagger}_{\Lambda\alpha}(\hat{\mathbf{r}})\boldsymbol{\sigma}_{\alpha\beta}c_{\Lambda'\beta}(\hat{\mathbf{r}}')\bigg]\nonumber\\
%&+&\sum_{\substack{\Lambda'\Lambda<\Lambda_{N}, \\ \hat{\mathbf{r}}\hat{\mathbf{r}}'\neq \hat{\mathbf{R}}_{i}'s}}\frac{J^{4}\sum_{i,j,k=1,\sigma\sigma'\sigma''}^{l}\tau_{\Lambda_{N}\sigma}(\hat{\mathbf{R}}_{i})\tau_{\Lambda_{N}\sigma'}(\hat{\mathbf{R}}_{j})\tau_{\Lambda_{N}\sigma''}(\hat{\mathbf{R}}_{k})}{32\text{Vol}^{4}(\omega - \epsilon_{\Lambda_{N}}\tau_{\Lambda_{N}\uparrow}(\hat{\mathbf{R}}_{1})-\frac{1}{2}(\epsilon_{\Lambda}-\epsilon_{\Lambda'})-\frac{J}{4\text{Vol}})^{3}}\bigg[\sum_{\sigma}\left(c^{\dagger}_{\Lambda\sigma}(\mathbf{\hat{r}})c_{\Lambda'\sigma}(\mathbf{\hat{r}}')+h.c.\right)\nonumber\\
%&+& 4\mathbf{S}\cdot c^{\dagger}_{\Lambda\alpha}(\hat{\mathbf{r}})\boldsymbol{\sigma}_{\alpha\beta}c_{\Lambda'\beta}(\hat{\mathbf{r}}')\bigg]+\ldots~.
%\end{eqnarray}
%The RG equation for the Kondo coupling with contributions from the leading two body scattering potential just accounts for the energy density of states on the outermost shell i.e. expectation value of $\langle \sum_{i=1,\sigma}^{l}\tau_{\Lambda_{N},\sigma}(\hat{\mathbf{R}}_{i})\rangle =N_{(j)}$ is given by($J_{\Lambda\Lambda'}^{(j)}/\text{Vol}=\bar{J}_{\Lambda\Lambda'}^{(j)}$),
%\begin{eqnarray}
%\Delta\bar{J}_{\Lambda\Lambda'}^{(j)} = \frac{N_{(j)}\bar{J}_{\Lambda\Lambda_{j}}^{(j)}\bar{J}_{\Lambda_{j}\Lambda'}^{(j)}}{\omega - \frac{1}{2}\epsilon_{\Lambda_{j}}-\frac{1}{2}(\epsilon_{\Lambda}-\epsilon_{\Lambda'})-\frac{\bar{J}^{(j)}}{4}}
%\end{eqnarray}
%where  $J_{\Lambda\Lambda}^{(j)} = J^{(j)}$ and $N_{(j)}/\text{Vol}=N^{0}_{(j)}$ is the density of states on the jth spherical shell. The Anderson poor man scaling form can be arrived at from this RG equation by dropping the $\epsilon_{\Lambda},\epsilon_{\Lambda'}, J^{(j)}$ dependence in the denominator and replacing $\omega-\frac{1}{2}\epsilon_{\Lambda_{j}}$ by the rescaled bandwidth ($D$).
%
% The fixed point condition of this RG equation can be obtained from the zero of the denominator(here we take $\epsilon_{\Lambda}=\hbar v_{F}\Lambda$ and 
% $\epsilon_{\Lambda'}=\hbar v_{F}(\Lambda-\delta)$), $\epsilon_{\Lambda_{j^{*}_{\delta}}}=\epsilon_{\Lambda^{*}_{\delta}}=\hbar v_{F}\Lambda^{*}_{\delta}$
% 
% \begin{eqnarray}
% \omega = \frac{1}{2}\hbar v_{F}\Lambda^{*}_{\delta}+\frac{1}{2}\hbar v_{F}\delta+\frac{\bar{J}^{*}(\delta)}{4}~.\label{fixed-point-condition}
% \end{eqnarray}
% The fixed point Hamiltonian is given by,
% \begin{eqnarray}
% H^{*}(\omega) = \sum_{\Lambda<\max_{\delta}\Lambda^{*}_{\delta},\sigma,\hat{\mathbf{r}}}\epsilon_{\Lambda}\hat{n}_{\Lambda\sigma}(\hat{\mathbf{r}})+\frac{1}{2}\sum_{\delta,\Lambda<\Lambda^{*}_{\delta},\hat{r},\hat{r}'}\bar{J}^{*}(\delta)\mathbf{S}\cdot (c^{\dagger}_{\Lambda+\delta,\alpha}(\hat{\mathbf{r}})\boldsymbol{\sigma}_{\alpha\beta}c_{\Lambda,\beta}(\hat{\mathbf{r}'})+h.c.)
% \end{eqnarray}
% where $J^{*}_{\Lambda\Lambda'}=J^{*}(\Lambda-\Lambda')$ depends only on the distance between the two momentum space shells. The electronic creation and annihilation($\tilde{c}^{\dagger}_{\Lambda,\alpha},\tilde{c}^{\dagger}_{\Lambda,\alpha}$) operators for a given shell can be obtained by superposed the creation/annhilation at every point on the shell($c^{\dagger}_{\Lambda,\alpha}(\hat{\mathbf{r}}),c_{\Lambda,\alpha}(\hat{\mathbf{r}})$) in the transverse direction,
% \begin{eqnarray}
% \tilde{c}^{\dagger}_{\Lambda,\alpha}=\frac{1}{\sqrt{N_{\Lambda}}}\sum_{\hat{\mathbf{r}}}c^{\dagger}_{\Lambda,\alpha}(\hat{\mathbf{r}}).
% \end{eqnarray}
%  In terms of this new fermion operators the Hamiltonian is given by
%  ($\hat{N}_{\Lambda}=\sum_{\hat{\mathbf{r}},\sigma}\hat{n}_{\Lambda\sigma}(\hat{\mathbf{r}}$),
% \begin{eqnarray}
% H^{*}(\omega) = \sum_{\Lambda<\Lambda^{*}_{0}}\epsilon_{\Lambda}\hat{N}_{\Lambda}+\frac{1}{2}\sum_{\delta,\Lambda<\Lambda^{*}_{\delta}}\bar{J}^{*}(\delta)\sqrt{N_{\Lambda}N_{\Lambda+\delta}}\mathbf{S}\cdot (\tilde{c}^{\dagger}_{\Lambda+\delta,\alpha}\boldsymbol{\sigma}_{\alpha\beta}\tilde{c}_{\Lambda,\beta}+h.c.)~.
%\end{eqnarray}
%Among the couplings $\bar{J}^{*}(\delta)$ the coupling $\bar{J}^{*}(0)$ has the highest magnitude for $\delta=0$(seen from the fixed point condition). The ordering of the couplings for increasing $\delta$ is given as $\bar{J}^{*}(0)>\bar{J}^{*}(\delta)>\ldots>\bar{J}^{*}(n\delta)$. By taking account of only the $\delta=0$ scattering vertex we write down the truncated Hamiltonian($\mathbf{s}_{\Lambda} = \tilde{c}^{\dagger}_{\Lambda,\alpha}\boldsymbol{\sigma}_{\alpha\beta}\tilde{c}_{\Lambda,\beta}$),
%\begin{eqnarray}
%H^{*}(\omega)
%&=&\sum_{\Lambda<\Lambda^{*}_{0}}\epsilon_{\Lambda}\hat{N}_{\Lambda}+\bar{J}^{*}(0)\sum_{\Lambda<\Lambda^{*}_{0}}N_{\Lambda}\mathbf{S}\cdot\mathbf{s}_{\Lambda}
%\end{eqnarray}
%where $\mathbf{s}_{\Lambda}=c^{\dagger}_{\Lambda\alpha}\boldsymbol{\sigma}_{\alpha\beta}c_{\Lambda\beta}$. For the shells close to the Fermi energy $E_{F}$ we can approximate the energy density of states to $N_{0}$ i.e. the number of states at $E_{F}$. Then the Hamiltonian is given by,
%\begin{eqnarray}
%H^{*}(\omega) &=&\sum_{\Lambda<\Lambda^{*}_{0}}\epsilon_{\Lambda}\hat{N}_{\Lambda}+\bar{J}^{*}(0)N_{0}\mathbf{S}\cdot\mathbf{S}_{<\Lambda^{*}_{0}},~\mathbf{S}_{<\Lambda^{*}_{0}} = \sum_{\Lambda<\Lambda^{*}_{0}}\mathbf{s}_{\Lambda}~.
% \end{eqnarray}
%We will now perform an entanglement renormalization group study for the low energy Hilbert space, for that we start with the following Hamiltonian,
%\begin{eqnarray}
%\hat{H}^{(j)}(\omega) = \sum_{\Lambda<\Lambda_{j}}\epsilon_{\Lambda}\hat{N}_{\Lambda}+\frac{1}{2}\sum_{\Lambda<\Lambda_{j},\delta}\bar{J}^{(j)}(\delta)\mathbf{S}\cdot\tilde{c}^{\dagger}_{\Lambda+\delta,\alpha}\boldsymbol{\sigma}_{\alpha\beta}\tilde{c}_{\Lambda,\beta}~.
%\end{eqnarray}
%An ansatz wavefunction that can be an eigenstate belonging to the low energy Hilbert space is given by,
%\begin{eqnarray}
%|\Psi^{(j)}\rangle = \sum_{\delta,\rho(\sigma),\sigma}C^{(j)}_{\rho(\sigma),\chi_{\delta}(-\sigma)}|\rho(\sigma),\chi_{\delta}(-\sigma)\rangle 
%\end{eqnarray}
%where $\rho(\sigma)$ is a set of electron occupied n+1 momentum states in the spin configuration $\sigma$ i.e.,
%\begin{equation}
%\rho(\sigma) = \lbrace (\Lambda_{l_{1}},\sigma),(\Lambda_{l_{2}},\sigma),\ldots,(\Lambda_{l_{n+1}},\sigma)\rbrace~,
%\end{equation}
%and $\chi_{\delta}(-\sigma)$ is a set of n occupied momentum states and a impurity spin $d$ in the spin configuration $-\sigma$,
%\begin{equation}
%\chi_{\delta}(-\sigma) = \lbrace (\Lambda_{l_{1}},-\sigma),(\Lambda_{l_{2}},-\sigma),\ldots,(d,-\sigma)\rbrace~.
%\end{equation}
%The coefficient RG mapping from $(j)\to (j-1)$ is given by,
%\begin{eqnarray}
%C^{(j-1)}_{\rho(\sigma),\chi_{\delta}(-\sigma)}=\sqrt{N^{(j)}}C^{(j)}_{\rho(\sigma),\chi_{\delta}(-\sigma)}+\sqrt{N^{(j)}}\Delta J^{(j)}(\delta)\sum_{\sigma',\bar{\rho}(\sigma')}\frac{C^{(j)}_{\bar{\rho}(\sigma')\bar{\chi}_{\delta}(-\sigma')}}{\omega -\frac{1}{2}\hbar v_{F}\delta-\frac{1}{4}J^{(j)}}
%\end{eqnarray}
%where the summation is over those sets for whom $\bar{\rho}(\sigma')\cup\rho(\sigma)-\bar{\rho}(\sigma')\cap\rho(\sigma)=\lbrace(\Lambda_{l_{i}},\sigma'),(\Lambda_{l_{i}}+\delta,\sigma)\rbrace$. The above set describes both the spin flip and non-spin flip scattering processes for $\sigma'=\sigma$  and $\sigma'=-\sigma$ respectively.
\end{document}