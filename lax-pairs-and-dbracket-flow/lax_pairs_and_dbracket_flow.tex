\documentclass[14pt]{extarticle}
\usepackage[T1]{fontenc}
\usepackage[utf8]{inputenc}
\usepackage{amsthm,amsmath,amssymb,braket,graphicx,enumitem,booktabs,multirow,booktabs,tcolorbox,wrapfig,cancel,caption}
\usepackage[margin=1in, top=0.5in]{geometry}
\usepackage{charter}
\begin{document}
\title{Lax Pairs and Double Bracket Flows}
\author{}
\date{}
\maketitle
\section{Definition of a Lax Pair}
Two operator \(A\) and \(B\) are said to form a lax pair if they satisfy the equation
\begin{equation}\begin{aligned}
	\label{lax}
	\frac{\mathrm{d}A(t)}{\mathrm{d}t} = \left[B(t), A(t)\right]
\end{aligned}\end{equation}
\section{Unitary Nature of the Flow}
It can be shown that this defines a unitary time evolution on \(A(t)\), in the following manner. Let \(U(t,t_0)\) be the unitary operator that carries this evolution through. We then need to construct a \(U(t,t_0)\). 
\begin{equation}
	A(t) = U(t,t_0) A(t_0) U^\dagger (t,t_0)
\end{equation}
where \(A(t_0)\) is the operator \(A\) at a particular time \(t_0\). The time change of \(A\) can then be written as
\begin{equation}\begin{aligned}
	\frac{\mathrm{d}A(t)}{\mathrm{d}t} &= \frac{\mathrm{d}U(t, t_0)}{\mathrm{d}t} A(t_0) U^\dagger(t,t_0) + U(t,t_0) A(t_0) \frac{\mathrm{d}U^\dagger(t, t_0)}{\mathrm{d}t}\\
					   &= \frac{\mathrm{d}U(t, t_0)}{\mathrm{d}t} U^\dagger(t,t_0) A(t) + A(t) U(t,t_0) \frac{\mathrm{d}U^\dagger(t, t_0)}{\mathrm{d}t} && \left[ A(t) = U A U^\dagger\right]\\ 
					   &= \frac{\mathrm{d}U(t, t_0)}{\mathrm{d}t} U^\dagger(t,t_0) A(t) - A(t) \frac{\mathrm{d}U(t, t_0)}{\mathrm{d}t} U^\dagger(t,t_0) && \left[ U U^\dagger = 1 \right]\\ 
					   &= \left[\frac{\mathrm{d}U(t, t_0)}{\mathrm{d}t} U^\dagger(t,t_0), A(t)\right]
\end{aligned}\end{equation}
Looking at the definition of a lax pair, we can now make the connection
\begin{equation}\begin{aligned}
	\label{unitary}
	B(t) = \frac{\mathrm{d}U(t, t_0)}{\mathrm{d}t} U^\dagger(t,t_0)
\end{aligned}\end{equation}

\textbf{The equation of motion characterised by the lax pair eq.~\ref{lax} can thus be said to generate a family of unitarily connected operators \(A(t)\), related by the unitaries defined by eq.~\ref{unitary}. A direct corrolary is that the spectrum of \(A(t)\) is preserved during this evolution.}
\section{Double Bracket Flow}
The double bracket flows correspond to a special choice of the operator \(B(t)\): \(B(t) \equiv \left[A(t), C\right]\). A consequence of this choice is that the lax pair evolution then serves to minimize the commutator \(\left[A(t), C\right]\). To see how, we first write down a function
\begin{equation}
	\chi \equiv \text{Tr}\left(\left[A(t) - C\right]^2\right) = \text{Tr}\left[A(t)^2 + C^2 - A(t) C - C A(t)\right]
\end{equation}
Since \(A^2(t) = U A^2 U^\dagger\), we get \(\text{Tr}(A^2(t)) = \text{Tr}(A)\). Also, from the cyclic nature of trace, we can write \(\text{Tr}(A(t)C)=\text{Tr}(CA(t))\). These considerations (and the fact that \(C\) does not depend on \(t\)) allows us to write
\begin{equation}
	\frac{\mathrm{d} \chi}{\mathrm{d}t} = -2\text{Tr}\left(\frac{\mathrm{d} A(t)}{\mathrm{d}t}\;C\right) = -2\text{Tr}\left(\left[B(t),A(t)\right] C\right)
\end{equation}
Using the cyclic property of trace, this becomes
\begin{equation}\begin{aligned}
	\text{Tr}\left(\left[B(t),A(t)\right] C\right) &= \text{Tr}\left(B(t)A(t) C - A(t) B(t)C\right)\\
						       &= \text{Tr}\left(B(t)A(t) C - B(t)A(t) C\right)\\
						       &= \text{Tr}\left(B(t)\left[A(t), C\right]\right)\\
\end{aligned}\end{equation}
If we now substitute the choice of \(B(t)\) we made above, we get
\begin{equation}
	\frac{\mathrm{d} \chi}{\mathrm{d}t} = -2\text{Tr}\left(\left[A(t), C\right]^2\right) \leq 0
\end{equation}
Since \(\chi\), the way it is defined, must necessarily be positive semi-definite for all \(t\), the derivative \(\frac{\mathrm{d} \chi}{\mathrm{d}t}\) must vanish in the limit \(t \to \infty\), otherwise \(\chi(t)\) will become negative. This gives the result
\begin{equation}
	\lim_{t\to \infty}\frac{\mathrm{d} \chi}{\mathrm{d}t} = -2 \lim_{t\to \infty}\text{Tr}\left(\left[A(t), C\right]^2\right) = 0 \implies \lim_{t\to \infty}\left[A(t), C\right] = 0
\end{equation}
In other words, \textbf{the lax pair evolution of \(A(t)\) against \(\left[A(t),C\right]\) leads to the diagonalization of \(A(t)\) with respect to \(C\).} This can be used as an iterative algorithm to diagonalize a general matrix with respect to another matrix:
\begin{itemize}
	\item Define matrices A and B, A being the one we want to diagonalize w.r.t B
	\item Iteratively run the next two steps until a desired accuracy is reached
	\item Compute a new matrix C = A*B - B*A
	\item Change A as follows: A = A + C*A - A*c
\end{itemize}

\end{document}

