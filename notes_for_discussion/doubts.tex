\documentclass[12pt,aspectratio=169]{beamer}
\setbeamertemplate{footline}[frame number]{}
\setbeamertemplate{navigation symbols}{}
\setbeamertemplate{footline}{}
\setbeamersize{text margin left=5mm,text margin right=5mm}
\usepackage{xcolor,helvet}
\definecolor{maroon}{rgb}{0.6,0,0}
\definecolor{bottlegreen}{rgb}{0,0.3,0}
\usepackage{graphicx}
\usetheme{boxes}
\setbeamertemplate{frametitle}{\small\insertframetitle}
%\linespread{1.2}
%\usefonttheme{serif}     % Font theme: serif
%\usepackage{ccfonts}     % Font family: Concrete Math
%\usepackage[T1]{fontenc} % Font encoding: T1
\newcommand{\qs}[1]{\textbf{\textcolor{bottlegreen}{#1}}}
\usepackage[absolute,overlay]{textpos}

\begin{document}

\begin{frame}{Question 1}
	Section 2.2, Equation \textcolor{blue}{2.18} of thesis
\begin{equation*}
\begin{gathered}
	\frac{1}{H^\prime - H_e\hat n_N}c^\dagger_N T	= c^\dagger_N T\frac{1}{H^\prime - H_h(1-\hat n_N)}\\
	\implies \textcolor{blue}{H_e\hat n_Nc^\dagger_N T = c^\dagger_N TH_h(1-\hat n_N)}
\end{gathered}
\end{equation*}
This seems to \textbf{require \(H^\prime\) commuting with \(T\)}, because
\begin{equation*}
	\textcolor{red}{c^\dagger_N TH^\prime} - \textcolor{blue}{c^\dagger_N TH_h(1-\hat n_N)}= \textcolor{red}{H^\prime c^\dagger_N T} - \textcolor{blue}{H_e\hat n_Nc^\dagger_N T}
\end{equation*}
\qs{Why should $H^\prime$ commute with $T$?}
\begin{equation*}
	(\text{where } H_e = Tr\left(H\hat n_N\right) \text{, } H_h = Tr\left[H\left(1-\hat n_N\right)\right] \text{ and }T = Tr\left(Hc_N\right))
\end{equation*}
\end{frame}

\begin{frame}{Question 2}
	Section 2.2, Equation \textcolor{blue}{2.19} of thesis
\begin{equation*}
	\eta_N H \eta_N^\dagger = H_h \left(1 - n_N\right)
\end{equation*}
If I try to derive this using the result on the previous slide:
\begin{equation*}
\begin{aligned}
	\eta H \eta^\dagger = \eta H_e \eta^\dagger = \eta \textcolor{blue}{H_e c^\dagger T} G &= \eta \textcolor{blue}{c^\dagger T H_h} G \\
											       &= \eta c^\dagger T \textcolor{red}{G H_h} = \eta \eta^\dagger H_h = H_h \left(1 - \hat n\right)
\end{aligned}
\end{equation*}
\qs{That required $\left[G,H_h\right]=0$. How does that work out?}
\begin{equation*}
	(\text{where } H_e = Tr\left(H\hat n_N\right) \text{, } H_h = Tr\left[H\left(1-\hat n_N\right)\right] \text{ and }T = Tr\left(Hc_N\right))
\end{equation*}
\end{frame}

\begin{frame}{Question 3}
	\only<1>{Kondo Model appendix, Equation \textcolor{blue}{9.61} of thesis}
\begin{equation*}
\begin{aligned}
&\Delta\hat{H}_{(j)} = \sum_{\substack{m=1,\\ \beta=\uparrow/\downarrow}}^{n_{j}}\frac{(J^{(j)})^2}{\textcolor{red}{2}}\frac{\textcolor{red}{\tau_{j,\hat{s}_{m},\beta}}}{(2\omega\tau_{j,\hat{s}_{m},\beta} - \epsilon_{j,l}\tau_{j,\hat{s}_{m},\beta}-J^{(j)}S^{z}s^{z}_{j,\hat{s}_{m}})}\nonumber\\
&\times\bigg[S^{a}S^{b}\sigma^{a}_{\alpha\beta}\sigma^{b}_{\beta\gamma} \sum_{\substack{(j_{1},j_{2}< j),\\ n,o}}c^{\dagger}_{j_{1},\hat{s}_{n},\alpha}c_{j_{2},\hat{s}_{o},\gamma}(1-\hat{n}_{j,\hat{s}_{m},\beta})+...%S^{b}S^{a}\sigma^{b}_{\beta\gamma}\sigma^{a}_{\alpha\beta} \sum_{\substack{(j_{1},j_{2}<j),\\ n,o}}c_{j_{2},\hat{s}_{o},\gamma}c^{\dagger}_{j_{1},\hat{s}_{n},\alpha}\hat{n}_{j,\hat{s}_{m},\beta}\bigg]\nonumber
\\
&+\sum_{\substack{m=1,\\ \beta=\uparrow/\downarrow}}^{n_{j}}\frac{(J^{(j)})^{2}}{2(2\omega\tau_{j,\hat{s}_{m},\beta} - \epsilon_{j,l}\tau_{j,\hat{s}_{m},\beta}-J^{(j)}S^{z}s^{z}_{j,\hat{s}_{m}})} \bigg[S^{x}S^{y}\sigma^{x}_{\alpha\beta}\sigma^{y}_{\beta\alpha}c^{\dagger}_{j,\hat{s}_{m},\alpha}c_{j,\hat{s}_{m},\beta}c^{\dagger}_{j,\hat{s}_{m},\beta}c_{j,\hat{s}_{m},\alpha}\nonumber + ...%S^{y}S^{x}\sigma^{x}_{\alpha\beta}\sigma^{y}_{\beta\alpha}c^{\dagger}_{j,\hat{s}_{m},\beta}...%c_{j,\hat{s}_{m},\alpha}c^{\dagger}_{j,\hat{s}_{m},\alpha}c_{j,\hat{s}_{m},\beta}\bigg]~.
\end{aligned}
\end{equation*}
\only<2-5>{\begin{itemize}
		\only<2>{\item \qs{The \(\tau\) should \textcolor{red}{not} be there in numerator i presume?}}
		\only<3>{\item \qs{Since coupling is \(\pmb{\frac{J}{2}}\), shouldn't the thing be \textcolor{red}{\(\pmb{\frac{J^2}{4}}\)} instead of \textcolor{red}{\(\pmb{\frac{J^2}{2}}\)}?}}
		\only<4>{\item \qs{You mentioned the following in the google document- "\textit{interchange sigma\_a and sigma\_b (you get -1 sign)}". But these are matrix elements (numbers). So \textcolor{red}{why the minus sign}?}}
		\only<5>{\item \qs{How do you combine the product of two sigmas ( \textcolor{red}{\(\pmb{\sigma^a_{\alpha\beta}\sigma^b_{\beta\gamma}}\)} ) into a single \textcolor{red}{\(\pmb{\sigma^c_{\alpha\gamma}}\)}?}}
\end{itemize}}
\end{frame}
\begin{frame}{Question 4}
Kondo URG coupling equation for \(J\) (equation 9.65):
\begin{equation*}
\begin{aligned}
	\Delta J^{(j)}=n_{j}(J^{(j)})^{2}\left[\omega- \frac{\epsilon_{j,l}}{2}\right]\left[(\frac{\epsilon_{j,l}}{2}-\omega)^{2}-\frac{\left(J^{(j)}\right)^{2}}{16}\right]^{-1}
\end{aligned}
\end{equation*}
One-loop form (after setting \(\omega = \epsilon_{j,l}\)):
\begin{equation*}
\begin{aligned}
	\Delta J^{(j)}= \frac{n_{j}(J^{(j)})^{2}}{\omega-\frac{\epsilon_{j,l}}{2}} = 2\frac{n_{j}(J^{(j)})^{2}}{\epsilon_{j,l}} \rightarrow \textcolor{blue}{\frac{\textcolor{red}{2} \rho |\Delta D| J^2}{D}} && [n_j, \rho \rightarrow \text{DOS per spin}]
\end{aligned}
\end{equation*}
One-loop form in Coleman (Introduction to Many-Body Physics) (\(\tilde J = J/2\)):
\begin{equation*}
\begin{aligned}
	\Delta \tilde J = \frac{2 \rho |\Delta D| \tilde J^2}{D} \implies \Delta J = \textcolor{blue}{\frac{\rho |\Delta D| J^2}{D}}
\end{aligned}
\end{equation*}
\qs{Is there any reason for this difference?}
\end{frame}

\begin{frame}{Question 5}
	\begin{itemize}
		\item In the Kondo URG, are you considering \textcolor{red}{two electrons} on the shell \(\Lambda_N\), one that we are decoupling (\(q\beta\)) and another with the same momentum but \textcolor{red}{opposite spin} (\(q\overline\beta\))? \\[30pt]
		\item If so, why does that kinetic energy piece (\(\epsilon_q \tau_{q\overline\beta}\)) not come down in the denominator?\\[30pt]
		\item Is that what gives rise to the second RG equation and hence the \textcolor{red}{\(S^z s^z\) term} in the effective Hamiltonian?
	\end{itemize}
\end{frame}

\begin{frame}{Question 6}
\begin{equation*}
\begin{aligned}
	\Delta H^{2}_{(j)}&=\sum_{\substack{m=1,\\ \beta=\uparrow/\downarrow}}^{n_{j}}\frac{(J^{(j)})^{2}}{(2\omega\tau_{j,\hat{s}_{m},\beta} - \epsilon_{j,l}\tau_{j,\hat{s}_{m},\beta}-J^{(j)}S^{z}s^{z}_{j,\hat{s}_{m}})}\bigg[\textcolor{blue}{S^{x}S^{y}\sigma^{x}_{\alpha\beta}\sigma^{y}_{\beta\alpha}c^{\dagger}_{j,\hat{s}_{m},\alpha}c_{j,\hat{s}_{m},\beta}c^{\dagger}_{j,\hat{s}_{m},\beta}c_{j,\hat{s}_{m},\alpha}}\nonumber\\
				   &+S^{y}S^{x}\sigma^{x}_{\alpha\beta}\sigma^{y}_{\beta\alpha}c^{\dagger}_{j,\hat{s}_{m},\beta}c_{j,\hat{s}_{m},\alpha}c^{\dagger}_{j,\hat{s}_{m},\alpha}c_{j,\hat{s}_{m},\beta}\bigg]\nonumber\\
				   &=\sum_{\substack{m=1,\\ \beta}}^{n_{j}}\frac{(J^{(j)})^{2}}{(2\omega\tau_{j,\hat{s}_{m},\beta} - \epsilon_{j,l}\tau_{j,\hat{s}_{m},\beta}-J^{(j)}S^{z}s^{z}_{j,\hat{s}_{m}})}\textcolor{blue}{S^{z}\frac{\sigma^{z}_{\alpha\alpha}}{2}\bigg[\hat{n}_{j,\hat{s}_{m},\alpha}(1-\hat{n}_{j,\hat{s}_{m},\beta})}-\only<2>{...}
	\only<1>{\\&\quad\quad\quad\quad\quad\quad\quad\quad\quad\quad\quad\quad\quad\quad\quad\quad\quad\quad\quad\quad\quad\quad\quad\hat{n}_{j,\hat{s}_{m},\beta}(1-\hat{n}_{j,\hat{s}_{m},\alpha})\bigg]\nonumber\\}
\end{aligned}
\end{equation*}
\only<2>{\qs{What I got:}
\begin{equation*}
	S^{x}S^{y}\sigma^{x}_{\alpha\beta}\sigma^{y}_{\beta\alpha}c^{\dagger}_{j,\hat{s}_{m},\alpha}c_{j,\hat{s}_{m},\beta}c^{\dagger}_{j,\hat{s}_{m},\beta}c_{j,\hat{s}_{m},\alpha} = \textcolor{red}{i^2} S^z\sigma^{z}_{\alpha\alpha}\hat n_{j,\hat{s}_{m},\alpha}\left(1 - \hat n_{j,\hat{s}_{m},\beta}\right)
\end{equation*}}
\end{frame}
\begin{frame}{Question 7}
In eq. 2.21 of thesis,
\begin{equation*}
	U H U^\dagger = \frac{1}{2}Tr(H) + \tau Tr(H \tau) + \textcolor{blue}{\tau\{c^\dagger T,\eta\}}
\end{equation*}
so the renormalization is
\begin{equation*}
	\tau\{c^\dagger T,\eta\} = \frac{1}{2}\left[\overbrace{c^\dagger T \eta}^\text{particle sector} - \underbrace{\eta c^\dagger T}_\text{hole sector}\right] = \text{difference of the 2 sectors}
\end{equation*}
\qs{Yet in most RG equations (\(\Delta H_F\) of 2d Hubbard, \(\Delta H_j\) of Kondo), you have \textit{added} the two sectors. How/Why?}
\end{frame}
\begin{frame}{Question 8}
	In the Kondo URG, you simplify the \(\hat \omega\) as 
\begin{equation*}
	\pmb{\hat \omega = \textcolor{blue}{\omega \tau}}
\end{equation*}
\qs{What is the formal way of doing this?} Shouldn't it be 
\begin{equation*}
	\pmb{\hat \omega = \omega_1 \hat n + \omega_1 (1 - \hat n)}
\end{equation*}
Is this just an assumption?\\[10pt]
In the RG equation for BCS instability (eq. 8.130 of thesis), you use 
\begin{equation*}
	\pmb{G^{-1} = \omega - \epsilon_1 \tau_1 - \epsilon_2 \tau_2}
\end{equation*}
\qs{How is this choice of \(\hat \omega\) consistent with what was done in Kondo URG?}

\end{frame}

\end{document}
