\chapter{Conclusions and Future Directions}
In this work, we have seen that the fate of the SIAM at low temperatures is a growth in the spin, charge and hybridisation couplings \(J,K\) and \(V\). This confirms the NRG and Bethe ansatz result that the low energy behavior is that of a screened impurity. We stress that the spin or the charge exchange coupling, \(J\) or \(K\), is necessary to screen the corresponding degree of freedom on the impurity. The ground state wavefunction for the low energy theory is primarily a spin-singlet with some charge-triplet content in  the spin-Kondo regime.  In the charge-Kondo regime, we find a charge-singlet instead. The susceptibility is seen to increase as we go to lower temperatures, signaling increased spin-flip scattering and providing a justification for the increase in resistivity after the Kondo temperature \(T_K\). At high temperatures the susceptibility saturates to \(\frac{1}{8}\), mimicking a four-fold degenerate object because at high temperatures, the impurity energy scales become blurred. We encountered a maximum in the susceptibility vs temperature plot which is not present in NRG or Bethe Ansatz plots, and we believe it can be attributed to the approximation made in working with a zero mode Hamiltonian. We also calculated the specific heat, and it shows two peaks.

We also obtain a Hamiltonian for the Kondo cloud (the set of electrons which screen the impurity) - it has a Fermi liquid number diagonal term as well as an off-diagonal non-Fermi liquid piece. It is this off-diagonal piece that is responsible for the strong entanglement between the impurity and the conduction electrons. This contribution is found to increase as we move towards the infrared  fixed point. Higher order scattering processes will give more complicated terms in the effective Hamiltonian. We also calculate information-theoretic quantities like mutual information between impurity and conduction electrons. They show that the conduction electrons become more strongly-entangled as we move towards the infrared fixed point. They also reveal an increase in off-diagonal correlations between conduction electrons which is indicative of an increase in the off-diagonal scattering content in the effective Hamiltonian of the Kondo cloud. We also find the value of the zero-temperature Wilson ratio to be 2 in the Kondo regime. Closely related to this is the result that the Luttinger's volume at the fixed point accommodates the impurity site into the Fermi volume such that Luttinger's count increases by 1 compared to the value at the free-orbital or local moment fixed points. The renormalization of the impurity spectral across the RG shows the emergence of the central resonance at the strong-coupling fixed point.

This work still leaves some problems untouched. Some new questions and interesting prospects have also emerged. We discuss them now.
\begin{itemize}
	\item Can we obtain an analytical expression for the finite temperature Wilson ratio? The zero temperature value was obtained using low temperature approximations, so a finite temperature calculation would need to be non-perturbative. In other words, the full effect of the conduction bath will most probably come into play.
	\item The spectral function of the impurity is another interesting quantity to explore. It is not yet completely certain which parts of the effective Hamiltonian for the cloud lead to which features of the spectral function. It would be interesting to check whether the Fermi liquid part gives the central peak and the non-Fermi liquid the side humps, or something else altogether. We also intend to explore the bath spectral function and its variation under the RG.
	\item In Chapter \ref{urg_canonical}, we find that the URG renormalization in Hamiltonian has a form similar to that of URG - a generalized double-bracket form. This suggests that URG might have uses in linear algebra as well. It is well-known that transformations of the double-bracket form are unitary and can be used in minimization and sorting problems \cite{brockett_1991}. It has also been shown that the unitary flows correspond to motion along a geodesic if a suitable manifold is defined. This suggests that we can look for a quantity which is minimized by URG in the journey towards the fixed point and see if it can be used to provide a faster and more robust algorithm for optimization problems.
	\item One can also look into the lattice versions of these models - the Kondo lattice and the Anderson lattice problems which involve two interacting bands - one for the impurity and the other for the conduction electrons. Since there is a macroscopic number of impurity electrons, such a model shows phase transitions. One big project for the future can be to look into such models, map the phase diagram and search for superconducting phases or non-Fermi liquid phases.
	\item We have seen that the zero-mode of the low-energy effective Hamiltonian is essentially a two-site Anderson model (a.k.a Anderson dimer), which can be solved exactly. An interesting endeavor in this context could be to see if there is a unitary transformation that converts the Anderson dimer to a two-site Hubbard model (Hubbard dimer). The point of this exercise would be to shed some light on the inner workings of DMFT. The imposition of the self-consistency requirement in DMFT introduces a translational invariance into the Anderson model and converts it into the Hubbard model. If we could figure out a transformation that does the same (albeit, between dimers and not the full-fledged models), it might lead to new insights into DMFT). This might allows us to see a metal-insulator phase transition. With the single-impurity Anderson model, it is not possible to see such a transition because the impurity can always tunnel into the conduction bath through the origin of the lattice. The central peak of the spectral disappears only at \(U \to \infty\). However, if one introduces correlation into the bath, the impurity electron electron will find it harder to disperse, leading to the possibility that we end up with an insulator at some critical value of the onsite repulsion. An interesting project could be to redo the URG of the SIAM with more complicated baths (by inserting non-trivial self-energies into the bath), and see if the phase-diagram changes. As it stands now, there is no provision for a phase transition in the SIAM because there is only one stable phase. If the inclusion of a non-trivial bath brings about a phase-transition in the impurity phase diagram, that is essentially a mirror of the phase transition happening in the Hubbard model. The importance of this potential discovery is that the self-energies that we wish to insert into the bath have been obtained from a URG treatment of the Hubbard model, and the appearance of the phase transition would then show that URG can be used to do much of the work that DMFT does, and it provides an avenue for improving/remodeling DMFT.
\end{itemize}
