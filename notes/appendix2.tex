\chapter{Derivation of RG equations for the impurity model with \(U,V,J,K,U_b,\eta\)}
\label{app-derivation}

\section{The Hamiltonian}

The Hamiltonian for the impurity problem is
\begin{equation}\begin{aligned}
	\mathcal{H} = \sum_{k\sigma}\epsilon_k \tau_{k\sigma} - \frac{1}{2} U \left( \tau_{d \uparrow} - \tau_{d \downarrow} \right) ^2 + \frac{1}{2}\eta \left(\tau_{d \uparrow} + \tau_{d \downarrow}\right) + \sum_{k\sigma} \left(V_{k} c^\dagger_{k\sigma} c_{d\sigma} + h.c.\right) +J \vec{S_d}\cdot\vec{s} + K \vec{C}_d\cdot\vec{C} - U_b\left(\tau_{0 \uparrow} - \tau_{0 \downarrow}\right)^2 
\end{aligned}\end{equation}
wher \(\tau \equiv \hat n -\frac{1}{2}\). The first term is the kinetic energy term with a dispersion \(\epsilon_k\). The second term is the particle-hole symmetric impurity correlation term that sets the uncorrelated half-filled level \(\tau_{d\uparrow} = - \tau_{d \downarrow} = \pm \frac{1}{2}\) at energy \(-U/2\) below the Fermi surface, and the correlated doublon and holon states \(\tau_{d\downarrow} = \tau_{d \uparrow} = \pm \frac{1}{2}\) at the Fermi surface. The particle-hole symmetry is in the fact that the term is invariant under \(\tau_{d\sigma} \to-\tau_{d\sigma}\), and manifests in the symmetric positioning of the doublon and holon levels with respect to each other. The third term is the particle-hole asymmetry term that explicitly measures the deviation away from particle-hole symmetry - it is non-zero only when we are away from half-filling. The fourth, fifth and sixth terms describe various mechanisms in which the impurity can hybridise with the conduction bath. The first of these is a single-particle hopping \(V\), the second mechanism is through spin-flip scattering, and the third is through isospin-flip scattering. The single-particle hybridisation \(V\) allows the impurity electron to tunnel into the conduction bath and this leads to a frequency-dependent self-energy renormalisation of the impurity, implying that the impurity electron can spend  large amount of time in the bath~\cite{coleman2015}. \(\vec S_d = \frac{1}{2}\sum_{\alpha\beta}\vec \sigma_{\alpha\beta}c^\dagger_{d\alpha}c_{d\beta} \) is the impurity spin, while \(\vec C_d = \frac{1}{2}\sum_{\alpha\beta} \psi^\dagger_\alpha \vec \sigma_{\alpha\beta}\psi_\beta\) is the impurity isospin, where \(\psi = \left(c_{d\uparrow} ~ ~ ~ ~ c^\dagger_{d \downarrow}\right)\)~
\cite{anderson1958random,nambu_1960}. The former acts on the \(2-\)dimensional Hilbert-space formed by the magnetic up and down states on the impurity, while the latter acts on the \(2-\)dimensional space formed by the doublon and holon states. The former reacts to a magnetic field, while the latter responds to a chemical potential. Such a chemical potential term that couples with the isospin is already present in the Hamiltonian - it is the assymetry term \(\eta\). In fact, that term can be rewritten as \(\eta C_d^z\), because \(C^z_d = \frac{1}{2}\left(\hat n_{d \uparrow} + \hat n_{d \downarrow} - 1\right) = \frac{1}{2}\left( \tau_{d \uparrow} + \tau_{d \downarrow} \right) \). The bath operators \(\vec s\) and \(\vec C\) are defined in the same way as the operator counterparts: \(\vec s = \sum_{kk^\prime\alpha\beta}\vec \sigma_{\alpha\beta}c^\dagger_{k\alpha}c_{k^\prime\beta}, \vec C = \sum_{kk^\prime\alpha\beta}{\psi^k}^\dagger_\alpha\vec\sigma_{\alpha\beta}\psi^{k^\prime}_\beta\), where \(\psi^k = \left(c_{k\alpha} ~ ~ ~ ~ c^\dagger_{k^\prime\beta}\right) \). The final seventh term is a local particle-hole symmetric correlation on the zeroth site of the bath. For \(U_b>0\), this term promotes half-filling on the zeroth site, while for \(U_b<0\), a doublon or holon is more favourable on that site.

We first Fourier transform the \(U_b\)-term to \(k-\)space. In \(k-\)space, the diagonal contribution (to \(H_D\)) coming from this term is the single-particle self-energy \(-U_b\left(\hat n_{q \beta}\right)^2\) which can be made particle-hole symmetric in the form:
\begin{equation}\begin{aligned}
	-U_b\left(\tau_{q \beta}\right)^2
\end{aligned}\end{equation}
where \(q\) is the \(k-\)state being decoupled and \(\tau \equiv \hat n - 1/2\). In the initial state \(\ket{\Psi}_i\), we have \(\langle \hat n_{q\beta} \rangle = 1/2 \implies \tau_{q\beta} = 0\), so the contribution of \(U_b\) to that state is 0. For both hole excitations \(c_{q\beta}\ket{\Psi}_i\) as well as particle excitations \(c^\dagger_{q\beta}\ket{\Psi}_i\), the intermediate state energy lowers to \(-U_b/4\).

The off-diagonal part is
\begin{equation}\begin{aligned}
	-\frac{U_b}{2}\sum_{kk^\prime\sigma}c^\dagger_{k\sigma}c_{k^\prime\sigma} + U_b \sum_{k_1,k_2,k_1^\prime,k_2^\prime} c^\dagger_{k_1 \uparrow}c_{k_2 \uparrow} c^\dagger_{k^\prime_1 \downarrow}c_{k^\prime_2 \downarrow} 
\end{aligned}\end{equation}
We ignore the potential scattering arising from the first term.

\section{In the absence of \(U_b\)}
\subsection{Renormalisation of the impurity energy levels}
A single site has four degrees of freedom arising from the spin-degeneracy, and can be represented by the Hamiltonian
\begin{equation}\begin{aligned}
	H = \epsilon_0 \ket{0}\bra{0} + \epsilon_1 \sum_\sigma\ket{\sigma}\bra{\sigma} + \epsilon_2 \ket{2}\bra{2}
\end{aligned}\end{equation}
\(\ket{0},\ket{2}\) are the holon and doublon states, while \(\ket{\sigma}\) is the state where the site is singly-occupied by a particle of spin \(\sigma\). We will obtain the renormalisation to the energy levels \(\epsilon_0,\epsilon_1\) and \(\epsilon_2\), and thereby calculate the renormalisation in \(U\) and \(\eta\). This can be done by noting that \(\epsilon_1 = -\frac{U}{2}\), and \(\epsilon_2 = \eta\). A qualifier is, however, needed here. In the bare Hamiltonian, the energies of the doublon and holon states are negatives of each other, and this would correspond to requiring \(\epsilon_2 = -\epsilon_0\). This may not hold true under renormalisation, and the way we account for this is that at each step of the RG, we subtract a constant energy from the Hamiltonian so as to set the energy of the holon state to 0. This means that the energy of the singly-occupied and doublon states become \(\epsilon_1 - \epsilon_0\) and \(\epsilon_2 - \epsilon_0\). And since we have identified \(\epsilon_1\) and \(\epsilon_2\) as \(-\frac{U}{2}\) and \(\eta\) respectively, we can use \(\Delta U = -2\left(\Delta \epsilon_1 - \Delta \epsilon_0\right), \Delta \eta = \Delta \epsilon_2 - \Delta \epsilon_0 \).

We need to look at three kinds of scattering vertices here: \(V^2\), \(J^2\) and \(K^2\). We will consider these processes one after another. We define \(n_j\) as the number of states being decoupled on each side of the Fermi surface, at the \(j^\text{th}\) RG step. In order to treat both spin and isospin exchanges democratically, we take \(\ket{\Psi}_i = \frac{1}{2}\left(\ket{0} + \ket{q\uparrow} + \ket{q\downarrow} + \ket{q\uparrow, q\downarrow}\right) \) as the \textit{initial} state for the scattering processes. The intermediate states \(\ket{\Psi}_\text{int}\) in the particle sector \(\left(c_{q\beta}\ket{\Psi}_i\right)\) and hole sector \(\left(c^\dagger_{q\beta}\ket{\Psi}_i\right)\) will then have both spin and isospin excitations which can couple with the corresponding impurity degree of freedom. We will assume that states  \(q > k_F~\left(\epsilon_q > 0\right) \) above the Fermi surface can have only particle excitations and states below the Fermi surface can only have hole excitations. The kinetic energy part \(\epsilon_q \tau_{q\beta}\) of \(H_D\) for \(\ket{\Psi}_i\) is then zero, whereas it is always \(D/2\) for \(\ket{\Psi}_\text{int}\). To demonstrate this for a typical \(q < k_F\), the hole excitation is \(c_{q \uparrow}\ket{\Psi}_i = \frac{1}{\sqrt 2}\left(\ket{0} + \ket{q \downarrow}\right)\). This has an isospin term in the form of the \textit{holon} and a spin term in the form of the down state. Since \(\tau_{q \uparrow} = -\frac{1}{2}\) in the excited state, the kinetic energy for \(\ket{\Psi}_\text{int}\) is \(\epsilon_q \tau_{q \uparrow} = \left(-D\right)\times\left(-\frac{1}{2}\right) = D/2\).

The renormalisation arising from the first kind of terms, in the particle sector, is
\begin{equation}\begin{aligned}
	\sum_{q\beta}c^\dagger_{q\beta}c_{d\beta}\frac{V^2}{\omega - H_D}c^\dagger_{d\beta}c_{q\beta} = \sum_{q\beta}V^2 \hat n_{q\beta} \left( 1 - \hat n_{d\beta} \right)\left( \frac{1-\hat n_{d \overline\beta }}{\omega - E_0} + \frac{\hat n_{d \overline\beta}}{\omega^\prime - E_1}\right) = V^2 n_j\sum_{\beta}\left( 1 - \hat n_{d\beta} \right)\left( \frac{1-\hat n_{d \overline\beta }}{\omega_0 - E_0} + \frac{\hat n_{d \overline\beta}}{\omega_1 - E_1}\right)
\end{aligned}\end{equation}
\(q\) runs over the momentum states that are being decoupled at this RG step: \(|q| = \Lambda_j\). \(E_{1,0}\) are the diagonal parts of the Hamiltonian at \(\hat n_{d\overline \beta}=1,0\) respectively. We have \(\hat n_{d\beta}=1\) in the intermediate state because of the \(c^\dagger_{d\beta}\) in front of the Greens function. Applying \(c_{q\beta}\) on the initial state \(\ket{\Psi}_i\) leaves us with \(C^z_q = - \frac{1}{2}\) and \(s^z_q = \frac{1}{2}\overline\beta\). We also know that
\begin{equation}\begin{aligned}
	\hat n_{d\beta}=1,
	\begin{cases}
		\hat n_{d\overline\beta}=0 &\implies S_d^z = \frac{1}{2}\beta, C_d^z = 0, \epsilon_d\left(\hat n_{d\uparrow} - \hat n_{d \downarrow}\right)^2 = \epsilon_d\\	
		\hat n_{d\overline\beta}=1 &\implies S_d^z = 0, C_d^z = \frac{1}{2}, \epsilon_d\left(\hat n_{d\uparrow} - \hat n_{d \downarrow}\right)^2 = 0
	\end{cases}
\end{aligned}\end{equation}
Combining all this, we can write \(E_1 = \frac{D}{2} - \frac{K}{4}\) and \(E_0 = \frac{D}{2} + \epsilon_d - \frac{J}{4}\). In order to relate \(\omega_0\) with \(\omega_1\) with the common fluctuation scale \(\omega\) for the conduction electrons, we will replace these quantum fluctuation scales by the current renormalised values of the single-particle self-energy for the initial state from which we started scattering. For \(\hat n_{d\overline\beta}=0\), there is no additional self-energy because the impurity does not have any spin: \(\omega_0 = \omega\). For \(\hat n_{d\overline\beta} = 1\), we have an additional self-energy of \(\epsilon_d\) arising from the correlation on the impurity: \(\omega_1 = \omega + \epsilon_d\).
Substituting the values of \(E_{0,1}\) and \(\omega_{0,1}\), we get
\begin{equation}\begin{aligned}
	\label{ren_ed_Vp}
	V^2 n_j\sum_{\beta}\left( 1 - \hat n_{d\beta} \right)\left( \frac{1-\hat n_{d \overline\beta }}{\omega - \frac{D}{2} - \epsilon_d + \frac{J}{4}} + \frac{\hat n_{d \overline\beta}}{\omega - \frac{D}{2} + \epsilon_d + \frac{K}{4}}\right)
\end{aligned}\end{equation}
Performing a similar calculation for the hole sector gives the contribution:
\begin{equation}\begin{aligned}
	\label{ren_ed_Vh}
	V^2 n_j\sum_{\beta}\hat n_{d\beta}\left( \frac{1-\hat n_{d \overline\beta }}{\omega - \frac{D}{2} + \epsilon_d + \frac{K}{4}} + \frac{\hat n_{d \overline\beta}}{\omega - \frac{D}{2} - \epsilon_d + \frac{J}{4}}\right)
\end{aligned}\end{equation}
We now come to the second type of terms: spin-spin. We first look at the particle sector:
\begin{equation}\begin{aligned}
	\label{ren_ed_Jpp}
	\frac{J^2}{4}\sum_{q\beta}c^\dagger_{d\overline\beta}c_{d\beta}c^\dagger_{q\beta}c_{-q\overline\beta} \frac{1}{\omega - H_D}c^\dagger_{d\beta}c_{d\overline\beta}c^\dagger_{q\overline\beta}c_{q\beta} = \frac{J^2}{4} n_j\frac{1}{\omega - \frac{D}{2} + \frac{J}{4}} \sum_{\beta}\hat n_{d\overline\beta}\left( 1 - \hat n_{d\beta} \right)
\end{aligned}\end{equation}
The diagonal part in the denominator was simple to deduce in this case, because the nature of the scattering requires the spins \(S_d^z\) and \(\frac{\beta}{2}\left(\hat n_{q\beta} - \hat n_{q \overline\beta}\right)\) to be anti-parallel. This ensures that the intermediate state has an energy of \(E = \frac{D}{2} + \epsilon_d - \frac{J}{4}\), and the quantum fluctuation scale is \(\omega^\prime = \omega + \epsilon_d\), such that \(\omega^\prime - E = \omega - \frac{D}{2} + \frac{J}{4}\). In the hole sector, we have
\begin{equation}\begin{aligned}
	\label{ren_ed_Jph}
	\frac{J^2}{4} n_j\frac{1}{\omega - \frac{D}{2} + \frac{J}{4}} \sum_{\beta}\hat n_{d\beta}\left( 1 - \hat n_{d\overline\beta} \right)
\end{aligned}\end{equation}
The final kind of scattering is the \(K^2\) type. Similar to the \(J^2\) term, we get the following contribution:
\begin{equation}\begin{aligned}
	\label{ren_ed_Kpp}
	\frac{K^2}{4}\sum_{q\beta}c^\dagger_{q\beta}c^\dagger_{q\overline\beta}c_{d\overline\beta}c_{d\beta} \frac{1}{\omega - H_D}c^\dagger_{d\beta}c^\dagger_{d\overline\beta}c_{q\overline\beta}c_{q\beta} = \frac{K^2}{2} n_j\frac{1}{\omega - \frac{D}{2} + \frac{K}{4}} \left(1 - \hat n_{d \uparrow}\right) \left( 1 - \hat n_{d \downarrow} \right)
\end{aligned}\end{equation}
in the particle sector. This is again because \(E = \frac{D}{2} - \frac{K}{4}\) in the intermediate state and \(\omega^\prime = \omega\). In the hole sector, we get
\begin{equation}\begin{aligned}
	\label{ren_ed_Kph}
	\frac{K^2}{2} n_j\frac{1}{\omega - \frac{D}{2} + \frac{K}{4}} \hat n_{d\uparrow}\hat n_{d \downarrow}~.
\end{aligned}\end{equation}

We now have all possible renormalisation to the impurity energy \(\epsilon_d\). To actually compute the renormalisation, we will first calculate the renormalisation in the energies \(\epsilon_0, \epsilon_1\) and \(\epsilon_2\) of the impurity states \(\ket{\hat n_d = 0}, \ket{\hat n_d = 1}, \ket{\hat n_d = 2}\) respectively. The renormalisation of these states are given by the following terms:
\begin{itemize}
	\item \(\Delta \epsilon_0\) is given by the renormalisation of the term \(\left(1 - \hat n_{d\uparrow}\right)\left(1 - \hat n_{d \downarrow}\right)\)
	\item \(\Delta \epsilon_1\) is given by the renormalisation of either \(\left(1 - \hat n_{d\uparrow}\right)\hat n_{d \downarrow}\) or \(\left(1 - \hat n_{d\downarrow}\right)\hat n_{d \uparrow}\)
	\item \(\Delta \epsilon_2\) is given by the renormalisation of \(\hat n_{d\uparrow}\hat n_{d \downarrow}\)
\end{itemize}
From eqs.~\ref{ren_ed_Vp}, \ref{ren_ed_Vh}, \ref{ren_ed_Jpp}, \ref{ren_ed_Jph}, \ref{ren_ed_Kpp} and \ref{ren_ed_Kph}, we write
\begin{equation}\begin{aligned}
	\Delta \epsilon_0 = \Delta \epsilon_2 = \frac{2V^2 n_j}{\omega - \frac{D}{2} - \epsilon_d + \frac{J}{4}} + \frac{K^2 n_j/2}{\omega - \frac{D}{2} + \frac{K}{4}}, && \Delta \epsilon_1 = \frac{2V^2 n_j}{\omega - \frac{D}{2} + \epsilon_d + \frac{K}{4}} + \frac{J^2 n_j/2}{\omega - \frac{D}{2} + \frac{J}{4}}
\end{aligned}\end{equation}
We had started with a particle-hole symmetric Hamiltonian \((2\epsilon_d + U = 0)\); the fact that \(\Delta \epsilon_0 = \Delta \epsilon_2\) means the RG transformation has preserved that symmetry. The renormalisation of \(\epsilon_d\) is simply the renormalisation in the energy difference between the singly-occupied and vacant impurity levels: \(\Delta \epsilon_d = \Delta \epsilon_1 - \Delta \epsilon_0\). This gives our first RG equation:
\begin{equation}\begin{aligned}
	\Delta \epsilon_d = 2V^2 n_j\left(\frac{1}{\omega - \frac{D}{2} + \epsilon_d + \frac{K}{4}} - \frac{1}{\omega - \frac{D}{2} - \epsilon_d + \frac{J}{4}}\right) + \frac{n_j}{2}\left(\frac{J^2}{\omega - \frac{D}{2} + \frac{J}{4}} - \frac{K^2}{\omega - \frac{D}{2} + \frac{K}{4}}\right)
\end{aligned}\end{equation}

\subsection{Renormalisation of the hybridisation \(V\)}
Renormalisation of \(V\) happens through two kinds of processes: \(VJ\) and \(VK\). In order words, the two vertices involve one single-particle scattering and one spin or isospin exchange respectively. We first look at the vertices that involve a spin-exchange scattering.

Within spin-exchange, the scattering can be either via \(S_d^z\) or through \(S_d^\pm\). For the first kind, we have the following contribution in the particle sector:
\begin{equation}\begin{aligned}
	\sum_{q\beta} Vc^\dagger_{q\beta} c_{d\beta} \frac{1}{\omega - H_D}\frac{1}{4}J \sum_{k} \left(\hat n_{d\beta} - \hat n_{d\overline\beta}\right) c^\dagger_{k\beta}c_{q\beta} = \frac{1}{4}V J n_j \frac{1}{2}\left(\frac{1}{\omega^\prime_1 - E} + \frac{1}{\omega^\prime_2 - E}\right)\sum_{k\beta} \left(1 - \hat n_{d\overline\beta}\right) c_{d\beta}c^\dagger_{k\beta}
\end{aligned}\end{equation}
The transformation from \(\frac{1}{\omega - H_D}\) to \(\frac{1}{2}\left(\frac{1}{\omega^\prime_1 - E} + \frac{1}{\omega^\prime_2 - E}\right)\) is made so that we can account for both the initial state and the final state energies through the two fluctuation scales \(\omega^\prime_1\) and \(\omega_2^\prime\) respectively; we calculate the denominators for both the initial and final states, and then take the mean of the two (hence the factor of half in front). This was not required previously because in the earlier scattering processes, the impurity returned to its initial state at the end, at least in terms of \(\epsilon_d \left( \hat n_{d \uparrow} - \hat n_{d \downarrow} \right)^2 \), and so we had \(\omega_1^\prime = \omega_2^\prime = \omega^\prime\).

Note that the \(c_{d\beta}\) in front of the Greens function resulted in \(\left(\hat n_{d\beta} - \hat n_{d\overline\beta}\right) \to \left(1 - \hat n_{d\overline\beta}\right)\). The intermediate state is characterised by \(\hat n_{d\beta} = 1 - \hat n_{d \overline \beta} = 1\), which means that \(E = \frac{D}{2} + \epsilon_d - \frac{J}{4}\). Moreover, the initial state gives \(\omega_1^\prime = \omega + \epsilon_d\) while the final state gives \(\omega^\prime_2 = \omega\). Therefore, the renormalisation becomes
\begin{equation}\begin{aligned}
	-\frac{n_j}{4}V J \frac{1}{2}\left(\frac{1}{\omega - \frac{D}{2} + \frac{J}{4}} + \frac{1}{\omega - \frac{D}{2} - \epsilon_d + \frac{J}{4}}\right)\sum_{k\beta}\left(1 - \hat n_{d\overline\beta}\right) c^\dagger_{k\beta} c_{d\beta}
\end{aligned}\end{equation}
One can generate another such process by exchanging the single-particle process and the spin-exchange process:
\begin{equation}\begin{aligned}
	\sum_{q\beta} \frac{1}{4}J \sum_{k} \left(\hat n_{d\beta} - \hat n_{d\overline\beta}\right) c^\dagger_{q\beta}c_{k\beta} \frac{1}{\omega - H_D} V c^\dagger_{d\beta} c_{q\beta}
\end{aligned}\end{equation}
This is simply the Hermitian conjugate of the previous contribution. Combining this with the previous then gives
\begin{equation}\begin{aligned}
	-\frac{n_j}{8}V J \left(\frac{1}{\omega - \frac{D}{2} + \frac{J}{4}} + \frac{1}{\omega - \frac{D}{2} - \epsilon_d + \frac{J}{4}}\right) \sum_{k\beta}\left(1 - \hat n_{d\overline\beta}\right)\left(c^\dagger_{d\beta} c_{k\beta} + \text{h.c.}\right)
\end{aligned}\end{equation}

We now consider the spin-exchange processes involving \(S_d^\pm\):
\begin{equation}\begin{aligned}
	\sum_{q\beta} Vc^\dagger_{q\beta} c_{d\beta} \frac{1}{\omega - H_D}\frac{1}{2}J \sum_{k} c^\dagger_{d\beta}c_{d\overline\beta} c^\dagger_{k\overline\beta}c_{q\beta} = \frac{1}{2}V J n_j \frac{1}{2}\left(\frac{1}{\omega^\prime_1 - E} + \frac{1}{\omega^\prime_2 - E}\right) \sum_{k\beta} \left(1 - \hat n_{d\beta}\right) c_{d\overline\beta}c^\dagger_{k\overline\beta}
\end{aligned}\end{equation}
We again have \(E = \frac{D}{2} + \epsilon_d - \frac{J}{4},\omega_1^\prime = \omega + \epsilon_d\) and \(\omega_2^\prime = \omega\), which gives
\begin{equation}\begin{aligned}
	-\frac{1}{4}V J n_j \left(\frac{1}{\omega - \frac{D}{2} + \frac{J}{4}} + \frac{1}{\omega - \frac{D}{2} - \epsilon_d + \frac{J}{4}}\right) \sum_{k\beta} \left(1 - \hat n_{d\beta}\right)c^\dagger_{k\overline\beta} c_{d\overline\beta}
\end{aligned}\end{equation}
Combining this with the Hermitian conjugate obtained from exchanging the processes gives
\begin{equation}\begin{aligned}
	-\frac{1}{4}V J n_j \left(\frac{1}{\omega - \frac{D}{2} + \frac{J}{4}} + \frac{1}{\omega - \frac{D}{2} - \epsilon_d + \frac{J}{4}}\right) \sum_{k\beta} \left(1 - \hat n_{d\beta}\right)\left(c^\dagger_{k\overline\beta} c_{d\overline\beta} + \text{h.c.}\right)
\end{aligned}\end{equation}

The contributions from the hole sector are obtained making the transformation \(\hat n_{d\overline\beta} \to 1 - \hat n_{d\overline\beta}\) on the particle sector contributions. The total renormalisation to \(V\) from \(VJ\) processes are
\begin{equation}\begin{aligned}
	-\frac{3n_j}{8}V J \left(\frac{1}{\omega - \frac{D}{2} + \frac{J}{4}} + \frac{1}{\omega - \frac{D}{2} - \epsilon_d + \frac{J}{4}}\right) \sum_{k\beta}\left(c^\dagger_{d\beta} c_{k\beta} + \text{h.c.}\right)
\end{aligned}\end{equation}

We now look at the \(VK\) processes. The first one is
\begin{equation}\begin{aligned}
	\sum_{q\beta} Vc^\dagger_{q\beta} c_{d\beta} \frac{1}{\omega - H_D}\frac{1}{4}K \sum_{k} \left(\hat n_{d} - 1\right) c^\dagger_{k\beta}c_{q\beta} = -\frac{1}{8}V K n_j \left(\frac{1}{\omega - \frac{D}{2} + \frac{K}{4}} + \frac{1}{\omega - \frac{D}{2} + \epsilon_d + \frac{K}{4}}\right) \sum_{k\beta} \hat n_{d\overline \beta}c^\dagger_{k\beta}c_{d\beta}
\end{aligned}\end{equation}
The exchanged process again gives the Hermitian conjugate, so the combined contribution is
\begin{equation}\begin{aligned}
	-\frac{1}{8}V K n_j \left(\frac{1}{\omega - \frac{D}{2} + \frac{K}{4}} + \frac{1}{\omega - \frac{D}{2} + \epsilon_d + \frac{K}{4}}\right) \sum_{k\beta} \hat n_{d\overline \beta} \left(c^\dagger_{k\beta}c_{d\beta} + \text{h.c.}\right)
\end{aligned}\end{equation}

The isospin-flip vertex gives
\begin{equation}\begin{aligned}
	\sum_{q\beta} V c^\dagger_{q\beta} c_{d\beta} \frac{1}{\omega - H_D}\frac{1}{2}K \sum_{k} c^\dagger_{d\beta}c^\dagger_{d\overline\beta} c_{k\overline\beta} c_{q\beta} = \frac{1}{4}K V n_j \left(\frac{1}{\omega - \frac{D}{2} + \frac{K}{4}} + \frac{1}{\omega - \frac{D}{2} + \epsilon_d + \frac{K}{4}}\right) \sum_{k\beta} \left(1 - \hat n_{d\beta}\right) c^\dagger_{d\overline\beta}c_{k\overline\beta}~.
\end{aligned}\end{equation}
Combining with Hermitian conjugate gives
\begin{equation}\begin{aligned}
	\frac{1}{4}K V n_j \left(\frac{1}{\omega - \frac{D}{2} + \frac{K}{4}} + \frac{1}{\omega - \frac{D}{2} + \epsilon_d + \frac{K}{4}}\right) \sum_{k\beta} \left(1 - \hat n_{d\beta}\right) \left(c^\dagger_{d\overline\beta}c_{k\overline\beta} + \text{h.c.}\right)~.
\end{aligned}\end{equation}

After obtaining the hole sector contributions, the total renormalisation from \(VK\) processes is
\begin{equation}\begin{aligned}
	-\frac{3n_j}{4}V K \left(\frac{1}{\omega - \frac{D}{2} + \frac{K}{4}} + \frac{1}{\omega - \frac{D}{2} + \epsilon_d + \frac{K}{4}}\right) \sum_{k\beta}\left(c^\dagger_{d\beta} c_{k\beta} + \text{h.c.}\right)~.
\end{aligned}\end{equation}

The RG equation for \(V\) is
\begin{equation}\begin{aligned}
	\Delta V = -\frac{3n_j V}{8}\left[J\left(\frac{1}{\omega - \frac{D}{2} + \frac{J}{4}} + \frac{1}{\omega - \frac{D}{2} - \epsilon_d + \frac{J}{4}}\right) + K \left(\frac{1}{\omega - \frac{D}{2} + \frac{K}{4}} + \frac{1}{\omega - \frac{D}{2} + \epsilon_d + \frac{K}{4}}\right)\right]
\end{aligned}\end{equation}

\subsection{Renormalisation of the exchange couplings \(J\) and \(K\)}
We will just note the renormalisation in \(J^z\), which will be equal to \(J^\pm\) due to spin-rotation symmetry. The terms that renormalise \(J^z\) are of the form \(S_d^\pm S_d^\mp\). In the particle sector, we have
\begin{equation}\begin{aligned}
	\sum_{q} \sum_{kk^\prime}\frac{1}{4}J^2 S_d^\pm c^\dagger_{q\mp}c_{k^\prime\pm} \frac{1}{\omega - H_D}S_d^\mp c^\dagger_{k\pm}c_{q\mp} = -n_j \frac{1}{4}J^2 \left(\frac{1}{2} \pm S_d^z\right) \sum_{kk^\prime}c^\dagger_{k \pm}c_{k\pm} \frac{1}{\omega - \frac{D}{2} + \frac{J}{4}}~.
\end{aligned}\end{equation}
The denominator is determined using \(E = \frac{D}{2} + \epsilon_d - \frac{J}{4}\) and \(\omega^\prime = \omega + \epsilon_d\).
In the hole sector, we similarly have
\begin{equation}\begin{aligned}
	\sum_{q} \sum_{kk^\prime}\frac{1}{4}J^2 S_d^\mp c^\dagger_{k\pm}c_{q\mp} \frac{1}{\omega - H_D}S_d^\pm c^\dagger_{q\mp}c_{k^\prime\pm} = n_j \frac{1}{4}J^2 \left(\frac{1}{2} \mp S_d^z\right) \sum_{kk^\prime}c^\dagger_{k \pm}c_{k\pm} \frac{1}{\omega - \frac{D}{2} + \frac{J}{4}}~.
\end{aligned}\end{equation}
Adding all four expressions and dropping the constant part, we get
\begin{equation}\begin{aligned}
	-n_j \frac{1}{2}J^2 S_d^z \sum_{kk^\prime}\left(c^\dagger_{k \uparrow}c_{k^\prime \uparrow} - c^\dagger_{k \downarrow}c_{k^\prime \downarrow}\right) \frac{1}{\omega - \frac{D}{2} + \frac{J}{4}}~.
\end{aligned}\end{equation}
We can now directly read off the RG equation for \(J\):
\begin{equation}\begin{aligned}
	\Delta J = -\frac{n_j J^2}{\omega - \frac{D}{2} + \frac{J}{4}}
\end{aligned}\end{equation}

Since the spin and charge degrees of freedom are treated on an equal footing in the model, we obtain the RG equation for \(K\) by simply changing \(J \to K\):
\begin{equation}\begin{aligned}
	\Delta K = -\frac{n_j K^2}{\omega - \frac{D}{2} + \frac{K}{4}}
\end{aligned}\end{equation}

\section{In the presence of \(U_b\)}

We first Fourier transform the \(U_b\)-term to \(k-\)space. In \(k-\)space, the diagonal contribution (to \(H_D\)) coming from this term is the single-particle self-energy \(-U_b\left(\hat n_{q \beta}\right)^2\) which can be made particle-hole symmetric in the form:
\begin{equation}\begin{aligned}
	-U_b\left(\tau_{q \beta}\right)^2
\end{aligned}\end{equation}
where \(q\) is the \(k-\)state being decoupled and \(\tau \equiv \hat n - 1/2\). In the initial state \(\ket{\Psi}_i\), we have \(\langle \hat n_{q\beta} \rangle = 1/2 \implies \tau_{q\beta} = 0\), so the contribution of \(U_b\) to that state is 0. For both hole excitations \(c_{q\beta}\ket{\Psi}_i\) as well as particle excitations \(c^\dagger_{q\beta}\ket{\Psi}_i\), the intermediate state energy lowers to \(-U_b/4\).

The off-diagonal part is
\begin{equation}\begin{aligned}
	-\frac{U_b}{2}\sum_{kk^\prime\sigma}c^\dagger_{k\sigma}c_{k^\prime\sigma} + U_b \sum_{k_1,k_2,k_1^\prime,k_2^\prime} c^\dagger_{k_1 \uparrow}c_{k_2 \uparrow} c^\dagger_{k^\prime_1 \downarrow}c_{k^\prime_2 \downarrow} 
\end{aligned}\end{equation}
We ignore the potential scattering arising from the first term.

\subsection{Renormalisation of \(U_b\)}
\(U_b\) can renormalise only via itself. The relevant renormalisation term in the particle sector is
\begin{equation}\begin{aligned}
	U_b^2 \sum_{q\beta}\sum_{k_1,k_2,k_3,k_1^\prime,k_2^\prime,k_3^\prime} c^\dagger_{q\beta}c_{k_1\beta}c^\dagger_{k_3\overline\beta}c_{k_1^\prime\overline\beta}\frac{1}{\omega - H_D}c^\dagger_{k_2^\prime\overline\beta}c_{k_3^\prime\overline\beta}c^\dagger_{k_2\beta}c_{q\beta}
\end{aligned}\end{equation}
In order to renormalise \(U_b\), we need to contract one more pair of momenta. There are two choices. The first is by setting \(k_3 = k_3^\prime = q\). The two internal states, then, are \(q\beta\) and \(q\overline\beta\). As discussed above, the intermediate state energy is \(-U_b/4\). We therefore have
\begin{equation}\begin{aligned}
	\frac{U_b^2 n_j}{\omega - D/2 + U_b/4}\sum_{\beta}\sum_{k_1,k_2,k_1^\prime,k_2^\prime} c_{k_1\beta}c_{k_1^\prime\overline\beta}c^\dagger_{k_2^\prime\overline\beta}c^\dagger_{k_2\beta} = \frac{U_b^2 n_j}{\omega - D/2 + U_b/4}\sum_{\beta}\sum_{k_1,k_2,k_1^\prime,k_2^\prime} c^\dagger_{k_2^\prime\overline\beta}c_{k_1^\prime\overline\beta}c^\dagger_{k_2\beta}c_{k_1\beta}
\end{aligned}\end{equation}
Another way to contract the momenta is by setting \(k_1^\prime = k_2^\prime = q\), which gives a renormalisation of
\begin{equation}\begin{aligned}
	\frac{U_b^2 n_j}{\omega - D/2 + U_b/4}\sum_{\beta}\sum_{k_1,k_2,k_3,k_3^\prime} c_{k_1\beta}c^\dagger_{k_3 \overline\beta}c_{k_3\prime\overline\beta}c^\dagger_{k_2\beta} = -\frac{U_b^2 n_j}{\omega - D/2}\sum_{\beta}\sum_{k_1,k_2,k_3,k_3^\prime} c^\dagger_{k_3 \overline\beta}c_{k_3\prime\overline\beta}c^\dagger_{k_2\beta}c_{k_1\beta}
\end{aligned}\end{equation}
The two contributions cancel each other. The same cancellation happens in the hole sector as well.

\subsection{Renormalisation of \(U\)}
\(U_b\) does not have any new renormalisation term on account of \(U_b\). \(U_b\) does however modify the existing RG equation for \(U\), by shifting the denominator. The existing RG equation is
\begin{equation}\begin{aligned}
	\Delta U &= -4V^2 n_j\left(\frac{1}{\omega - \frac{D}{2} + \epsilon_d + \frac{K}{4}} - \frac{1}{\omega - \frac{D}{2} - \epsilon_d + \frac{J}{4}}\right) - n_j\left(\frac{J^2}{\omega - \frac{D}{2} + \frac{J}{4}} - \frac{K^2}{\omega - \frac{D}{2} + \frac{K}{4}}\right)~.
\end{aligned}\end{equation}
On accounting for the contribution of \(U_b\) to the denominator, we get
\begin{equation}\begin{aligned}
	\Delta U &= -4V^2 n_j\left(\frac{1}{\omega - \frac{D}{2} + \frac{U_b}{4} + \epsilon_d + \frac{K}{4}} - \frac{1}{\omega - \frac{D}{2} + \frac{U_b}{4} - \epsilon_d + \frac{J}{4}}\right) - n_j\left(\frac{J^2}{\omega - \frac{D}{2} + \frac{U_b}{4} + \frac{J}{4}} - \frac{K^2}{\omega - \frac{D}{2} + \frac{U_b}{4} + \frac{K}{4}}\right)~.
\end{aligned}\end{equation}

\subsection{Renormalisation of \(V\)}
The single-particle  hybridisation \(V\) renormalises through terms of \(V U_b\) and \(U_b V\) kind. The first term gives
\begin{equation}\begin{aligned}
	&\sum_{q\beta}\sum_{k}U_b V c^\dagger_{q\beta}c_{k\beta} \hat n_{q\overline\beta} \frac{1}{\omega - H_D} c^\dagger_{d\beta}c_{q\beta} \\
	&= n_jU_b V\sum_{k\beta} c_{k\beta} \left[\frac{\hat n_{d\overline\beta}}{2}\left(\frac{1}{\omega_1 - E_1} + \frac{1}{\omega^\prime_1 - E_1}\right) + \frac{1-\hat n_{d\overline\beta}}{2}\left(\frac{1}{\omega_0 - E_0} + \frac{1}{\omega_0^\prime - E_0}\right)\right] c^\dagger_{d\beta}
\end{aligned}\end{equation}
\(E_1\) and \(E_0\) are the intermediate state energies for \(\hat n_{d\overline\beta}=1\) and 0 respectively. \(\omega_{1,0}\) are the quantum fluctuation scales for the corresponding initial states. \(\omega^\prime_{1,0}\) are the fluctuation scales for the corresponding final states.
The intermediate energies are \(E_1 = D/2 - U_b/4 - K/4,~ ~ ~ E_0 = D/2 - U_b/4 - U/2 - J/4\). The fluctuation scales are \(\omega_1 = \omega - U/2= \omega_0^\prime,~ ~ ~ \omega_1^\prime = \omega = \omega_0\). Substituting these gives
\begin{equation}\begin{aligned}
	-n_jU_b V\sum_{k\beta} c^\dagger_{d\beta} c_{k\beta} \left[\frac{\hat n_{d\overline\beta}}{2}\left(\frac{1}{\omega - \frac{D}{2} - \frac{U}{2} + \frac{U_b}{4} + \frac{K}{4}} + \frac{1}{\omega - \frac{D}{2} + \frac{U_b}{4} + \frac{K}{4}}\right) \right.\\
+\left. \frac{1-\hat n_{d\overline\beta}}{2}\left(\frac{1}{\omega - \frac{D}{2} + \frac{U_b}{4} + \frac{U}{2} + \frac{J}{4}} + \frac{1}{\omega - \frac{D}{2} + \frac{U_b}{4} + \frac{J}{4}}\right)\right]
\end{aligned}\end{equation}

The second term is of the form
\begin{equation}\begin{aligned}
	\sum_{q\beta}\sum_{k}U_b V c^\dagger_{q\beta}c_{d\beta} \frac{1}{\omega - H_D} \hat n_{q\overline\beta} c^\dagger_{k\beta}c_{q\beta}
\end{aligned}\end{equation}
and this is just the Hermitian conjugate of the previous term, so these two terms together lead to
\begin{equation}\begin{aligned}
	-n_jU_b V\sum_{k\beta} \left(c^\dagger_{d\beta} c_{k\beta} + \text{h.c.}\right) \left[\frac{\hat n_{d\overline\beta}}{2}\left(\frac{1}{\omega - \frac{D}{2} - \frac{U}{2} + \frac{U_b}{4} + \frac{K}{4}} + \frac{1}{\omega - \frac{D}{2} + \frac{U_b}{4} + \frac{K}{4}}\right) \right.\\
	+ \left.\frac{1-\hat n_{d\overline\beta}}{2}\left(\frac{1}{\omega - \frac{D}{2} + \frac{U_b}{4} + \frac{U}{2} + \frac{J}{4}} + \frac{1}{\omega - \frac{D}{2} + \frac{U_b}{4} + \frac{J}{4}}\right)\right]
\end{aligned}\end{equation}

In the hole sector, we have
\begin{equation}\begin{aligned}
	&\sum_{q\beta}\sum_{k}U_b V \hat n_{q\overline\beta} c^\dagger_{k\beta}c_{q\beta} \frac{1}{\omega - H_D} c^\dagger_{q\beta}c_{d\beta}\\
	&-\sum_{q\beta}\sum_{k}U_b V \left(1 - \hat n_{q\overline\beta}\right) c^\dagger_{k\beta}c_{q\beta} \frac{1}{\omega - H_D} c^\dagger_{q\beta}c_{d\beta}\\
	&= -n_jU_b V\sum_{k\beta} c^\dagger_{k\beta} \left[\frac{\hat n_{d\overline\beta}}{2}\left(\frac{1}{\omega_1 - E_1} + \frac{1}{\omega^\prime_1 - E_1}\right) + \frac{1-\hat n_{d\overline\beta}}{2}\left(\frac{1}{\omega_0 - E_0} + \frac{1}{\omega_0^\prime - E_0}\right)\right] c_{d\beta}
\end{aligned}\end{equation}
\(E_1 = D/2 - U_b/4 - U/2 - J/4,~ ~ ~ E_0 = D/2 - U_b/4 - K/4\). The fluctuation scales are \(\omega_1 = \omega = \omega_0^\prime,~ ~ ~ \omega_1^\prime = \omega - U/2 = \omega_0\). Substituting these gives
\begin{equation}\begin{aligned}
	-n_jU_b V\sum_{k\beta} c^\dagger_{d\beta} c_{k\beta} \left[\frac{1 - \hat n_{d\overline\beta}}{2}\left(\frac{1}{\omega - \frac{D}{2} - \frac{U}{2} + \frac{U_b}{4} + \frac{K}{4}} + \frac{1}{\omega - \frac{D}{2} + \frac{U_b}{4} + \frac{K}{4}}\right) \right.\\
+\left. \frac{\hat n_{d\overline\beta}}{2}\left(\frac{1}{\omega - \frac{D}{2} + \frac{U_b}{4} + \frac{U}{2} + \frac{J}{4}} + \frac{1}{\omega - \frac{D}{2} + \frac{U_b}{4} + \frac{J}{4}}\right)\right]
\end{aligned}\end{equation}
The other term, obtained by exchanging \(V\) and \(U_b\), gives the Hermitian conjugate, so the overall contribution from the hole sector is the same as the total contribution from the particle sector, but with \(\hat n_{d\overline\beta} \to 1 - \hat n_{d\overline\beta}\). Combining both the sectors, we get
\begin{equation}\begin{aligned}
	-n_jU_b V\sum_{k\beta} \left(c^\dagger_{d\beta} c_{k\beta} + \text{h.c.}\right) \frac{1}{2}\left[\left(\frac{1}{\omega - \frac{D}{2} - \frac{U}{2} + \frac{U_b}{4} + \frac{K}{4}} + \frac{1}{\omega - \frac{D}{2} + \frac{U_b}{4} + \frac{K}{4}}\right) \right.\\
+\left. \left(\frac{1}{\omega - \frac{D}{2} + \frac{U_b}{4} + \frac{U}{2} + \frac{J}{4}} + \frac{1}{\omega - \frac{D}{2} + \frac{U_b}{4} + \frac{J}{4}}\right)\right]
\end{aligned}\end{equation}

Combining with the already existing RG equations, the complete RG equation for \(V\) becomes
\begin{equation}\begin{aligned}
	\Delta V =& -\frac{3n_j V}{8}\left[\left(\frac{J}{\omega - \frac{D}{2} + \frac{U_b}{4} + \frac{J}{4}} + \frac{J}{\omega - \frac{D}{2} + \frac{U_b}{4} + \frac{U}{2} + \frac{J}{4}}\right) + K \left(\frac{K}{\omega - \frac{D}{2} + \frac{U_b}{4} + \frac{K}{4}} + \frac{K}{\omega - \frac{D}{2} + \frac{U_b}{4} - \frac{U}{2} + \frac{K}{4}}\right)\right]\\
		 &-\frac{n_jU_b}{2}\left[\left(\frac{V}{\omega - \frac{D}{2} - \frac{U}{2} + \frac{U_b}{4} + \frac{K}{4}} + \frac{V}{\omega - \frac{D}{2} + \frac{U_b}{4} + \frac{K}{4}}\right) + \left(\frac{V}{\omega - \frac{D}{2} + \frac{U_b}{4} + \frac{U}{2} + \frac{J}{4}} + \frac{V}{\omega - \frac{D}{2} + \frac{U_b}{4} + \frac{J}{4}}\right)\right]\\
		 &=-\frac{n_j V}{8}\left[\left(\frac{3J + 4U_b}{\omega - \frac{D}{2} + \frac{U_b}{4} + \frac{J}{4}} + \frac{3J + 4U_b}{\omega - \frac{D}{2} + \frac{U_b}{4} + \frac{U}{2} + \frac{J}{4}}\right) + \left(\frac{3K + 4U_b}{\omega - \frac{D}{2} + \frac{U_b}{4} + \frac{K}{4}} + \frac{3K + 4U_b}{\omega - \frac{D}{2} + \frac{U_b}{4} - \frac{U}{2} + \frac{K}{4}}\right)\right]
\end{aligned}\end{equation}

\subsection{Renormalisation of \(J\) and \(K\)}
We will track the entire renormalisation purely from that of \(J^+\), by virtue of the SU(2) symmetry. \(J^+\) renormalises through the \(J U_b\) terms. One of the terms is
\begin{equation}\begin{aligned}
	\frac{1}{2} J U_b \sum_{q} \sum_{k,k^\prime} S_d^+ c^\dagger_{q \downarrow} c_{k \uparrow} \frac{1}{\omega - H_D} \hat n_{q \uparrow} c^\dagger_{k^\prime \downarrow}c_{q \downarrow} = -\frac{1}{2}\frac{J U_b n_j}{\omega - \frac{D}{2} + \frac{U_b}{2} + \frac{J}{4}} \sum_{k,k^\prime} S_d^+ c^\dagger_{k^\prime \downarrow} c_{k \uparrow}
\end{aligned}\end{equation}
The factor of half in front is the same half factor that appears in front of the \(S_1^+ S_2^-, S_1^-S_2^+\) terms when we rewrite \(\vec{S}_1\cdot\vec{S}_2\) in terms of \(S^z, S^\pm\). Another term is obtained by switching \(J\) and \(U_b\):
\begin{equation}\begin{aligned}
	\frac{1}{2} J U_b \sum_{q} \sum_{k,k^\prime} \hat n_{q \downarrow} c^\dagger_{q \uparrow} c_{k \uparrow} \frac{1}{\omega - H_D}S_d^+ c^\dagger_{k^\prime \downarrow} c_{q \uparrow} = -\frac{1}{2}\frac{J U_b n_j}{\omega - \frac{D}{2} + \frac{U_b}{2} + \frac{J}{4}} \sum_{k,k^\prime} S_d^+ c^\dagger_{k^\prime \downarrow} c_{k \uparrow}
\end{aligned}\end{equation}

The corresponding terms in the hole sector are
\begin{equation}\begin{aligned}
	\frac{1}{2} J U_b \sum_{q} \sum_{k,k^\prime} S_d^+ c^\dagger_{k^\prime \downarrow} c_{q \uparrow} \frac{1}{\omega - H_D} \hat n_{q \downarrow} c^\dagger_{q \uparrow}c_{k \uparrow} = -\frac{1}{2}\frac{J U_b n_j}{\omega - \frac{D}{2} + \frac{U_b}{2} + \frac{J}{4}} \sum_{k,k^\prime} S_d^+ c^\dagger_{k^\prime \downarrow} c_{k \uparrow}
\end{aligned}\end{equation}
\begin{equation}\begin{aligned}
	\frac{1}{2} J U_b \sum_{q} \sum_{k,k^\prime} \hat n_{q \uparrow} c^\dagger_{k^\prime \downarrow} c_{q \downarrow} \frac{1}{\omega - H_D}S_d^+ c^\dagger_{q \downarrow} c_{k \uparrow} = -\frac{1}{2}\frac{J U_b n_j}{\omega - \frac{D}{2} + \frac{U_b}{2} + \frac{J}{4}} \sum_{k,k^\prime} S_d^+ c^\dagger_{k^\prime \downarrow} c_{k \uparrow}
\end{aligned}\end{equation}

Adding all these terms and combining with the existing RG equation, we get the updated RG equation for \(J\):
\begin{equation}\begin{aligned}
	\Delta J = -J n_j\frac{4 U_b + J}{\omega - \frac{D}{2} + \frac{U_b}{2} + \frac{J}{4}}
\end{aligned}\end{equation}

We will follow the same strategy with \(K\) - we will calculate the renormalisation in \(K^+\). The first term is
\begin{equation}\begin{aligned}
	\frac{1}{2} K U_b \sum_{q} \sum_{k,k^\prime} \hat n_{q \downarrow} c^\dagger_{q \uparrow}c_{k^\prime \uparrow} \frac{1}{\omega - H_D} C_d^+ c_{k \downarrow} c_{q \uparrow} = -\frac{1}{2}\frac{K U_b n_j}{\omega - \frac{D}{2} + \frac{U_b}{2} + \frac{K}{4}} \sum_{k,k^\prime} C_d^+ c_{k \downarrow} c_{k^\prime \uparrow}
\end{aligned}\end{equation}
The second term in the same sector is obtained by flipping the spins of \(k\) and \(q\):
\begin{equation}\begin{aligned}
	\frac{1}{2} K U_b \sum_{q} \sum_{k,k^\prime} \hat n_{q \uparrow} c^\dagger_{q \downarrow}c_{k^\prime \downarrow} \frac{1}{\omega - H_D} C_d^+ c_{q \downarrow} c_{k \uparrow} = -\frac{1}{2}\frac{K U_b n_j}{\omega - \frac{D}{2} + \frac{U_b}{2} + \frac{K}{4}} \sum_{k,k^\prime} C_d^+ c_{k \downarrow} c_{k^\prime \uparrow}
\end{aligned}\end{equation}

The terms in the hole sector give identical contributions. The RG equation for \(K\) is
\begin{equation}\begin{aligned}
	\Delta K = -K n_j\frac{4 U_b + K}{\omega - \frac{D}{2} + \frac{U_b}{2} + \frac{K}{4}}
\end{aligned}\end{equation}
