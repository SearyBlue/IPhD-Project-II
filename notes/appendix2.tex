\chapter{URG Equations for Generalised SIAM in Presence of Bath Correlation}

We will include a local particle-hole symmetric correlation of strength \(U_b\) on the origin of the lattice:
\begin{equation}\begin{aligned}
	\mathcal{H} = \sum_{k\sigma}\epsilon_k \tau_{k\sigma} + \epsilon_d \left( \hat n_{d \uparrow} - \hat n_{d \downarrow} \right) ^2 + \sum_{k\sigma} \left(V_{k} c^\dagger_{k\sigma} c_{d\sigma} + h.c.\right) +J \vec{S_d}\cdot\vec{s} + K \vec{C_d}\cdot\vec{C}- U_b\left(\hat n_{0 \uparrow} - \hat n_{0 \downarrow}\right)^2 
\end{aligned}\end{equation}

To treat this under RG, we first Fourier transform this term to \(k-\)space. In \(k-\)space, the diagonal contribution (to \(H_D\)) coming from this term is the single-particle self-energy \(-U_b\left(\hat n_{q \beta}\right)^2\) which can be made particle-hole symmetric in the form:
\begin{equation}\begin{aligned}
	-U_b\left(\tau_{q \beta}\right)^2
\end{aligned}\end{equation}
where \(q\) is the \(k-\)state being decoupled and \(\tau \equiv \hat n - 1/2\). In the initial state \(\ket{\Psi}_i\), we have \(\langle \hat n_{q\beta} \rangle = 1/2 \implies \tau_{q\beta} = 0\), so the contribution of \(U_b\) to that state is 0. For both hole excitations \(c_{q\beta}\ket{\Psi}_i\) as well as particle excitations \(c^\dagger_{q\beta}\ket{\Psi}_i\), the intermediate state energy lowers to \(-U_b/4\).

The off-diagonal part is
\begin{equation}\begin{aligned}
	-\frac{U_b}{2}\sum_{kk^\prime\sigma}c^\dagger_{k\sigma}c_{k^\prime\sigma} + U_b \sum_{k_1,k_2,k_1^\prime,k_2^\prime} c^\dagger_{k_1 \uparrow}c_{k_2 \uparrow} c^\dagger_{k^\prime_1 \downarrow}c_{k^\prime_2 \downarrow} 
\end{aligned}\end{equation}
We ignore the potential scattering arising from the first term.

\section{Renormalisation of \(U_b\)}
\(U_b\) can renormalise only via itself. The relevant renormalisation term in the particle sector is
\begin{equation}\begin{aligned}
	U_b^2 \sum_{q\beta}\sum_{k_1,k_2,k_3,k_1^\prime,k_2^\prime,k_3^\prime} c^\dagger_{q\beta}c_{k_1\beta}c^\dagger_{k_3\overline\beta}c_{k_1^\prime\overline\beta}\frac{1}{\omega - H_D}c^\dagger_{k_2^\prime\overline\beta}c_{k_3^\prime\overline\beta}c^\dagger_{k_2\beta}c_{q\beta}
\end{aligned}\end{equation}
In order to renormalise \(U_b\), we need to contract one more pair of momenta. There are two choices. The first is by setting \(k_3 = k_3^\prime = q\). The two internal states, then, are \(q\beta\) and \(q\overline\beta\). As discussed above, the intermediate state energy is \(-U_b/4\). We therefore have
\begin{equation}\begin{aligned}
	\frac{U_b^2 n_j}{\omega - D/2 + U_b/4}\sum_{\beta}\sum_{k_1,k_2,k_1^\prime,k_2^\prime} c_{k_1\beta}c_{k_1^\prime\overline\beta}c^\dagger_{k_2^\prime\overline\beta}c^\dagger_{k_2\beta} = \frac{U_b^2 n_j}{\omega - D/2 + U_b/4}\sum_{\beta}\sum_{k_1,k_2,k_1^\prime,k_2^\prime} c^\dagger_{k_2^\prime\overline\beta}c_{k_1^\prime\overline\beta}c^\dagger_{k_2\beta}c_{k_1\beta}
\end{aligned}\end{equation}
Another way to contract the momenta is by setting \(k_1^\prime = k_2^\prime = q\), which gives a renormalisation of
\begin{equation}\begin{aligned}
	\frac{U_b^2 n_j}{\omega - D/2 + U_b/4}\sum_{\beta}\sum_{k_1,k_2,k_3,k_3^\prime} c_{k_1\beta}c^\dagger_{k_3 \overline\beta}c_{k_3\prime\overline\beta}c^\dagger_{k_2\beta} = -\frac{U_b^2 n_j}{\omega - D/2}\sum_{\beta}\sum_{k_1,k_2,k_3,k_3^\prime} c^\dagger_{k_3 \overline\beta}c_{k_3\prime\overline\beta}c^\dagger_{k_2\beta}c_{k_1\beta}
\end{aligned}\end{equation}
The two contributions cancel each other. The same cancellation happens in the hole sector as well.

\section{Renormalisation of \(U\)}
\(U_b\) does not have any new renormalisation term on account of \(U_b\). \(U_b\) does however modify the existing RG equation for \(U\), by shifting the denominator. The existing RG equation is
\begin{equation}\begin{aligned}
	\Delta U &= -4V^2 n_j\left(\frac{1}{\omega - \frac{D}{2} + \epsilon_d + \frac{K}{4}} - \frac{1}{\omega - \frac{D}{2} - \epsilon_d + \frac{J}{4}}\right) - n_j\left(\frac{J^2}{\omega - \frac{D}{2} + \frac{J}{4}} - \frac{K^2}{\omega - \frac{D}{2} + \frac{K}{4}}\right)~.
\end{aligned}\end{equation}
On accounting for the contribution of \(U_b\) to the denominator, we get
\begin{equation}\begin{aligned}
	\Delta U &= -4V^2 n_j\left(\frac{1}{\omega - \frac{D}{2} + \frac{U_b}{4} + \epsilon_d + \frac{K}{4}} - \frac{1}{\omega - \frac{D}{2} + \frac{U_b}{4} - \epsilon_d + \frac{J}{4}}\right) - n_j\left(\frac{J^2}{\omega - \frac{D}{2} + \frac{U_b}{4} + \frac{J}{4}} - \frac{K^2}{\omega - \frac{D}{2} + \frac{U_b}{4} + \frac{K}{4}}\right)~.
\end{aligned}\end{equation}

\section{Renormalisation of \(V\)}
The single-particle  hybridisation \(V\) renormalises through terms of \(V U_b\) and \(U_b V\) kind. The first term gives
\begin{equation}\begin{aligned}
	&\sum_{q\beta}\sum_{k}U_b V c^\dagger_{q\beta}c_{k\beta} \hat n_{q\overline\beta} \frac{1}{\omega - H_D} c^\dagger_{d\beta}c_{q\beta} \\
	&= n_jU_b V\sum_{k\beta} c_{k\beta} \left[\frac{\hat n_{d\overline\beta}}{2}\left(\frac{1}{\omega_1 - E_1} + \frac{1}{\omega^\prime_1 - E_1}\right) + \frac{1-\hat n_{d\overline\beta}}{2}\left(\frac{1}{\omega_0 - E_0} + \frac{1}{\omega_0^\prime - E_0}\right)\right] c^\dagger_{d\beta}
\end{aligned}\end{equation}
\(E_1\) and \(E_0\) are the intermediate state energies for \(\hat n_{d\overline\beta}=1\) and 0 respectively. \(\omega_{1,0}\) are the quantum fluctuation scales for the corresponding initial states. \(\omega^\prime_{1,0}\) are the fluctuation scales for the corresponding final states.
The intermediate energies are \(E_1 = D/2 - U_b/4 - K/4,~ ~ ~ E_0 = D/2 - U_b/4 - U/2 - J/4\). The fluctuation scales are \(\omega_1 = \omega - U/2= \omega_0^\prime,~ ~ ~ \omega_1^\prime = \omega = \omega_0\). Substituting these gives
\begin{equation}\begin{aligned}
	-n_jU_b V\sum_{k\beta} c^\dagger_{d\beta} c_{k\beta} \left[\frac{\hat n_{d\overline\beta}}{2}\left(\frac{1}{\omega - \frac{D}{2} - \frac{U}{2} + \frac{U_b}{4} + \frac{K}{4}} + \frac{1}{\omega - \frac{D}{2} + \frac{U_b}{4} + \frac{K}{4}}\right) \right.\\
+\left. \frac{1-\hat n_{d\overline\beta}}{2}\left(\frac{1}{\omega - \frac{D}{2} + \frac{U_b}{4} + \frac{U}{2} + \frac{J}{4}} + \frac{1}{\omega - \frac{D}{2} + \frac{U_b}{4} + \frac{J}{4}}\right)\right]
\end{aligned}\end{equation}

The second term is of the form
\begin{equation}\begin{aligned}
	\sum_{q\beta}\sum_{k}U_b V c^\dagger_{q\beta}c_{d\beta} \frac{1}{\omega - H_D} \hat n_{q\overline\beta} c^\dagger_{k\beta}c_{q\beta}
\end{aligned}\end{equation}
and this is just the Hermitian conjugate of the previous term, so these two terms together lead to
\begin{equation}\begin{aligned}
	-n_jU_b V\sum_{k\beta} \left(c^\dagger_{d\beta} c_{k\beta} + \text{h.c.}\right) \left[\frac{\hat n_{d\overline\beta}}{2}\left(\frac{1}{\omega - \frac{D}{2} - \frac{U}{2} + \frac{U_b}{4} + \frac{K}{4}} + \frac{1}{\omega - \frac{D}{2} + \frac{U_b}{4} + \frac{K}{4}}\right) \right.\\
	+ \left.\frac{1-\hat n_{d\overline\beta}}{2}\left(\frac{1}{\omega - \frac{D}{2} + \frac{U_b}{4} + \frac{U}{2} + \frac{J}{4}} + \frac{1}{\omega - \frac{D}{2} + \frac{U_b}{4} + \frac{J}{4}}\right)\right]
\end{aligned}\end{equation}

In the hole sector, we have
\begin{equation}\begin{aligned}
	&\sum_{q\beta}\sum_{k}U_b V \hat n_{q\overline\beta} c^\dagger_{k\beta}c_{q\beta} \frac{1}{\omega - H_D} c^\dagger_{q\beta}c_{d\beta}\\
	&-\sum_{q\beta}\sum_{k}U_b V \left(1 - \hat n_{q\overline\beta}\right) c^\dagger_{k\beta}c_{q\beta} \frac{1}{\omega - H_D} c^\dagger_{q\beta}c_{d\beta}\\
	&= -n_jU_b V\sum_{k\beta} c^\dagger_{k\beta} \left[\frac{\hat n_{d\overline\beta}}{2}\left(\frac{1}{\omega_1 - E_1} + \frac{1}{\omega^\prime_1 - E_1}\right) + \frac{1-\hat n_{d\overline\beta}}{2}\left(\frac{1}{\omega_0 - E_0} + \frac{1}{\omega_0^\prime - E_0}\right)\right] c_{d\beta}
\end{aligned}\end{equation}
\(E_1 = D/2 - U_b/4 - U/2 - J/4,~ ~ ~ E_0 = D/2 - U_b/4 - K/4\). The fluctuation scales are \(\omega_1 = \omega = \omega_0^\prime,~ ~ ~ \omega_1^\prime = \omega - U/2 = \omega_0\). Substituting these gives
\begin{equation}\begin{aligned}
	-n_jU_b V\sum_{k\beta} c^\dagger_{d\beta} c_{k\beta} \left[\frac{1 - \hat n_{d\overline\beta}}{2}\left(\frac{1}{\omega - \frac{D}{2} - \frac{U}{2} + \frac{U_b}{4} + \frac{K}{4}} + \frac{1}{\omega - \frac{D}{2} + \frac{U_b}{4} + \frac{K}{4}}\right) \right.\\
+\left. \frac{\hat n_{d\overline\beta}}{2}\left(\frac{1}{\omega - \frac{D}{2} + \frac{U_b}{4} + \frac{U}{2} + \frac{J}{4}} + \frac{1}{\omega - \frac{D}{2} + \frac{U_b}{4} + \frac{J}{4}}\right)\right]
\end{aligned}\end{equation}
The other term, obtained by exchanging \(V\) and \(U_b\), gives the Hermitian conjugate, so the overall contribution from the hole sector is the same as the total contribution from the particle sector, but with \(\hat n_{d\overline\beta} \to 1 - \hat n_{d\overline\beta}\). Combining both the sectors, we get
\begin{equation}\begin{aligned}
	-n_jU_b V\sum_{k\beta} \left(c^\dagger_{d\beta} c_{k\beta} + \text{h.c.}\right) \frac{1}{2}\left[\left(\frac{1}{\omega - \frac{D}{2} - \frac{U}{2} + \frac{U_b}{4} + \frac{K}{4}} + \frac{1}{\omega - \frac{D}{2} + \frac{U_b}{4} + \frac{K}{4}}\right) \right.\\
+\left. \left(\frac{1}{\omega - \frac{D}{2} + \frac{U_b}{4} + \frac{U}{2} + \frac{J}{4}} + \frac{1}{\omega - \frac{D}{2} + \frac{U_b}{4} + \frac{J}{4}}\right)\right]
\end{aligned}\end{equation}

Combining with the already existing RG equations, the complete RG equation for \(V\) becomes
\begin{equation}\begin{aligned}
	\Delta V =& -\frac{3n_j V}{8}\left[\left(\frac{J}{\omega - \frac{D}{2} + \frac{U_b}{4} + \frac{J}{4}} + \frac{J}{\omega - \frac{D}{2} + \frac{U_b}{4} + \frac{U}{2} + \frac{J}{4}}\right) + K \left(\frac{K}{\omega - \frac{D}{2} + \frac{U_b}{4} + \frac{K}{4}} + \frac{K}{\omega - \frac{D}{2} + \frac{U_b}{4} - \frac{U}{2} + \frac{K}{4}}\right)\right]\\
		 &-\frac{n_jU_b}{2}\left[\left(\frac{V}{\omega - \frac{D}{2} - \frac{U}{2} + \frac{U_b}{4} + \frac{K}{4}} + \frac{V}{\omega - \frac{D}{2} + \frac{U_b}{4} + \frac{K}{4}}\right) + \left(\frac{V}{\omega - \frac{D}{2} + \frac{U_b}{4} + \frac{U}{2} + \frac{J}{4}} + \frac{V}{\omega - \frac{D}{2} + \frac{U_b}{4} + \frac{J}{4}}\right)\right]\\
		 &=-\frac{n_j V}{8}\left[\left(\frac{3J + 4U_b}{\omega - \frac{D}{2} + \frac{U_b}{4} + \frac{J}{4}} + \frac{3J + 4U_b}{\omega - \frac{D}{2} + \frac{U_b}{4} + \frac{U}{2} + \frac{J}{4}}\right) + \left(\frac{3K + 4U_b}{\omega - \frac{D}{2} + \frac{U_b}{4} + \frac{K}{4}} + \frac{3K + 4U_b}{\omega - \frac{D}{2} + \frac{U_b}{4} - \frac{U}{2} + \frac{K}{4}}\right)\right]
\end{aligned}\end{equation}

\section{Renormalisation of \(J\)}
We will track the entire renormalisation purely from that of \(J^+\), by virtue of the SU(2) symmetry. \(J^+\) renormalises through the \(J U_b\) terms. One of the terms is
\begin{equation}\begin{aligned}
	\frac{1}{2} J U_b \sum_{q} \sum_{k,k^\prime} S_d^+ c^\dagger_{q \downarrow} c_{k \uparrow} \frac{1}{\omega - H_D} \hat n_{q \uparrow} c^\dagger_{k^\prime \downarrow}c_{q \downarrow} = -\frac{1}{2}\frac{J U_b n_j}{\omega - \frac{D}{2} + \frac{U_b}{2} + \frac{J}{4}} \sum_{k,k^\prime} S_d^+ c^\dagger_{k^\prime \downarrow} c_{k \uparrow}
\end{aligned}\end{equation}
The factor of half in front is the same half factor that appears in front of the \(S_1^+ S_2^-, S_1^-S_2^+\) terms when we rewrite \(\vec{S}_1\cdot\vec{S}_2\) in terms of \(S^z, S^\pm\). Another term is obtained by switching \(J\) and \(U_b\):
\begin{equation}\begin{aligned}
	\frac{1}{2} J U_b \sum_{q} \sum_{k,k^\prime} \hat n_{q \downarrow} c^\dagger_{q \uparrow} c_{k \uparrow} \frac{1}{\omega - H_D}S_d^+ c^\dagger_{k^\prime \downarrow} c_{q \uparrow} = -\frac{1}{2}\frac{J U_b n_j}{\omega - \frac{D}{2} + \frac{U_b}{2} + \frac{J}{4}} \sum_{k,k^\prime} S_d^+ c^\dagger_{k^\prime \downarrow} c_{k \uparrow}
\end{aligned}\end{equation}

The corresponding terms in the hole sector are
\begin{equation}\begin{aligned}
	\frac{1}{2} J U_b \sum_{q} \sum_{k,k^\prime} S_d^+ c^\dagger_{k^\prime \downarrow} c_{q \uparrow} \frac{1}{\omega - H_D} \hat n_{q \downarrow} c^\dagger_{q \uparrow}c_{k \uparrow} = -\frac{1}{2}\frac{J U_b n_j}{\omega - \frac{D}{2} + \frac{U_b}{2} + \frac{J}{4}} \sum_{k,k^\prime} S_d^+ c^\dagger_{k^\prime \downarrow} c_{k \uparrow}
\end{aligned}\end{equation}
\begin{equation}\begin{aligned}
	\frac{1}{2} J U_b \sum_{q} \sum_{k,k^\prime} \hat n_{q \uparrow} c^\dagger_{k^\prime \downarrow} c_{q \downarrow} \frac{1}{\omega - H_D}S_d^+ c^\dagger_{q \downarrow} c_{k \uparrow} = -\frac{1}{2}\frac{J U_b n_j}{\omega - \frac{D}{2} + \frac{U_b}{2} + \frac{J}{4}} \sum_{k,k^\prime} S_d^+ c^\dagger_{k^\prime \downarrow} c_{k \uparrow}
\end{aligned}\end{equation}

Adding all these terms and combining with the existing RG equation, we get the updated RG equation for \(J\):
\begin{equation}\begin{aligned}
	\Delta J = -J n_j\frac{4 U_b + J}{\omega - \frac{D}{2} + \frac{U_b}{2} + \frac{J}{4}}
\end{aligned}\end{equation}

\section{Renormalisation of \(K\)}
We will follow the same strategy with \(K\) - we will calculate the renormalisation in \(K^+\). The first term is
\begin{equation}\begin{aligned}
	\frac{1}{2} K U_b \sum_{q} \sum_{k,k^\prime} \hat n_{q \downarrow} c^\dagger_{q \uparrow}c_{k^\prime \uparrow} \frac{1}{\omega - H_D} C_d^+ c_{k \downarrow} c_{q \uparrow} = -\frac{1}{2}\frac{K U_b n_j}{\omega - \frac{D}{2} + \frac{U_b}{2} + \frac{K}{4}} \sum_{k,k^\prime} C_d^+ c_{k \downarrow} c_{k^\prime \uparrow}
\end{aligned}\end{equation}
The second term in the same sector is obtained by flipping the spins of \(k\) and \(q\):
\begin{equation}\begin{aligned}
	\frac{1}{2} K U_b \sum_{q} \sum_{k,k^\prime} \hat n_{q \uparrow} c^\dagger_{q \downarrow}c_{k^\prime \downarrow} \frac{1}{\omega - H_D} C_d^+ c_{q \downarrow} c_{k \uparrow} = -\frac{1}{2}\frac{K U_b n_j}{\omega - \frac{D}{2} + \frac{U_b}{2} + \frac{K}{4}} \sum_{k,k^\prime} C_d^+ c_{k \downarrow} c_{k^\prime \uparrow}
\end{aligned}\end{equation}

The terms in the hole sector give identical contributions. The RG equation for \(K\) is
\begin{equation}\begin{aligned}
	\Delta K = -K n_j\frac{4 U_b + K}{\omega - \frac{D}{2} + \frac{U_b}{2} + \frac{K}{4}}
\end{aligned}\end{equation}




