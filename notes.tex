\documentclass[14pt]{extarticle}
\usepackage{common}
\begin{document}
\tableofcontents
\subsection{\il{T-}matrix and \il{S-}matrix}
It is defined as
\beq
V\psi = T\phi
\eeq
where \il{\psi} is the total scattered wavefunction and \il{\phi} is the incoming wavefunction. They satisfy the Schrodinger equations
\begin{gather}
H_0 \phi = E \phi\\
(H_0 + V)\psi = E \psi
\end{gather}
Since we are assuming elastic scattering, both have the same energy. The Schrodinger equation for \il{\psi} can be rearranged into
\beq[lse]
\psi = \phi + G_0V\psi
\eeq
where \il{G_0^{-1} = E - H_0}. This is also called the Lippmann-Schwinger equation. Using the definition of \il{T} gives
\beq
\psi &= \phi + G_0 T \phi \\
\implies \psi &=(1+G_0T) \phi
\eeq
Eq.~\ref{lse} can also be written as 
\beq
\psi = (1- G_0V)^{-1} \phi
\eeq
Comparing the last two equations gives
\beq
1 &= (1-G_0 V)(1+G_0 T) \\
\implies T &= V +VG_0 T
\eeq
The last equation allows us to perturbatively expand the \il{T-} matrix.
\beq
T = V + VG_0V + VG_0VG_0V + ...
\eeq
From scattering theory, we can write
\beq
\psi = \rr{2\pi}^{-\fr{3}{2}}\qq{e^{ikx} + f \fr{e^{ikr}}{r}}
\eeq
where the wave amplitude \il{f(k^\prime,k) \sim \bra{k^\prime} V \ket{\psi}}. Using the definition of \il{T}, we get
\beq
f(k^\prime,k) \sim \bra{k^\prime} T \ket{k}
\eeq
By definition, the \il{S-}matrix is
\beq
S(\omega) = e^{2i\pi\delta(\omega)} 
\eeq
Also,
\beq
S = 1-2 i \pi \rho T
\eeq
Therefore,
\beq[tmatphase]
T = \fr{1}{2i\pi \rho}\rr{1 - e^{2i\pi\delta(\omega)}} = -\fr{e^{i\delta}\sin \delta}{\pi \rho}
\eeq

\subsection{An identity}
If, for some operator \il{A}, we have \il{\qq{H,A} = \lambda A}, where \il{\lambda} is some scalar, then we can write
\beq
HA = A(\lambda+H)
\eeq
A consequence of this is, for another scalar E, we can write
\begin{gather}
(E - H)A = AE - A(\lambda+H) = A\rr{E -\lambda -H} \\
\implies A(E - \lambda - H)^{-1} = (E-H)^{-1}A\label{identity}
\end{gather}

\subsection{Landau's theory of Fermi liquid}
A ideal Fermi gas is a collection of non-interacting Fermions. Since they are non-interacting, the eigenstates of the complete system are just the collections of the eigenstates of the particles and holes. The eigenstate will be of  the form \il{\{n_{k_1\ua},n_{k_1\da},n_{k_2\ua},n_{k_2\da},n_{k_3\ua},n_{k_3\da},...,\}}, where \il{n_{k\sigma}\in\{0,1\}} is the number of particles with momentum \il{k} and spin \il{\sigma}. In the ground state,
\beq
n_{k\sigma} = \begin{cases} 1 & k \leq k_F \\ 0 & k>k_F \end{cases} 
\eeq
Excitations involve adding an electron above \il{k_F} or deleting an electron below \il{k_F}. The former is called a \it{particle} while the latter is called a \it{hole}. \bf{A general excited state of the total system is a collection of particle and hole excitations.}\\\\
We next consider interacting systems, but very specific ones. That is, we consider interacting systems whose excitations can be mapped one-to-one with the excitations of the ideal system, provided the interactions are turned on sufficiently slowly. Alternatively, there exists a sufficiently slow rate of turning off the interactions such that any interacting excited state \il{\Psi^*} continuously flows into some excited state \il{\Phi} of the ideal system as the interactions flow to 0.
\begin{gather}
\Psi^* \xrightarrow{\text{turn interactions off}} \Phi \\
\Phi^* \xrightarrow{\text{turn interactions off}} \Psi
\end{gather}
If some state \il{\Gamma^*}, instead of flowing into an ideal excited state, gets lost while the interactions are being turned off, then we arent considering that system. All excited states must flow into some ideal state and vice-versa. This means that the interacting excited states can be labelled by the same good quantum numbers \il{\vec k} and \il{\sigma}.That is, if some eigenstate of the interacting system flows into the ideal eigenstate \il{\{n_{k\sigma}\}}, we can just as well use the distribution \il{\{n_{k\sigma}\}} to label the interacting eigenstate. \it{The particle and hole excitations of the ideal system might flow into some very complicated state of the interacting system, which we call a quasiparticle(hole)}. In other words, a system (ideal ground state + particle of momentum \il{k}) goes to (interacting ground state + quasip of momentum \il{k}). If an interacting eigenstate corresponds to the state \il{\{n_{k\sigma}\}}, then \il{\{n_{k\sigma}\}} is said to be the quasiparticle distribution function for that interacting state. Just as we denote eigenstates of the total ideal system using the collection of particles and holes, similarly we use the quasiparticle and quasiholes to describe eigenstates of the interacting system. Let \il{k_+>k_F} and \il{k_- < k_F}. Then,
\begin{gather}
	\ket{IGS}\otimes\ket{k_+} \xrightarrow{\text{turn interactions on}} \ket{RGS}\otimes\ket{k_+}^* \\
	\ket{IGS}\otimes\ket{k_-} \xrightarrow{\text{turn interactions on}} \ket{RGS}\otimes\ket{k_-}^* \\
\end{gather}
\il{\ket{}^*} denotes a quasiket. IGS and RGS are ideal and real(interacting) ground states. Another way of putting this is
\begin{gather}
c^\dagger_{k_+}\ket{IGS} \xrightarrow{\text{turn interactions on}} \eta^\dagger_{k_+}\ket{RGS}\\
c_{k_-}\ket{IGS} \xrightarrow{\text{turn interactions on}} \eta_{k_-}\ket{RGS}
\end{gather}
\il{\eta^\dagger} is the creation operator for the quasiparticle. \il{n_k} gives the distribution of momentum \il{k} quasiparticles. If \il{n^0_k} is the distribution in the ground state, the excitation can be measured as 
\beq
\delta n_k = n_k - n^0_k
\eeq
In general, the total energy \il{\mathcal{E}} of the system will be a functional of the distribution function \il{n_k}. For the ideal system, this function is very simple.
\beq
\mathcal{E}^0[n_k] = \sum_k n_k \epsilon_k
\eeq
The functional in case of the interacting system might be very complex. Upto first order in the functional, we can write
\beq
\mathcal{E}[n_k] = \mathcal{E}[n^0_k] + \sum_k \pd{\mathcal{E}}{n_k}\delta n_k
\eeq
The first order variation in \il{\mathcal{E}} is thus
\beq[tote]
\delta \mathcal{E} \equiv \mathcal{E}[n_k] -\mathcal{E}[n^0_k] = \sum_k \xi_k \delta n_k
\eeq
where \il{\xi_k = \pd{\mathcal{E}}{n_k}}. \il{\xi_k} is the energy of the quasiparticles(hole) or momentum \il{k}. To see this, note that if \il{\mathcal{E}[n_k^1]} and \il{\mathcal{E}[n_k^2]} are the energies before and after adding a quasiparticle of momentum \il{q}, we have
\beq
n_k^2 - n_k^1 = \begin{cases} 0 & k \neq q\\ 1 & k = q \end{cases}
\eeq
Then, up to first order,
\beq
\mathcal{E}[n_k^2] - \mathcal{E}[n_k^1] = \sum_k \xi_k \rr{n^2_k - n^1_k} = \xi_q
\eeq
This shows that the effect of adding a quasiparticle of momentum \il{q} is to raise the total energy by \il{\xi_q}. It is thus sensible to call that the energy of the quasiparticle. \il{\xi_k} itself might depend on whether other quasiparticles are present; there might be interactions among them. This effectively means that \il{xi_k} itself is, in general, a functional of \il{n_k}. Consequently, we expand it upto first order.
\beq[quasienergy]
\xi_k = \xi_k^0 + \sum_q \pd{\xi_k}{n_q} \delta n_q
\eeq
Eq.~\ref{tote} then becomes
\beq
\delta\mathcal{E} = \sum_k \xi^0_k \delta n_k + \sum_{k,q} f(k,q) \delta n_k \delta n_q
\eeq
where \il{f(k,q) = \pd{\xi_k}{n_q} = \fr{\partial^2 \mathcal{E}}{\partial n_k \partial n_q}} is the interaction between two quasiparticles of momenta \il{k} and \il{q}. The first term is the self energy of the quasiparticles, the other they would have had even if no other quasiparticle was present. Its sort of like their kinetic energy. The second term is the interaction energy between all the quasiparticles. Hence, the term \il{f(k_1,k_2)} comes into play only when \il{\delta n_{k_1} \neq 0} and \il{\delta n_{k_2} \neq 0}, that is when both the quasiparticles are present.\\\\
Since the quasiparticles are in direct correspondence with the fermionic particles, they must also be fermions. This allows us to write down the probability of finding a quasiparticle at energy \il{\xi},
\beq
f(\xi) = \qq{e^{(\xi - \mu)\beta} + 1}^{-1}
\eeq
where \il{\mu = \mathcal{E_0}(N+1) - \mathcal{E_0}(N)= \pd{\mathcal{E}_0}{N}} is the change in ground state energy on adding one quasiparticle. There is a subtlety here though. Since \il{\xi} itself depends on the occupancy, and the occupancy also depends on \il{\xi} through the probability distribution, there is a feedback effect in action here. If any perturbation or field modifies the occupation \il{n_{k\sigma}}, it will produce a feedback effect on all the occupations, through the \il{\xi}. \\\\
A temperature-dependent free energy can be concocted using
\beq
F[\delta n_k] = \delta \mathcal{E}[\delta n_k] - T S[\delta n_k]
\eeq
Minimizing this gives an expectation value of the excitation distribution \il{\avg{\delta n_k}}. This in turn gives a temperature-dependent quasiparticle energy
\beq[temp_en]
\xi_k(T) = \xi^0_k + \sum_{q} f(k,q)\avg{\delta n_q}
\eeq
where \il{\avg{\delta n_q}} is obtained by
\beq
\dv{F}{\delta n_q}\bigg|_{\delta n_q = \avg{\delta n_q}} = 0
\eeq
\subsection{The model}
\beq
H = \epsilon_d \hat n_d + \sum_{k} \epsilon_k \hat n_k + \sum_{k\sigma}t\rr{c^\dagger_{k\sigma}c_{d\sigma}+c^\dagger_{d\sigma}c_{k\sigma}} + U\hat n_{d\ua}\hat n_{d\da}
\eeq
\paragraph{Energy scales:}
\begin{itemize}
	\item \il{\epsilon_d}
	\item \il{U}
	\item \il{\fr{2\Delta}{\hbar} = \tau^{-1} = \fr{2\pi}{\hbar}t^2\sum_k \rho(\epsilon_k) \ra } extent of hybridisation (rate of transition) between conduction band and impurity site
\end{itemize}
\paragraph{Situations:}
\begin{itemize}
	\item \il{U \gg \epsilon_d \gg \Delta}: Double occupation is not possible. \il{\Delta} being small means very small hybridisation. So, d-site is either up or down, hence magnetic.
	\item \il{U \gg \Delta\gg \epsilon_d }: Double occupation is still not possible, but now hybridisation will allows the up and down spins to fluctuate on the d-site, leading to zero average magnetization.
	\item \il{\Delta\gg U \gg \epsilon_d}: Hybridisation now fluctuates the up and down spins , leading to zero average magnetization.
\end{itemize}

\subsection{Atomic limit (\il{t=0})}
\beq
H_\text{atomic} = E_d + E_{CB} + U n_{d\ua}n_{d\da}
\eeq
Since we are not interested in the Fermi sea, the \il{E_{CB}} is dropped:
\beq
H_\text{atomic} = \epsilon_d n_d + U n_{d\ua}n_{d\da}
\eeq
For a magnetic solution, we need
\beq
(\epsilon_\ua = \epsilon_\da =)\epsilon_d < (\epsilon_0,\epsilon_{\ua\da}) 0,2\epsilon_d + U
\eeq
Assuming \il{\epsilon_d = -|\epsilon_d|}, this is equivalent to
\beq
\epsilon_d > -U
\eeq

\subsection{Non-interacting limit (\il{U=0}):}
\beq
H_\text{non-int} = \epsilon_d n_d + \sum_k \epsilon_k n_k + \sum_{k\sigma} t\rr{c^\dagger_{k\sigma}c_{d\sigma}+c^\dagger_{d\sigma}c_{k\sigma}}
\eeq
\subsubsection*{Green's function of impurity site:}
We want to write down the \it{Green's function} \il{G_d} for the impurity site. In the absence of the hybridisation, this quantity is
\beq
G^0_d(E) = \fr{1}{E - \epsilon_d}
\eeq
In the presence of the coupling with the conduction band, there are several ways of creating an excitation at the impurity site, with an energy \il{E}. The first is the bare Green's function. This is the situation when the impurity site electron has not scattered. Next is the case that there is an excitation with energy E (\il{G^0_d(E)}) followed by a scattering to the conduction band at some momentum \il{k}. The probability of the scattering is \il{t}. The Greens function for creating the electron \il{k} is \il{G^0_k = \fr{1}{E-\epsilon_k}}, and the probability of again scattering back to the impurity site is \il{t}, with the Greens function for this final excitation being \il{G^0_d}. The total Greens function contribution for this case is
\beq
G^0_d \Sigma_c G^0_d, \text{  where  }\Sigma_c = t \rr{\sum_k G^0_k} t = \sum_k \fr{t^2}{E - \epsilon_k}
\eeq
Considering higher scatterings lead to terms like \il{G^0_d \Sigma_c G^0_d\Sigma_c G^0_d},\\\il{G^0_d \Sigma_c G^0_d\Sigma_c G^0_d\Sigma_c G^0_d} and so on. The total Greens function is
\beq
G_d(E) &= G^0_d + G^0_d \Sigma_c G^0_d + G^0_d \Sigma_c G^0_d\Sigma_c G^0_d + G^0_d \Sigma_c G^0_d\Sigma_c G^0_d\Sigma_c G^0_d + ... \\
       &= G^0_d\qq{1+\rr{\Sigma_c G^0_d}^2+...} = G^0_d \fr{1}{1-\Sigma_c G^0_d} = \fr{1}{E - \epsilon_d - \Sigma_c(E)} 
\eeq
Now,
\begin{gather}
\fr{1}{t^2}\Sigma_c(E) = \sum_k \fr{1}{E - \epsilon_k} = \lim_{\eta \ra 0}\int_{-W}^W d\epsilon \rho(\epsilon) \fr{1}{E - \epsilon + i\eta}\\
\implies \fr{1}{t^2}\text{Re} \qq{\Sigma_c(E)} = \int_{-W}^W d\epsilon \rho(\epsilon)\fr{1}{E - \epsilon}, \text{and }\\
\fr{1}{t^2}\text{Im} \qq{\Sigma_c(E)} = \int_{-W}^W d\epsilon \rho(\epsilon) (-i\pi)\delta(E-\epsilon)
\end{gather}
Assuming \il{\rho(E)} varies sufficiently slowly, we can neglect the real part,
\beq
\Sigma_c(E) = \text{Im}\qq{\Sigma_c(E)} = -i\pi t^2 \rho(E) = -i\Delta
\eeq
Therefore,
\beq
G_d(E) = \fr{1}{E-\epsilon_d+i\Delta}
\eeq
The difference from \il{G^0_f} can be seen by computing the density of states for both the bare and the interacting ones:
\begin{gather}
	\rho_d^0(E) = -\fr{1}{\pi}\text{Im}\qq{G^0_f} = -\fr{1}{\pi} \lim_{\eta \ra 0} \fr{1}{E - \epsilon_d + i\eta} = \delta(E - \epsilon_d)\\
	\rho_d(E) = -\fr{1}{\pi}\text{Im}\qq{G_f}= -\fr{1}{\pi} \lim_{\eta \ra 0} \fr{1}{E - \epsilon_d + i(\eta+\Delta)} = \fr{1}{\pi}\fr{\Delta}{(E-\epsilon_d)^2 + \Delta^2}\label{densitys}
\end{gather}
The first density of states is delta function, because \il{\epsilon_d} is an eigenstate in that case, and the poles of the corresponding Green's function are real poles. But the presence of the hybridisation means that is no longer the case in the second density of states, so the delta function fades into a Lorentzian in that case, and the poles of the Greens function move off the real axis.

The total number of d-electrons can be calculated as:
\beq[total]
\avg{n_d} = 2\int d\epsilon \rho_d(\epsilon) = \fr{2\Delta}{\pi} \int \fr{d\epsilon}{(\epsilon-\epsilon_d)^2 + \Delta^2} = \fr{2}{\pi}\cot^{-1}\rr{\fr{\epsilon_d}{\Delta}}
\eeq
\subsubsection*{Phase shift of conduction electron due to scattering off the impurity:}
\il{T-}matrix is defined by
\beq
T = V + VGT 
\eeq
We also have
\beq
G = G_0 + G_0VG &= G_0 + G_0 T \fr{1}{1+GT}G \\
		&= G_0 + G_0T(1-GT+...)(G_0+G_0VG_0+...)\\
		&= G_0 + G_0 T G_0 \label{green}
\eeq
The conduction electron Green's function can be calculated as
\beq
G_c(k,k^\prime,E) = \delta_{k,k^\prime}G^0_c(k,E) + G_c^0(k)t G^0_f t G^0_c(k^\prime) + \\ G_c^0(k)t G^0_f t \sum_q G_c^0(q) t G^0_f t G^0_c(k^\prime) + ...\\
\eeq
Noting that 
\beq
t\sum_q G_c^0(q)t = \Sigma_c,
\eeq
we have
\beq
G_c(k,k^\prime,E) = \delta_{k,k^\prime}G^0_c(k,E) + G_c^0(k)t^2 G_f(E)G_c^0(k)
\eeq
Comparing with the final form of \il{G} in eq.~\ref{green}, we can write
\beq[tm]
T(k,k^\prime,E) = t^2 G_d(E) = \fr{t^2}{E-\epsilon_d + i\Delta}=-\fr{t^2}{\Delta} \fr{1}{\fr{ \epsilon_d- E}{\Delta}-i}
\eeq
As an aside, this form of the transition matrix allows us to make a connection:
\beq[dsfromtmat]
\text{Im}[T] = -\fr{t^2 \Delta}{\rr{E-\epsilon_d}^2+\Delta^2} = -\pi t^2 \rho_d
\eeq
The density of states of the impurity site is proportional to the imaginary part of the transition matrix element. This is a general relation, because
\beq
\rho_d = -\fr{1}{\pi}\text{Im}\qq{G_d} = -\fr{1}{\pi t^2}\text{Im}\qq{t^2 G_d} = -\fr{1}{\pi t^2}\text{Im}\qq{T}
\eeq
This relation will hold as long as the \il{T-}matrix is of the form \il{t^2 G_d}.
\\\\
If the phase shift of the conduction electrons due to scattering off the impurity is \il{\delta}, we have
\beq
T = e^{2i\delta} - 1 = e^{i\delta}\rr{e^{i\delta} - e^{-i\delta}} \sim \fr{1}{\cot \delta - i}
\eeq
Comparing with eq.~\ref{tm}, we can write
\beq[phaseshift]
\delta(E) = \cot^{-1}\rr{\fr{\epsilon_d - E}{\Delta}}
\eeq
When \il{E = \epsilon_d}, the phase shift is \il{\pi}, and the scattering is head on (the conduction electron is reflected back). Comparing with eq.~\ref{total},
\beq
\fr{2}{\pi}\delta(0) = \avg{n_d}
\eeq
This is an example of the Friedel sum rule which states that the total number of electrons bound inside a resonance is \il{\fr{1}{\pi}} times the total scattering phase shift at the Fermi surface. In other words, the impuritywill be singly occupied when \il{\delta(0) = \fr{\pi}{2}}.

\subsection{Total Hamiltonian: Mean field treatment}
\begin{gather}
n_{d\ua}n_{d\da} \approx n_{d\ua}\avg{n_{d\da}} + n_{d\da}\avg{n_{d\ua}} + \text{constant}\\
H \approx \sum_k \epsilon_k n_k + \sum_\sigma \qq{\epsilon_d+U \avg{n_{d\ol \sigma}}}n_{d\sigma} + t\sum_{k\sigma}\rr{c^\dagger_{k\sigma}c_{d\sigma}+c^\dagger_{d\sigma}c_{k\sigma}}
\end{gather}
The only change is \il{\epsilon_d \ra \epsilon_{d\sigma} = \epsilon_d + U\avg{n_{d\bar\sigma}}}. This allows us to write
\beq[rho]
\rho_{d\sigma} = \fr{1}{\pi}\fr{\Delta}{(E-\epsilon_{d\sigma})^2 + \Delta^2} \implies \avg{n_{d\sigma}} = \int \rho_{d\sigma} = \fr{1}{\pi}\cot^{-1}\rr{\fr{\epsilon_{d\sigma}}{\Delta}}
\eeq
An alternative way of writing that is
\beq[density]
\fr{\epsilon_{d\sigma}}{\Delta} = \fr{\epsilon_d + U\avg{n_{d\sigma}}}{\Delta} =  \cot\rr{\pi\avg{n_{d\sigma}}} \implies \avg{n_{d\sigma}} = \fr{\Delta}{U}\qq{{\cot\rr{\pi\avg{n_{d\ol\sigma}}} - \fr{\epsilon_d}{\Delta}}}
\eeq
Introducing \il{n_d = \avg{n_{d\ua}} + \avg{n_{d\da}}} and \il{m = \avg{n_{d\ua}} - \avg{n_{d\da}}}, we can write
\beq
\avg{n_{d\ua} - n_{d\da}} \equiv m = \fr{\Delta}{U}\qq{\cot\rr{\pi\avg{n_{d\da}}}-\cot\rr{\pi\avg{n_{d\ua}}}} \\= \fr{\Delta}{U}\qq{\cot\fr{\pi}{2}\rr{n_d-m}-\cot\fr{\pi}{2}\rr{n_d+m}}
\eeq
We want to find the critical condition for the onset of magnetism. This occurs when \il{m \ra 0^+}. This means we can expand the \il{\cot} around \il{m=0}. Since
\beq
\cot (a+x) \approx \cot a -x\rr{\sin a}^{-2} \implies \cot (a-x) - \cot(a+x) \approx 2x\rr{\sin a}^{-2}
\eeq
we get
\beq[final]
m = \fr{\Delta}{U}\qq{-\pi\fr{ m}{\sin^2 \fr{\pi}{2} n_d}} \implies 1 = \lim_{m \ra 0}\fr{U}{\pi \Delta}\fr{1}{1+\cot^2\fr{\pi n_d}{2}}
\eeq
At \il{m=0}, \il{\avg{n_{d\ua}} = \avg{n_{d\da}}}, therefore \il{\cot \fr{\pi n_d}{2} = \fr{U n_d}{2\Delta}+\fr{\epsilon_d}{\Delta}}. Substituting in eq.~\ref{final},
\beq
1 = \fr{U_c}{\pi}\fr{\Delta}{\Delta^2+\rr{\fr{U_c n_d}{2}+\epsilon_d}^2}
\eeq
Magnetism will prevail for \il{U \geq U_c}. Comparing with eq.~\ref{density},
\beq
1 = U_c \rho_d(E=0)
\eeq
At half-filling, \il{n_d = 1} and \il{\epsilon_d = -\fr{U}{2}}, which gives
\beq
U_c = \pi \Delta
\eeq
For higher values of \il{U}, we get a value of \il{m} far from \il{0}. This provides two peaks in the density of states.
\begin{gather}
\avg{n_{d\ua}} = \fr{1+m}{2}\\
\avg{n_{d\da}} = \fr{1-m}{2}\\
\epsilon_{d\sigma} = \epsilon_d + U\avg{n_{d\overline\sigma}} = \epsilon_d + \fr{U}{2} \pm \fr{U}{2}m = \pm \fr{U}{2}m\\
\rho_d = \rho_{d\ua} + \rho_{d\da} = \fr{\Delta}{\pi}\qq{\fr{1}{\Delta^2 + \rr{E - \fr{Um}{2}}^2}+\fr{1}{\Delta^2 + \rr{E + \fr{Um}{2}}^2}}
\end{gather}
We get two Lorentzian peaks at \il{E = \pm \fr{Um}{2}}, depending on whichever polarization the impurity local moment is in.

\subsection{Some points:}
\begin{itemize}
	\item The mean field solution predicts that local moments are sustained in the limit of large \il{U} and small \il{|\epsilon_d|}.
	\item This treatment becomes faulty at low temperatures.
	\item At low temperatures, the resistivity is found to reach a minimum and then vary as \il{\ln T}.
	\item This behaviour stops at some very low temperature \il{T_K}.
	\item The temperature \il{T_K} is also that at which the magnetisation vanishes, and the susceptibility becomes constant, suggesting that the impurity spin has condensed into a singlet.
	\item Since the disappearance of the \il{\ln T} behaviour is coincident with the condensation of the spin degree of freedom, it is natural to hope that the resistivity minimum is a result of the interaction between the impurity and the conduction spins.
	\item To describe such an interaction, the way to proceed is to strip the model of the charge excitations (via a \it{Schrieffer-Wolff transformation}). The resultant Hamiltonian consists of an antiferromagnetic interaction between the itinerant spins and the impurity spin, and is called the Kondo model.
	\item Calculating the scattering rate up to second order using the Kondo model produces a logarithmic term, which explains the log-dependence.
	\item Since this perturbative treatment will fail at small temperatures (where the log term diverges), we need some other technique to find out the fate of the model at low temperatures.
	\item Anderson's poor man's scaling wraps the effects of high energy scatterings into the low energy model, showing that the antiferromagnetic coupling diverges at low temperatures, producing a singlet.
	\item There are two routes that one can follow to note the changes in the system; one is by reducing the temperature which is equivalent to folding in the high energy fluctuations, aka scaling. The other is to reduce the onsite interaction \il{U} and note the changes in state.
	\item Reducing the temperature or performing the RG takes the model from the Anderson model (\il{T>0}) to the Fermi liquid state (\il{T \sim T_K}). This Fermi liquid may have interactions, depending on the value of \il{U} we are working in.
	\item Coming down to \il{T<T_K}, we can now modify the \il{U} from \il{\infty} to 0. Large \il{U} means the Fermi liquid has large interactions. Reducing \il{U} means coming down to a Fermi gas. For \il{T\neq 0}, reducing \il{U} means going from local moment regime to non-magnetic regime. For \il{T=0}, local moments persist for all \il{U>0}.
	\item It will be seen that in the large \il{U} regime, the singlet channel scattering phase shift (phase shift incurred when one singlet state scatters into another singlet state) at the Fermi energy is \il{\propto \tan^{-1} J_\text{eff}}. This effective coupling \il{J_\text{eff}} flows to \il{\infty} under poor man's scaling as \il{T \ra 0}. Thus, the singlet phase shift at \il{\epsilon_F} approaches \il{\fr{\pi}{2}} as \il{T \ra 0}.
\end{itemize}

\subsection{Derivation of the Kondo Hamiltonian:}
The space of the impurity electron can be divided into low energy and high energy subspaces:
\beq
\text{low energy (L)} \ra \begin{cases} \ket{\ua} \\ \ket{\da} \end{cases}\\
\text{high energy (H)} \ra \begin{cases} \ket{} \\ \ket{\ua\da} \end{cases}\\
\eeq
\beq
H = H_0 + V = \bordermatrix{~ & \text{low} & \text{high} \cr 
\text{low} & H^L & v^\dagger \cr
	   &&\cr
\text{high} & v & H^H }
\eeq
\beq
H_0 = \sum_{k}\epsilon_k n_{k}+ \epsilon_d n_d + U n_{d\ua}n_{d\da}, V=\sum_{k\sigma}\rr{V_k c^\dagger_{k\sigma}c_{d\sigma} +V_k^* c^\dagger_{d\sigma}c_{k\sigma}}
\eeq
Let \il{S} be some anti-Hermitian operator, of the order of \il{V}. Expanding in powers of \il{V},
\beq
\ol H = e^{-S} H e^S = H_0 + \rr{V+\qq{H_0,S}} + \fr{1}{2}\rr{\qq{V,S}+\qq{\qq{H_0,S},S}}
\eeq
Defining \il{S} such that the first order term vanishes,
\begin{gather}
	V = \qq{S,H_0} \label{sdef}\\
\ol H = H_0 + \fr{1}{2}\qq{V,S}
\end{gather}
Take \il{S = \begin{pmatrix} 0 & -s^\dagger \\ s & 0 \end{pmatrix}}. From eq.~\ref{sdef},
\beq
V = \begin{pmatrix} 0 & -s^\dagger \\ s & 0 \end{pmatrix} \begin{pmatrix} H^L & 0 \\ 0 & H^H \end{pmatrix} - \begin{pmatrix} H^L & 0 \\ 0 & H^H \end{pmatrix} \begin{pmatrix} 0 & -s^\dagger \\ s & 0 \end{pmatrix} \\= \begin{pmatrix} 0 & -s^\dagger H^H+H^L s^\dagger \\ s H^L - H^H s & 0 \end{pmatrix}
\eeq 
Comparing with the definition of \il{V}, we can write
\begin{gather}
v^\dagger_{ij} = s^\dagger_{ij}\rr{E^L_i - E^H_j}, v_{ij} = s_{ij}\rr{E^L_j - E^H_i}\\
\implies s^\dagger_{ij} = \fr{v^\dagger_{ij}}{E^L_i - E^H_j}, s_{ij} = \fr{v_{ij}}{E^L_j - E^H_i}
\end{gather}
From the structure of \il{S}, it is clear that \il{i \in H, j \in L}.
\beq
\qq{V,S} = \begin{pmatrix} 0 & v^\dagger \\ v & 0 \end{pmatrix}\begin{pmatrix} 0 & -s^\dagger \\ s & 0 \end{pmatrix} - \begin{pmatrix} 0 & -s^\dagger \\ s & 0 \end{pmatrix}\begin{pmatrix} 0 & v^\dagger \\ v & 0 \end{pmatrix} = \begin{pmatrix} v^\dagger s + s^\dagger v & 0 \\ 0 & -vs^\dagger -sv^\dagger \end{pmatrix}
\eeq
Hence,
\beq
\ol H = H_0 + \fr{\qq{V,S}}{2} = \begin{pmatrix} H^L + \fr{1}{2}\rr{v^\dagger s + s^\dagger v} & 0 \\ 0 & H^H -vs^\dagger -sv^\dagger \end{pmatrix}
\eeq
Since we want the low energy excitations, the effective low-energy Hamiltonian is
\beq
\ham = \bra{L} \ol H \ket{L} = H^L + \fr{1}{2}\rr{v^\dagger s + s^\dagger v}
\eeq
where \il{H^L = \sum_\sigma \bra{\sigma_d} H_0 \ket{\sigma_d} = \epsilon_d n_d + \sum_{k} n_{k}}. Now,
\beq
\Delta H = \fr{1}{2}\rr{v^\dagger s + s^\dagger v} &= \fr{1}{2}\rr{v^\dagger \sum_{HL} s_{HL}\ket{H}\bra{L} + \text{h.c.}} \\
&= \fr{1}{2}\sum_{HL}\qq{v^\dagger \ket{H}\bra{L}\fr{v_{HL}}{E_L - E_H} + \ket{L}\bra{H} \fr{v^\dagger_{LH}}{E_L - E_H}v}
\eeq
Taking a matrix element between two low energy states \il{l, l^\prime}, we get
\beq
\Delta H_{ll^\prime} = \bra{l} \Delta H \ket{l^\prime} &= \fr{1}{2}\sum_H v^\dagger_{lH}v_{Hl^\prime}\rr{\fr{1}{E_{l^\prime} - E_H}+\fr{1}{E_l - E_H}}
\eeq
This can also be written as
\beq[hamtmat]
\Delta H_{ll^\prime} = \fr{1}{2}\qq{T_{ll^\prime}(E_l) + T_{ll^\prime}(E_{l^\prime})}
\eeq
where 
\beq
T_{ll^\prime}(E) = \sum_H \fr{v^\dagger_{lH}v_{Hl^\prime}}{E-E_H} = \sum_H \fr{V^\dagger_{lH} V_{Hl^\prime}}{E-E_H}
\eeq
\il{T(E)}, here, is the second order contribution of the \il{T-}matrix due to scattering off the interaction \il{V}. The \il{\ket{H}} act as the intermediate states during the second order scatterings. This is a slight generalization from second order perturbation theory. In second order perturbation, we only consider the scattering amplitude between the same states, but here we consider the scattering between two potentially different states \il{\ket{l},\ket{l^\prime}}. The total amplitude is an average of these two amplitudes.
\\\\If we assume the high energy subspace is very far away from the low energy one (\il{E_H \gg E_L}), we can assume \il{E_l \approx E_{l^\prime} = E_L}, we can write
\beq
\Delta H_{ll^\prime} &=\sum_H v^\dagger_{lH}v_{Hl^\prime}\fr{1}{E_L-E_H}\\
\implies \Delta H &=V \rr{\sum_H \fr{1}{\Delta_{LH}}\ket{H}\bra{H}}V
\eeq
where \il{\Delta_{LH}=E_L - E_H} is the energy difference between the low energy subspace and the high energy state \il{\ket{H}}. For our Hamiltonian, \il{\ket{H_1} = \ket{0}, \ket{H_2} = \ket{\ua\da}}. Therefore,
\beq
\Delta_{LH_1} = \epsilon_d - 0 = \epsilon_d, \Delta_{LH_2} = \epsilon_d - \rr{2\epsilon_d + U} = -\epsilon_d - U
\eeq
Also, \il{V = \sum_{k\sigma}\qq{V(k) c^\dagger_{k\sigma}c_{d\sigma} + V^*(k) c^\dagger_{d\sigma}c_{k\sigma}}}. Hence,
\beq
\Delta H &= V\fr{\ket{0}\bra{0}}{\epsilon_d}V - V\fr{\ket{\ua\da}\bra{\ua\da}}{\epsilon_d + U}V\\
	 &= \sum_{k_1,k_2,\sigma_1,\sigma_2}V(k_1)V^*(k_2)\qq{\fr{c^\dagger_{d\sigma_2} c_{k_2 \sigma_2}\ket{0}\bra{0}c^\dagger_{k_1\sigma_1} c_{d \sigma_1}}{\epsilon_d} - \fr{c^\dagger_{k_1\sigma_1} c_{d \sigma_1}\ket{\ua\da}\bra{\ua\da}c^\dagger_{d\sigma_2} c_{k_2 \sigma_2}}{\epsilon_d+U}}\\
&=\sum_{k_1,k_2,\sigma_1,\sigma_2}V(k_1)V^*(k_2)\fr{c^\dagger_{d\sigma_2} c_{k_2 \sigma_2}c^\dagger_{k_1\sigma_1} c_{d \sigma_1}\ket{d\sigma_1,h_{k_1\sigma_1}}\bra{d\sigma_1,h_{k_1\sigma_1}}}{\epsilon_d} \\
&- \sum_{k_1,k_2,\sigma_1,\sigma_2}V(k_1)V^*(k_2)\fr{c^\dagger_{k_1\sigma_1} c_{d \sigma_1}c^\dagger_{d\sigma_2} c_{k_2 \sigma_2}\ket{d\ol{\sigma_2},e_{k_2\sigma_2}}\bra{d\ol{\sigma_2},e_{k_2\sigma_2}}}{\epsilon_d+U}\\
&=\sum_{k_1,k_2,\sigma_1,\sigma_2}V(k_1)V^*(k_2)\qq{\fr{c^\dagger_{d\sigma_2} c_{k_2 \sigma_2}c^\dagger_{k_1\sigma_1} c_{d \sigma_1}}{\epsilon_d} - \fr{c^\dagger_{k_1\sigma_1} c_{d \sigma_1}c^\dagger_{d\sigma_2} c_{k_2 \sigma_2}}{\epsilon_d+U}}P_{n_d=1}
\eeq
Using Fierz indentity \il{\delta_{\sigma_1\sigma_3}\delta_{\sigma_4\sigma_2} = \fr{1}{2}\delta_{\sigma_1\sigma_2}\delta_{\sigma_3\sigma_4} + \fr{1}{2}\vec\sigma_{\sigma_1\sigma_2}\cdot\vec\sigma_{\sigma_3\sigma_4}}, we can write
\beq
c^\dagger_{d\sigma_2} c_{k_2 \sigma_2}c^\dagger_{k_1\sigma_1} c_{d \sigma_1} &= \sum_{\sigma_3,\sigma_4}c^\dagger_{d\sigma_3} c_{k_2 \sigma_2}c^\dagger_{k_1\sigma_1} c_{d \sigma_4}\delta_{\sigma_1\sigma_3}\delta_{\sigma_4\sigma_2}\\
&=\fr{1}{2}\sum_{\sigma_3,\sigma_4}c^\dagger_{d\sigma_3} c_{k_2 \sigma_2}c^\dagger_{k_1\sigma_1} c_{d \sigma_4}\rr{\delta_{\sigma_1\sigma_2}\delta_{\sigma_3\sigma_4} + \vec\sigma_{\sigma_1\sigma_2}\cdot\vec\sigma_{\sigma_3\sigma_4}}\\
&=\fr{1}{2}c_{k_2 \sigma_1}c^\dagger_{k_1\sigma_1}n_d+c_{k_2 \sigma_2}c^\dagger_{k_1\sigma_1}\vec\sigma_{\sigma_1\sigma_2}\cdot\sum_{\sigma_3,\sigma_4}c^\dagger_{d\sigma_3}\fr{\vec \sigma_{\sigma_3\sigma_4}}{2}c_{d\sigma_4}
\eeq
Now, \il{c_{k_2 \sigma_1}c^\dagger_{k_1\sigma_1} = \delta_{k_1,k_2}-c^\dagger_{k_1\sigma_1}c_{k_2 \sigma_1}}, and  \il{c_{k_2 \sigma_2}c^\dagger_{k_1\sigma_1} = \delta_{\sigma_1,\sigma_2}\delta_{k_1,k_2}-c^\dagger_{k_1\sigma_1}c_{k_2 \sigma_1}}. The \il{\delta} will result in terms that have no interaction, so we drop these terms. Also, the \il{P_{n_d=1}} ensures we can substitute \il{n_d=1}.
\beq
c^\dagger_{d\sigma_2} c_{k_2 \sigma_2}c^\dagger_{k_1\sigma_1} c_{d \sigma_1} &= -\fr{1}{2}c^\dagger_{k_1\sigma_1}c_{k_2 \sigma_1} - c^\dagger_{k_1\sigma_1}\vec\sigma_{\sigma_1\sigma_2}c_{k_2 \sigma_2}\cdot\sum_{\sigma_3,\sigma_4}c^\dagger_{d\sigma_3}\fr{\vec\sigma_{\sigma_3\sigma_4}}{2}c_{d\sigma_4}\\
\eeq
Since the first term does not have any spin-spin interaction, we drop that term. 
Defining \il{\vec \sigma_d = \sum_{\sigma_3,\sigma_4}c^\dagger_{d\sigma_3}\vec\sigma_{\sigma_3\sigma_4}c_{d\sigma_4}}, we have
\beq
c^\dagger_{d\sigma_2} c_{k_2 \sigma_2}c^\dagger_{k_1\sigma_1} c_{d \sigma_1} =-\fr{1}{2}c^\dagger_{k_1\sigma_1}\vec\sigma_{\sigma_1\sigma_2}c_{k_2 \sigma_2}\cdot \vec \sigma_d
\eeq
Similarly,
\beq
c^\dagger_{k_1\sigma_1} c_{d \sigma_1} c^\dagger_{d\sigma_2} c_{k_2 \sigma_2}=-\fr{1}{2}c^\dagger_{k_1\sigma_1}\vec\sigma_{\sigma_1\sigma_2}c_{k_2 \sigma_2}\cdot \vec \sigma_d
\eeq
Finally, putting all this together,
\beq
\Delta H = \fr{1}{2}\sum_{k_1,k_2,\sigma_1,\sigma_2}V(k_1)V^*(k_2)\qq{\fr{1}{\epsilon_d+U}-\fr{1}{\epsilon_d}}c^\dagger_{k_1\sigma_1}\vec\sigma_{\sigma_1\sigma_2}c_{k_2 \sigma_2}\cdot \vec \sigma_d \\
= \fr{1}{2}\sum_{k_1,k_2,\sigma_1,\sigma_2} J(k_1,k_2)c^\dagger_{k_1\sigma_1}\vec\sigma_{\sigma_1\sigma_2}c_{k_2 \sigma_2}\cdot \vec \sigma_d
\eeq
where
\beq
J(k_1,k_2) = V(k_1)V^*(k_2)\qq{\fr{1}{\epsilon_d+U}-\fr{1}{\epsilon_d}}
\eeq
Assuming \il{V(k) \equiv t},
\beq
H_K = \sum_k \epsilon_k n_k + \fr{J}{2} \vec \sigma_e \cdot \vec \sigma_d
\eeq
where
\beq
\vec \sigma_e = \sum_{k_1,k_2,\sigma_1,\sigma_2}c^\dagger_{k_1\sigma_1}\vec\sigma_{\sigma_1\sigma_2}c_{k_2 \sigma_2} = \sum_{\sigma_1,\sigma_2}c^\dagger_{\sigma_1}(\vec r = 0)\vec\sigma_{\sigma_1\sigma_2}c_{\sigma_2}(\vec r = 0)
\eeq
\il{\vec \sigma_e} is thus the spin density at the origin.

\subsection{Obtaining the resistivity minimum and \il{\log}-dependence}
The model we are working with is
\beq
H_K &= H_0 + V = \sum_k \epsilon_k n_k + \fr{J}{2} \sum_{k_1,k_2,\sigma_1,\sigma_2}c^\dagger_{k_1\sigma_1}\vec \sigma_d \cdot \vec\sigma_{\sigma_1\sigma_2}c_{k_2 \sigma_2}
\eeq
\beq
\sum_{\sigma_1,\sigma_2}c^\dagger_{k_1\sigma_1}\vec \sigma_d \cdot \vec\sigma_{\sigma_1\sigma_2}c_{k_2 \sigma_2} = \sigma_d^z\rr{c^\dagger_{k_1\ua}c_{k_2\ua} - c^\dagger_{k_1\da}c_{k_2\da}} +\sigma_d^x\rr{c^\dagger_{k_1\da}c_{k_2\ua} + c^\dagger_{k_1\ua}c_{k_2\da}} \\
-i \sigma_d^y\rr{ c^\dagger_{k_1\ua}c_{k_2\da}- c^\dagger_{k_1\da}c_{k_2\ua}}
\eeq
\beq
=\sigma_d^z\rr{c^\dagger_{k_1\ua}c_{k_2\ua} - c^\dagger_{k_1\da}c_{k_2\da}} + c^\dagger_{k_1\da}c_{k_2\ua}\sigma_d^+ + c^\dagger_{k_1\ua}c_{k_2\da}\sigma_d^-
\eeq
where \il{\sigma^\pm = \sigma^x \pm i \sigma^y}. Therefore,
\beq
H_K &=\sum_k \epsilon_k n_k + \fr{J}{2} \sum_{k_1,k_2}\qq{\sigma_d^z\rr{c^\dagger_{k_1\ua}c_{k_2\ua} - c^\dagger_{k_1\da}c_{k_2\da}}+\sigma_d^+ c^\dagger_{k_1\da}c_{k_2\ua} + \sigma_d^- c^\dagger_{k_1\ua}c_{k_2\da}}\\
    &=\sum_k \epsilon_k n_k + J \sum_{k_1,k_2}\qq{S_d^z\rr{c^\dagger_{k_1\ua}c_{k_2\ua} - c^\dagger_{k_1\da}c_{k_2\da}}+S_d^+ c^\dagger_{k_1\da}c_{k_2\ua} + S_d^- c^\dagger_{k_1\ua}c_{k_2\da}}
\eeq
To see the \il{\log-}dependence, we need to calculate the transition matrix up to second order:
\beq
T = V + V G_0 V
\eeq
We wish to calculate the scattering probability of a conduction electron \il{\ket{k \ua}}.
\subsubsection*{First order scattering}
\begin{center}
$\left.\begin{tabular}{@{}l@{}}
\il{\ket{k \ua, d_\sigma} \ra \ket{q \ua, d_\sigma}}
%\il{\ket{k \da, d_\sigma} \ra \ket{q \da, d_\sigma}}\\
\end{tabular}\right\}$ non-spin-flip\\[10pt]
$\left.\begin{tabular}{@{}l@{}}
\il{\ket{k \ua, d_\da} \ra \ket{q \da, d_\ua}}
%\il{\ket{k \da, d_\ua} \ra \ket{q \ua, d_\da}}\\
\end{tabular}\right\}$ pro-spin-flip
\end{center}
For non-flip, the matrix elements for the \il{T-}matrix is
\beq
T^{(1)}_\text{nonflip} = T_{k_\ua,d_{\sigma} \ra q_\ua,d_{\sigma}} = \bra{q_\ua,d_{\sigma}}V\ket{k_\ua,d_{\sigma}} = m_d J
\eeq
where \il{m_d \in \{-s_d, s_d\}} is the spin of the impurity electron. The probability for this scattering is
\beq
\mathcal{P}_{k_\sigma,d_{\sigma^\prime} \ra q_\sigma,d_{\sigma^\prime}} = 2\pi\sum_{\epsilon} \rho(\epsilon)T_{k_\ua,d_{\sigma} \ra q_\ua,d_{\sigma}}^2 = 2\pi \rho(0) J^2 m_d^2
\eeq
Since we are considering scattering close to the Fermi surface, we replaced the sum with \il{\rho(0)}. 
\beq
\mathcal{P}_1 = 2\pi\rho(0)J^2 m_d^2
\eeq
For spin-flip, the matrix element is
\beq
T^{(1)}_\text{flip} =T_{k_\ua,d_{\da} \ra q_\da,d_{\ua}} = \bra{q_\da,d_{\ua}}V\ket{k_\ua,d_{\da}} = \lambda_+ J
\eeq
where \il{\lambda_\pm = \bra{m_d \pm 1} S_d^\pm \ket{m_d} = \sqrt{s_d(s_d+1)-m_d(m_d\pm 1)}}. The probability for this scattering is hence
\beq
\mathcal{P}_2 = \mathcal{P}_{k_\ua,d_{\da} \ra q_\da,d_{\ua}} = 2\pi \rho(0) J \qq{s_d(s_d+1)-m_d(m_d + 1)}
\eeq
The total first order scattering probability is (averaged over all configurations of the impurity)
\beq
\mathcal{P}^{(1)} = \fr{1}{2s_d+1}\sum_{m_d = -s_d}^{s_d}\rr{\mathcal{P}_1 + \mathcal{P}_2} = \fr{2\pi \rho(0) J^2}{(2s_d+1)}\sum_{m_d = -s_d}^{s_d}\rr{s_d(s_d+1) - m_d} \\
= 2\pi \rho(0)J^2 s_d(s_d+1)
\eeq
\subsubsection*{Second order scattering}

\begin{center}
$\left.\begin{tabular}{@{}l@{}}
\text{no-impurity-flip}\il{\begin{cases}
\ket{k \ua, d_\sigma} \ra \ket{q \ua, d_\sigma} \ra \ket{k^\prime \ua, d_\sigma}\\
\ket{k \ua, q \ua, d_\sigma} \ra \ket{k \ua, k^\prime \ua, d_\sigma} \ra \ket{k^\prime \ua,q \ua, d_\sigma}
\end{cases}}\\[30pt]
\text{pro-impurity-flip}\il{\begin{cases} 
\ket{k \ua, d_\da} \ra \ket{q \da, d_\ua} \ra \ket{k^\prime \ua, d_\da}\\
\ket{k \ua, q \da, d_\ua} \ra \ket{k \ua, k^\prime \ua, d_\da} \ra \ket{k^\prime \ua,q \da, d_\ua}
\end{cases}}
\end{tabular}\right\}$ no-cond-flip\\[40pt]
$\left.\begin{tabular}{@{}l@{}}
\text{flip-first}\il{\begin{cases} 
\ket{k \ua, d_\da} \ra \ket{q \da, d_\ua} \ra \ket{k^\prime \da, d_\ua}\\
\ket{k \ua, q \ua, d_\da} \ra \ket{k \ua, k^\prime \da, d_\ua} \ra \ket{k^\prime \da,q \ua, d_\ua}
\end{cases}}\\[30pt]
\text{flip-later}\il{\begin{cases} 
\ket{k \ua, d_\da} \ra \ket{q \ua, d_\da} \ra \ket{k^\prime \da, d_\ua}\\
\ket{k \ua, q \da, d_\da} \ra \ket{k \ua, k^\prime \da, d_\da} \ra \ket{k^\prime \da,q \da, d_\ua}
\end{cases}}\\
\end{tabular}\right\}$ pro-cond-flip
\end{center}
The second order transition matrix contribution is of the form
\beq
T^{(2)}_{i \ra j} = \bra{j} V G_0 V \ket{i} = \sum_l \fr{\bra{j}V\ket{l}\bra{l}V\ket{i}}{E_i - E_l}
\eeq
The sum is over all the intermediate states in going from \il{\ket{i}} to \il{\ket{k}}. For no flipping of the conduction electron, there are four possible processes. The first process has the following \it{T}-matrix:
\beq
T^{(2)}_{11}&=\sum_q\fr{\bra{k^\prime_\ua d_\sigma}V\ket{q_\ua d_\sigma}\bra{q_\ua d_\sigma}V\ket{k_\ua d_\sigma}}{\epsilon_k-\epsilon_q}\\
      &= \rr{J m_d}^2\sum_q \fr{1-P(q)}{\epsilon_k - \epsilon_q} = J^2 m_d^2 \sum_q \fr{1-P(q)}{\epsilon_k - \epsilon_q}
\eeq
where \il{m_d = \bra{d_\sigma}S_d^z\ket{d_\sigma}} and \il{1-P(q)} is the probability that the state \il{q\ua} is empty. For the second process,
\beq
T^{(2)}_{12} = \sum_q \fr{\bra{q_\ua k^\prime_\ua d_\sigma} V \ket{k^\prime_\ua k_\ua d_\sigma}\bra{k^\prime_\ua k_\ua d_\sigma}V\ket{q_\ua k_\ua d_\sigma}}{\epsilon_q - \epsilon_{k^\prime}}P(q)
\eeq
Note that if \il{\bra{k^\prime k}V\ket{q k} \sim \bra{k^\prime k} c^\dagger_{k^\prime}c_q\ket{q k} = 1}, then \il{\bra{q k^\prime }V\ket{k^\prime k} \sim \bra{qk^\prime }c^\dagger_q c_k\ket{k^\prime k} =}\\\il{ -\bra{qk^\prime }c^\dagger_q c_k\ket{k k^\prime} = -1}. Assuming the scattering conserves energy \il{(\epsilon_k = \epsilon_k^\prime)}, we get
\beq
T^{(2)}_{12} = -J^2 m_d^2\sum_q\fr{P(q)}{\epsilon_q - \epsilon_{k}} = J^2 m_d^2 \sum_q \fr{P(q)}{\epsilon_k - \epsilon_q}
\eeq
For the third process,
\beq
T^{(2)}_{13}=\sum_q\fr{\bra{k^\prime_\ua d_\da}V\ket{q_\da d_\ua}\bra{q_\da d_\ua}V\ket{k_\ua d_\da}}{\epsilon_k-\epsilon_q}
\eeq
Using \il{\bra{m_d\pm 1}S_d^\pm\ket{m_d} = \sqrt{s_d(s_d+1)-m_d(m_d\pm 1)} = \lambda_\pm}, we get
\beq
T^{(2)}_{13} = \lambda_+^2J^2 \sum_q \fr{1-P(q)}{\epsilon_k-\epsilon_q}
\eeq
For the fourth process,
\beq
T^{(2)}_{14}&=\sum_q\fr{\bra{q_\da k^\prime_\ua d_\ua}V\ket{k^\prime_\ua k_\ua d_\da}\bra{k^\prime_\ua k_\ua d_\da}V\ket{q_\da k_\ua d_\ua}}{\epsilon_q-\epsilon_k^\prime}\\
      &=-\lambda_-^2J^2 \sum_q \fr{P(q)}{\epsilon_q-\epsilon_k}\\
      &=\lambda_-^2J^2 \sum_q \fr{P(q)}{\epsilon_k-\epsilon_q}
\eeq
The sum of all the elements gives the transition matrix element for the scattering \il{k\ua \ra k^\prime\ua}:
\beq
T^{(2)}_{\text{nonflip}} = \sum_{i=1}^4 T^{(2)}_{1i} &= J^2 \sum_q \fr{m_d^2 + \lambda_+^2 -P(q)\rr{\lambda_+^2 - \lambda_-^2}}{\epsilon_k-\epsilon_q}\\
&= J^2 \sum_q \fr{s(s+1)-m_d + 2m_dP(q)}{\epsilon_k-\epsilon_q}\\
&= J^2\qq{s(s+1)-m_d}(\alpha+\gamma) + 2 J^2 m_d \gamma
\eeq
where \il{\gamma = \sum_q\fr{P(q)}{\epsilon_k-\epsilon_q},\alpha = \sum_q\fr{1-P_q}{\epsilon_k- \epsilon_q}}. The second term has the Fermi-Dirac distribution and hence is the only temperature dependent term. Accordingly, we drop the first term.
\beq
T^{(2)}_{\text{nonflip}} &= 2 J^2 m_d \gamma\\
			 &=2J^2 m_d \int d\epsilon N(\epsilon) \fr{P(\epsilon)}{\epsilon_k - \epsilon} = \fr{\sqrt 2 J^2 m_d m^{\fr{3}{2}}}{\pi^2 \hbar^3}\int d\epsilon \fr{\sqrt \epsilon P(\epsilon)}{\epsilon_k - \epsilon}
\eeq
Assuming \il{T=0}, \il{P(\epsilon) = \theta(\epsilon_F - \epsilon)}. Then
\beq
T^{(2)}_{\text{nonflip}} &=\fr{\sqrt 2 J^2 m_d m^{\fr{3}{2}}}{\pi^2 \hbar^3}\sqrt{\epsilon_k}\ln \bigg\vert \fr{\sqrt{\epsilon_k}+\sqrt{\epsilon_{F}}}{\sqrt {\epsilon_k}-\sqrt{\epsilon_{F}}} \bigg\vert \\
    &= \fr{\sqrt 2 J^2 m_d m^{\fr{3}{2}}}{\pi^2 \hbar^3}\sqrt{\epsilon_k} \ln \bigg\vert \fr{\epsilon_k + \epsilon_{F} + 2\sqrt{\epsilon_k \epsilon_F}}{\epsilon_k-\epsilon_{F}} \bigg\vert 
\eeq
For \il{T >0} but \il{\ll T_F}, the excitation energy of the electrons is very small and of the order of \il{k_B T}. Hence, we can replace \il{\epsilon_k - \epsilon_F = k_B T} and everywhere else replace \il{\epsilon_k = \epsilon_F}.

\beq
T^{(2)}_{\text{nonflip}} = \fr{\sqrt 2 J^2 m_d m^{\fr{3}{2}}}{\pi^2 \hbar^3}\sqrt{\epsilon_F} \ln \bigg\vert \fr{4T_F}{T} \bigg\vert 
\eeq
Dropping the temperature-independent \il{\log 4} term and recognizing \il{N(\epsilon_F)} in the pre-factor,
\beq
T^{(2)}_{\text{nonflip}} = 2J^2 m_d N(\epsilon_F) \ln \bigg\vert \fr{T_F}{T} \bigg\vert 
\eeq
Adding the first order non-flip contribution (\il{T^{(1)}_\text{nonflip}}) to the \il{T-}matrix, we get
\beq
T_\text{nonflip} = J m_d\qq{1+2N(\epsilon_F) J \ln \fr{T_F}{T}}
\eeq
The upshot is that the additional contribution in second order is obtained by replacing \il{J \ra 2J N(\epsilon_F) \ln \fr{T_F}{T}}. For the spin-flip scatterings (processes 5\uu{th} to 8\uu{th}),
\beq
T^{(2)}_{21} &= -J^2 (m_d+1)\lambda_+\sum_q\fr{1-P_q}{\epsilon_k- \epsilon_q}\\
T^{(2)}_{23} &= J^2 m_d \lambda_+\sum_q\fr{1-P_q}{\epsilon_k- \epsilon_q}\\
T^{(2)}_{22} &= J^2 (m_d+1) \lambda_+  \sum_q \fr{P(q)}{\epsilon_k - \epsilon_q}\\
T^{(2)}_{24} &= -\lambda_+ m_d J^2 \sum_q \fr{P(q)}{\epsilon_k-\epsilon_q}
\eeq
\beq
T^{(2)}_\text{flip} = -J^2 \lambda_+ \rr{\alpha - \gamma}
\eeq
The total spin-flip matrix element (temperature-dependent part) is
\beq
T^{(2)}_\text{flip} &= 2 J^2 \lambda_+ \sum_q \fr{P(q)}{\epsilon_k - \epsilon_q} \\
     &= 2 J^2 \lambda_+ N(\epsilon_F) \ln \bigg\vert \fr{T_F}{T} \bigg\vert 
\eeq
Adding the first order contribution,
\beq
T_\text{flip} = \lambda_+ J \qq{1 + 2 N(\epsilon_F) J \ln \fr{T}{T_F}}
\eeq
Here again, the second order contribution is obtained by replacing \\
\il{J \ra 2J N(\epsilon_F) \ln \fr{T_F}{T}}. Both the solutions together imply that the next order probability for scattering of \il{k\ua} is obtained by replacing the additional \il{J} with \il{2J N(\epsilon_F) \ln \fr{T_F}{T}}.
\beq[change]
\mathcal{P} = \mathcal{P}^{(2)}\qq{1 + 2J N(\epsilon_F) \ln \fr{T_F}{T}}
\eeq

\subsection{The Kondo resonance}
Since \il{V} conserves total angular momentum, \il{\bra{s}V\ket{s^\prime} \sim \delta_{s s^\prime}}. Hence
\beq
T_{a \ra b} = \sum_{s,m_s} |\langle{s,m_s}|{a}\rangle|^2 T_s
\eeq
Now, \il{\ket{k \ua, d_\ua} = \ket{s=1}}, so
\beq
T_{\ket{k \ua, d_\ua} \ra \ket{k^\prime \ua d_\ua}} = T_1
\eeq
But. since \il{\ket{k \ua, d_\da} = \fr{\ket{s=1}+\ket{s=0}}{\sqrt 2}},
\beq
T_{\ket{k \ua, d_\da} \ra \ket{k^\prime \ua d_\da}} = \fr{T_1 + T_0}{2}
\eeq
and \il{\ket{k \da, d_\ua} = \fr{\ket{s=1}-\ket{s=0}}{\sqrt 2}},
\beq
T_{\ket{k \ua, d_\da} \ra \ket{k^\prime \da d_\ua}} = \fr{T_1 - T_0}{2}
\eeq
Therefore,
\beq
T_1 = T_{\ket{k \ua, d_\da} \ra \ket{k^\prime \ua d_\da}} + T_{\ket{k \ua, d_\da} \ra \ket{k^\prime \da d_\ua}} = T_\text{nonflip} + T_\text{flip}\\
T_0 = T_{\ket{k \ua, d_\da} \ra \ket{k^\prime \ua d_\da}} - T_{\ket{k \ua, d_\da} \ra \ket{k^\prime \da d_\ua}} = T_\text{nonflip} - T_\text{flip}\\
\eeq
Assuming spin-half impurity, (\il{s=\fr{1}{2}})
\begin{gather}
T_\text{nonflip} =J\qq{m_d + \fr{J}{4}\cc{3(\alpha + \gamma) + 4m_d( \gamma - \alpha)}} \\
T_\text{flip} = J\qq{1 + J\rr{\gamma - \alpha}}
\end{gather}
Setting \il{m_d = -\fr{1}{2}},
\beq[tmatrix]
T_1 &= \fr{J}{2}\qq{1+\fr{J}{2}\rr{\alpha + 5 \gamma}}\\
T_0 &= -\fr{3J}{2}\qq{1-\fr{3J}{2}\rr{\alpha - \fr{\gamma}{3}}}
\eeq
The value of the prefactors can be understood as follows: The interaction term is 
\beq
J \vec S_d \cdot \vec \sigma_e = 2J \vec S_d \cdot \vec S_e = J \rr{S^2 - S_d^2 - S_e^2} = J \rr{s(s+1) - \fr{3}{2}} = \begin{cases} -\fr{3J}{2} &\text{(singlet)}\\ \fr{J}{2} &\text{(triplet)}\end{cases}
\eeq
Hence, the pre-factors are just the bare values of the interaction Hamiltonian, \il{V}. Hence, the equations \ref{tmatrix} can be written as 
\beq
T = V(1 + TG)
\eeq
For the singlet and triplet \il{T-}matrices, it becomes
\beq[texp]
T_1 &= \fr{J}{2}\qq{1+T_1\rr{\alpha + 5 \gamma}} \implies T_1 = \fr{J/2}{1-\fr{J}{2}(\alpha + 5 \gamma)}\\
T_0 &= -\fr{3J}{2}\qq{1+T_0\rr{\alpha - \fr{\gamma}{3}}} \implies T_0 = \fr{-3J/2}{1+\fr{3J}{2}(\alpha - \gamma/3)}
\eeq
We want to find the maximum value of \il{|T_s|}. To this end, rewrite
\begin{gather}
T_1 = \fr{1}{2/J - 5 \gamma - \alpha}\\
T_0 = \fr{1}{-2/3J + \gamma/2 - \alpha}
\end{gather}
For excitations \il{(k)} just above the Fermi surface, \il{\alpha} will encounter a zero in its denominator, because the integral in \il{\alpha} is outside the Fermi surface. On the other hand, the integral in \il{\gamma} is inside the Fermi surface, so the denominator in \il{\gamma} will never become zero for \il{k} just outside the Fermi surface.  Hence, \il{\alpha = \text{real part} - i\pi N(0), \gamma =\text{real part}}. Accordingly, the expressions for \il{T_s} can be written as
\beq
T_s = \fr{1}{\text{real part} + i \pi N(0)}
\eeq
The maximum value of \il{|T_s|} will occur when the denominator is minimum, that is, when \il{\text{real part} = 0}. Hence,
\beq
|T_s| \leq \fr{1}{\pi N_0}
\eeq
From eq.~\ref{tmatphase}, we can write
\beq
T_s = -\fr{e^{i \delta_s}\sin \delta_s}{\pi N(0)}
\eeq
Eq.~\ref{dsfromtmat} allows us to write
\beq
\rho_{d\sigma}(0) = -\fr{\text{Im}[T]}{t^2 \pi} = \fr{\sin^2 \delta_s}{t^2 \pi^2 N(0)} = \fr{\sin^2 \delta_s}{\pi \Delta} = \fr{1}{\pi \Delta}\sin^2 \rr{\fr{\pi n_c}{2}}
\eeq
where \il{n_c = \avg{n_{d\ua}+n_{d\da}}}. This is in contrast to the value obtained from the mean field analysis of the Anderson model, eq,~\ref{rho},
\beq
\rho_{d\sigma}(0) = \fr{1}{\pi \Delta}\qq{1 + \rr{\fr{\epsilon_d + Un_c}{\Delta}}^{2}}^{-1}
\eeq
For \il{n_c =1} (half-filling), the mean field value is less than the one obtained from the spin-spin scattering. This is because the mean-field analysis does not take these scatterings into account. The large density of states at the Fermi level means that the spectral function has three peaks in general, two of which are revealed in the mean field analysis, but a third one exists, which is of a width of the order of a very low temperature \il{T_K}, and hence is not noticed at higher temperatures.\\\\
Eq.~\ref{texp} can be written as
\begin{gather}
T_1 = \fr{J/2}{1 - 2 J \gamma - \fr{J}{2}\rr{\alpha + \gamma}}\\
T_0 = \fr{-3J/2}{1 - 2 J \gamma + \fr{3J}{2}\rr{\alpha + \gamma}}
\end{gather}
Defining \il{J_\text{eff} = \fr{J}{1 - 2J\gamma}}, the scattering amplitudes \il{T_1} and \il{T_0} can be written as
\beq
T_1 = \fr{1}{\fr{2}{J_\text{eff}} - \rr{\alpha + \gamma}}\\
T_0 = \fr{-1}{\fr{2}{3J_\text{eff}}+\alpha + \gamma}
\eeq
\il{\alpha + \gamma} can be calculated as 
\beq
\alpha + \gamma &= \lim_{\eta \ra 0}\int_0^\infty d\epsilon \fr{N(\epsilon)}{\epsilon_k - \epsilon+ i\eta} \\
		&\sim\lim_{\epsilon_\text{up} \ra \infty} \ln \bigg\lvert \fr{\sqrt \epsilon_k - \sqrt{\epsilon_\text{up}}}{\sqrt \epsilon_k + \sqrt{\epsilon_\text{up}}}\bigg \rvert - i \pi N(0)
\eeq
In the limit of \il{\epsilon_\text{up} \ra \infty}, the argument of the \il{\log} becomes
\beq
\bigg\lvert \fr{\sqrt \epsilon_k - \sqrt{\epsilon_\text{up}}}{\sqrt \epsilon_k + \sqrt{\epsilon_\text{up}}}\bigg \rvert \approx \bigg\lvert \fr{- \sqrt{\epsilon_\text{up}}}{\sqrt{\epsilon_\text{up}}}\bigg \rvert = 1
\eeq
Hence, the real part vanishes, and the expression for \il{T_1} becomes
\beq
T_1 = \fr{1}{2J^{-1}_\text{eff} + i \pi N(0)} \sim \fr{1}{\fr{2}{\pi N(0)J_\text{eff}} + i}
\eeq
Since
\beq
T_s \sim e^{i \delta_s} \sin \delta_s = \fr{1}{\cot \delta_s - i}
\eeq
we can write
\beq
\cot \delta_1 = -\fr{2}{\pi N(0) J_\text{eff}}\implies \tan \delta_1 = -\fr{\pi}{2} N(0) J_\text{eff}
\eeq 
Similarly,
\beq
T_0 =  \fr{-1}{\fr{2}{3 J_\text{eff}} - i\pi N(0)} \sim \fr{-1}{\fr{2}{3 J_\text{eff} \pi N(0)} - i}
\eeq
giving
\beq
\cot \delta_0 = \fr{2}{3 J_\text{eff} \pi N(0)} \implies \tan \delta_0 = \fr{3\pi}{2} J_\text{eff} N(0)
\eeq
Since \il{J_\text{eff} > 0}, \il{\delta_1 < 0} and \il{\delta_0 > 0}. The significance of this can be seen as follows. For scattering at the Fermi surface, the scattered wavefunction can be written as
\beq
\psi \sim \psi_\text{in} - e^{2 i \delta_d} \psi_\text{out}
\eeq
where \il{\psi_\text{in} = \fr{e^{ik_F r}}{r}} is the incoming wave and \il{\psi_\text{out} = \fr{e^{-ik_F r}}{r}} is the outgoing one. Hence,
\beq
\psi = \fr{e^{i \delta}}{r} \rr{e^{-i\rr{k_Fr + \delta_d}} - e^{i\rr{k_Fr + \delta_d}}} \sim \fr{e^{i \delta}}{r} \sin \qq{k_F\rr{r + \Delta r}}
\eeq
This scattered wave is thus another radial wave but its phase is shifted by an amount \il{\Delta r = \fr{\delta_d}{k_F}}. For a positive \il{\Delta r} (and hence a positive \il{\delta_d}), the wave will be drawn inward. Hence, the singlet channel having a positive \il{\delta} will lead to formation of bound states. On the  other hand, the triplet channel has a negative phase shift, meaning it is repulsive.

\subsection{Adiabatic route to the Kondo resonance}\label{adiab}
Assuming \il{T=0}, the interactions due to a single impurity are unlikely to break adiabaticity. Hence, we replace the effect of the \il{U} on the impurity by adding a self energy \il{\Sigma(\omega)} to the bare energy \il{\epsilon_d}. This self energy can be Taylor-expanded about \il{E = 0}:
\beq
\Sigma(E) = \Sigma(0) + E\dv{\Sigma}{E}\bigg \vert_{E=0} + O(E^2)
\eeq
Defining 
\beq
Z^{-1} \equiv 1 - \dv{\Sigma}{E}\bigg \vert_{E=0}
\eeq
we can write
\beq[sigma]
\Sigma(E) = \Sigma(0) + \rr{1-Z^{-1}}E
\eeq
The interacting Green's function for the impurity becomes
\beq
G_f(E) &= \fr{1}{E - \epsilon_d - \Sigma - i\Delta}\\
\eeq
where \il{\Delta} is the result of the hybridisation. Substituting eq.~\ref{sigma} and gathering the terms gives
\beq
G_f(E) = \fr{Z}{E - Z (\epsilon_d + \Sigma(0)) - iZ\Delta}
\eeq
Defining the renormalised parameters
\begin{gather}
\epsilon_d^* = Z (\epsilon_d + \Sigma(0))\\
\Delta^* = Z\Delta
\end{gather}
we have
\beq
G_f(E) = \fr{Z}{E - \epsilon_d^* - i \Delta^*}
\eeq
What this means is that as we adiabatically vary the interaction \il{U}, the parameters \il{\epsilon_d^*} and \il{\Delta^*} also morph, keeping the form of the Greens's function constant. In the non-interacting limit (\il{U=0}), we have 
\beq
Z = 1, \Sigma = 0 \implies \epsilon_d^* = \epsilon_d, \Delta^* = \Delta
\eeq
We then recover the atomic form of the Green's function. \il{Z} varies from 0 to 1. \il{Z=1} is the non-interacting limit, \il{Z=0} is the limit of \il{U=\infty}. The phase shift due to scattering can be calculated by looking at eq.\ref{phaseshift}, and replacing the bare quantities with the renormalised versions:
\beq[cotinv]
\delta_d(0) = \cot^{-1}\fr{\epsilon_d^*}{\Delta^*}
\eeq
Similarly, the renormalised version of eq.~\ref{densitys} is
\beq
\rho_d(0) = \fr{1}{\pi}\fr{\Delta^*}{{\epsilon_d^*}^2 + {\Delta^*}^2}
\eeq
Using eq.~\ref{cotinv} gives
\beq
\rho_d(0) = \fr{1}{\pi}\fr{\Delta^*}{{\Delta^*}^2\cot^2\delta_d + {\Delta^*}^2} = \fr{\sin^2 \delta_d}{\pi \Delta}
\eeq

\subsection{The Kondo temperature}
We consider a simplified model where a single conduction electron forms a singlet with the d-electron, and the rest of the conduction electrons simply fill the Fermi sea. For the singlet state, \il{\vec S_e \cdot \vec S_d = -\fr{3}{2}}. So,
\beq
H_K = \sum_{k > k_F} \epsilon_k n_k -\fr{3J}{2} \sum_{k,k^\prime > k_F} c^\dagger_{k^\prime \sigma}c_{k \sigma}
\eeq
The operator to create the singlet state \il{\ket{S_k} = \fr{1}{\sqrt 2}\rr{\ket{k\ua,d\da} - \ket{k\da,d\ua}}} off the Fermi sea (\il{\ket{\Phi}}) is
\beq
b_k^\dagger = \fr{1}{\sqrt 2}\rr{c^\dagger_{k \ua}c^\dagger_{d\da} - c^\dagger_{k \da}c^\dagger_{d\ua}}
\eeq
Hence the total wavefunction of singlet+Fermi-sea is
\beq
\ket{\Psi} = \sum_{k>k_F} a_k b_k^\dagger \ket{\Phi} = \ket{\Phi} \otimes \sum_{k > k_F}a_k \ket{S_k}
\eeq
\il{a_k} is the probability amplitude for the conduction electron in the single to have momentum \il{k}.
\beq[ak]
a_q = \bra{\Phi}\bra{S_q} \sum_k a_k \ket{S_k}\ket{\Phi} = \bra{\Phi}b_q\ket{\Psi}
\eeq
The Schrodinger equation for \il{\ket{\Psi}} is
\beq
E \ket{\Psi} = H_K \ket{\Psi} &= \ket{\Phi} \otimes H_k \sum_{k > k_F}a_k \ket{S_k}\\
			      &= \ket{\Phi} \otimes \sum_{k > k_F}a_k \rr{\epsilon_k \ket{S_k} -\fr{3J}{2}\sum_{k^\prime > k_F} \ket{S_{k^\prime}}}\\
			      &= \sum_{k>k_F} a_k \rr{\epsilon_k b^\dagger_k -\fr{3J}{2}\sum_{k^\prime > k_F}b^\dagger_{k^\prime}}\ket{\Phi}
\eeq
Multiplying \il{b_q} from left  gives
\beq
E b_q \ket{\Psi} = \epsilon_q a_q \ket{\Phi} - \fr{3J}{2}\sum_{k>k_F} a_k \ket{\Phi}
\eeq
Multiplying \il{\bra{\Phi}} from left and looking at eq.~\ref{ak} gives
\beq
E \bra{\Phi}b_q\ket{\Psi} = E a_q = a_q \epsilon_q - \fr{3J}{2}\sum_k a_k \\
\implies a_q = \fr{3J/2}{\epsilon_q - E}\sum_k a_k\\
\implies \sum_q a_q = \sum_q \fr{3J/2}{\epsilon_q - E}\sum_k a_k
\eeq
Since \il{\sum_q a_q =\sum_k a_k}, we get an equation for \il{E}
\beq
1 = \fr{3J}{2}\sum_{q>k_F} \fr{1}{\epsilon_q -E}
\eeq
Converting to integral,
\beq
1 = \fr{3J}{2}\int_{\epsilon_F}^D d\epsilon \fr{N(\epsilon)}{\epsilon -E}
\eeq
\il{D} is the upper limit of the conduction band. Assuming \il{N(\epsilon)} is constant \il{(N(0))} in this range, we get
\beq
\fr{2}{3JN(0)} &= \ln \bigg \vert \fr{D-E}{\epsilon_F - E} \bigg\vert \approx \ln \bigg \vert \fr{D}{\epsilon_F - E}\bigg \vert\\
\implies E &= \epsilon_F - D e^{-\fr{2}{3N(0)J}}
\eeq
Thus, the energy of the ground state is lowered from the Fermi energy by an amount
\beq
E_b = D e^{-\fr{2}{3N(0)J}}
\eeq
The temperature below which this will be stable, \il{T_K}, is given by the relation
\beq[tk]
k_B T_k \sim E_b \implies T_K = \fr{D}{k_B}e^{-\fr{2}{3N(0)J}}
\eeq

\subsection{Poor man's scaling}
The idea is to reduce the bandwidth from \il{D} to \il{D - \delta D}, by considering all possible excitations in that range, up to second order. The transition matrix second order contributions in that range
\beq
T^{(2)} = VG_0V
\eeq
can be clubbed into a term \il{\Delta V}. This term is a representative of the scatterings from that range. After reducing the bandwidth to \il{D -\delta D}, the effect of the excluded region can be incorporated by changing the interaction term \il{V \ra V^\prime = V + \Delta V}. The interaction part is
\beq
H^\prime = J_z \sum_{k_1,k_2}S_d^z\rr{c^\dagger_{k_1\ua}c_{k_2\ua} - c^\dagger_{k_1\da}c_{k_2\da}} + J_T \sum_{k_1,k_2}\rr{S_d^+ c^\dagger_{k_1\da}c_{k_2\ua} + S_d^- c^\dagger_{k_1\ua}c_{k_2\da}}
\eeq
Incorporating \il{\Delta V} will involve changing the coupling constants \il{J_z} and \il{J_T}. There are three types of scattering processes at second order:
\begin{enumerate}
	\item No spin-flip of impurity - involving \il{\rr{S_d^z}^2}
	\item one spin-flip of impurity - involving \il{S_d^z S_d^\pm} or \il{S_d^\pm S_d^z}
	\item two spin-flips of impurity - involving \il{S_d^\pm S_d^\mp}
\end{enumerate}

The first kind does not involve any spin impurity operator (\il{S_z^2 = \fr{1}{4}}), so it will be ignored. The second kind will leave the impurity spin flipped at the end, and will hence result in a renormalization of \il{J_T}. The third kind will leave the impurity spin unchanged (two flips), and hence will involve a renormalization of \il{J_z}.

\subsubsection*{Renormalization of \il{J_z}}
First consider the process
\beq
k \ua, d\da \ra q \da d \ua \ra k^\prime \ua d\da
\eeq
The \il{T-}matrix term is
\beq
T_1 = J_T^2\sum_q S_d^- c^\dagger_{k^\prime \ua}c_{q\da}\fr{1}{E - H_0}S_d^+ c^\dagger_{q \da}c_{k\ua}
\eeq
Using eq.~\ref{identity}, we can write
\beq
(E - H_0)^{-1} c^\dagger_{q \da}c_{k\ua} = c^\dagger_{q \da}c_{k\ua}(E - \lambda -H_0)^{-1}
\eeq
where \il{\lambda} is given by \il{\qq{H_0,c^\dagger_{q \da}c_{k\ua}} = (\epsilon_q - \epsilon_k) c^\dagger_{q \da}c_{k\ua} \implies \lambda = \epsilon_q - \epsilon_k}. Hence,
\beq
T_1 = J_T^2 S_d^- S_d^+ \sum_q c^\dagger_{k^\prime \ua}c_{q\da}c^\dagger_{q \da}c_{k\ua}\rr{E - \epsilon_q + \epsilon_k - H_0}^{-1}
\eeq
Since the upper momenta states are unoccupied, \il{c_{q\da}c^\dagger_{q \da} = 1 -n_q = 1}.
\beq
T_1 = J_T^2 S_d^- S_d^+ c^\dagger_{k^\prime \ua}c_{k\ua}\sum_q \rr{E - \epsilon_q + \epsilon_k - H_0}^{-1}
\eeq
If we set the Fermi level to 0, \il{H_0 = 0}. Since the summation is over the narrow band \il{\{D - \delta D, D\}}, we can approximate the result of the summation as 
\beq
\sum_q \rr{E - \epsilon_q + \epsilon_k - H_0}^{-1} = N |\delta D | \fr{1}{E - D + \epsilon_k}
\eeq
\il{N} is the density of states. Also,
\beq
S^- S^+= \rr{S^x - iS^y}\rr{S^x + iS^y} = \fr{1}{2} + i\qq{S^x,S^y} = \fr{1}{2} - S^z
\eeq 
Putting it all together, 
\beq
T_1 = J_T^2\rr{\fr{1}{2} - S_d^z}N|\delta D|c^\dagger_{k^\prime \ua}c_{k\ua}\fr{1}{E - D + \epsilon_k}
\eeq
For the second possible scattering,
\beq
q \da k \ua d\ua \ra k^\prime \ua k \ua d\da \ra k^\prime\ua q \da d\ua
\eeq
we get
\beq
T_2 = J_T^2\sum_q S_d^+S_d^- c^\dagger_{q\da}c_{k\ua}\fr{1}{E - H_0}c^\dagger_{k^\prime \ua}c_{q\da}
\eeq
Using \il{\qq{H_0, c^\dagger_{k^\prime \ua}c_{q\da}} = \rr{\epsilon_{k^\prime} - \epsilon_q} c^\dagger_{k^\prime \ua}c_{q\da}= \rr{\epsilon_{k^\prime} +D} c^\dagger_{k^\prime \ua}c_{q\da}}, and \il{S_d^+ S_d^- = \fr{1}{2}+S_d^z}, we get
\beq
T_2 &= J_T^2 \rr{\fr{1}{2}+S_d^z} N |\delta D|c_{k\ua}c^\dagger_{k^\prime \ua} \fr{1}{E - D - \epsilon_{k^\prime}} \\
	   &=-J_T^2 \rr{\fr{1}{2}+S_d^z} N |\delta D|c^\dagger_{k^\prime \ua} c_{k\ua}\fr{1}{E - D - \epsilon_{k^\prime}}
\eeq
The constant term resulting from the commutator at the last line was dropped. For each of these two processes, there are identical processes that start with the conduction electron in \il{\da}:
\begin{gather}
k \da, d\ua \ra q \ua d \da \ra k^\prime \da d\ua\\
q \ua k \da d\da \ra k^\prime \da k \da d\ua \ra k^\prime\da q \ua d\da
\end{gather}
The only difference from the previous processes is that \il{S^+} is replaced by \il{S^-} and vice versa. Hence, these  processes give
\begin{gather}
T_3 = J_T^2\rr{\fr{1}{2} + S_d^z}N|\delta D|c^\dagger_{k^\prime \da}c_{k\da}\fr{1}{E - D + \epsilon_k}\\
T_4 = -J_T^2 \rr{\fr{1}{2}-S_d^z} N |\delta D|c^\dagger_{k^\prime c_{k\da}\da} \fr{1}{E - D - \epsilon_{k^\prime}}
\end{gather}
The total second order contribution is
\beq
T^{(2)} &= -J_T^2 S_d^z N|\delta D|\rr{\fr{1}{E - D + \epsilon_k} + \fr{1}{E - D - \epsilon_{k^\prime}}}\rr{c^\dagger_{k^\prime \ua}c_{k\ua} - c^\dagger_{k^\prime \da}c_{k\da}}\\
\eeq
Comparing this with the \il{S_d^z} term in the Hamiltonian
\beq
J_z S_d^z\rr{c^\dagger_{k^\prime \ua}c_{k\ua} - c^\dagger_{k^\prime \da}c_{k\da}}\\
\eeq
we can easily write down the change in the coupling \il{J_d^z},
\beq
\delta J_d^z = -J_T^2 N|\delta D|\rr{\fr{1}{E - D + \epsilon_k} + \fr{1}{E - D - \epsilon_{k^\prime}}}
\eeq
For low energy excitations, we can neglect \il{E, \epsilon_k, \epsilon_{k^\prime}} with respect to \il{D}. Noting that the bandwidth is decreasing and hence \il{\delta D < 0},
\beq
\dv{J_d^z}{D}=-J_T^2N \fr{2}{D}
\eeq
This is the scaling equation for the coupling \il{J_d^z}.

\subsubsection*{Renormalization of \il{J_T}}
Consider the scattering
\beq
k \ua d\da \ra q\da d\ua \ra k^\prime \da d\ua
\eeq
\beq
T_1 = -J_T J_z S_d^z S_d^+N|\delta D|c^\dagger_{k^\prime \da}c_{k\ua}\fr{1}{E - D + \epsilon_k}
\eeq
The minus sign at the front comes from the term
\beq
-S_d^z c^\dagger_{k^\prime \da}c_{q\da}
\eeq
in the Hamiltonian. Using \il{S_d^z S_d^+ = \fr{S_d^+}{2}},
\beq
T_1 = -J_T J_z \fr{S_d^+}{2} N|\delta D|c^\dagger_{k^\prime \da}c_{k\ua}\fr{1}{E - D + \epsilon_k}
\eeq
The second process is
\beq
q \ua k\ua d\da \ra k^\prime \da k \ua d\ua \ra q \ua k^\prime \da d\ua
\eeq
\beq
T_2 = -J_T J_z  \fr{S_d^+}{2}N |\delta D|c^\dagger_{k^\prime \da} c_{k\ua}\fr{1}{E - D - \epsilon_{k^\prime}}
\eeq
Two more processes can be constructed from the above two processes, by switching the \il{S_d^+} and \il{S_d^z} operations. The change in the first process is that the \il{S_d^z} term will now become
\beq
+S_d^z c^\dagger_{k^\prime \ua}c_{q\ua}
\eeq
so that will invert the sign.
The change in the second process is that now the \il{q}-electron has to start off as \il{\da}, which means that the \il{S_d^z} term for this process becomes
\beq
-S_d^z c^\dagger_{k^\prime \da}c_{q\da}
\eeq
So the sign of the second process will also invert. The change common to both the process is that \il{S_d^z S_d^+} becomes \il{ S_d^+S_d^z}. Since  \il{ S_d^+S_d^z = -\fr{S_d^+}{2}}, this will involve a second change in sign for both processes. Thus, overall there is no change for either proces.
\begin{gather}
T_3 = T_1\\
T_4 = T_2
\end{gather}
The total contribution is
\beq
T^{(2)} = -J_T J_z S_d^+ N |\delta D|c^\dagger_{k^\prime \da} c_{k\ua}\rr{\fr{1}{E - D - \epsilon_{k^\prime}} + \fr{1}{E - D + \epsilon_k}}
\eeq
Comparing with the \il{S_d^+} term in the Hamiltonian
\beq
J_T S_d^+ c^\dagger_{k^\prime \da} c_{k\ua}
\eeq
we can write
\beq
\delta J_T = -J_T J_z N |\delta D|\rr{\fr{1}{E - D - \epsilon_{k^\prime}} + \fr{1}{E - D + \epsilon_k}}
\eeq
Again neglecting the terms in the denominator, we get
\beq
\dv{J_T}{D} = -J_T J_z N\fr{2}{D}
\eeq
This is the scaling equation for \il{J_T}.

\subsubsection*{Flow of the couplings}
Switching to the dimensionless couplings
\beq
g_1 = N J_z, g_2 = N J_T
\eeq
the equations become
\begin{gather}\label{ceq}
\dv{g_1}{D} = -\fr{2g_2^2}{D}\\
\dv{g_2}{D} = -\fr{2g_1g_2}{D}
\end{gather}
The first equation says that as the cutoff decreases, \il{g_1} will always increase. For \il{g<0} (ferromagnetic coupling), the coupling will go to zero. That is, at sufficiently low temperatures, the impurity electron becomes effectively decoupled from the conduction band. The phenomenon is called asymptotic freedom. For the antiferromagnetic case, the coupling should go to infinity. This means that at sufficiently low temperatures, the coupling will necessarily become appreciable large so as to render perturbation theory inapplicable.
Dividing the two coupling equations gives
\beq
\dv{g_1}{g_2} = \fr{g_2}{g_1}\implies g_1^2 - g_2^2 = \text{constant}
\eeq
Taking \il{g_1} as the x-axis and \il{g_2} as the y-axis, depending on the sign of the constant, the solution is a vertical hyperbola or horizontal hyperbola. Since the coupling equations are unchanged  under the transformation \il{g_2 \ra -g_2}, analyzing the upper half (\il{g_2 > 0}) suffices. The antiferromagnetic case is easy. \il{g_1 > 0} means \il{g_1} will always increase the RG flow. The only solution is that both \il{g_1} and \il{g_2} flow to infinity. For the ferromagnetic case, if \il{|g_1|>g_2}, \il{g_1} will increase and the representative point will reach the x-axis (\il{g_2 = 0}). At this point, both the couplings will stop changing because both the derivatives involve \il{g_2}. So the fixed point in this case is \il{g_2 = 0} and \il{g_1} is some negative value. However, if \il{|g_1|<g_2}, the representative point will reach the positive y-axis. Since \il{g_2 \neq 0} here, \il{g_1} will continue to grow and become positive at some point. From there, it becomes the antiferromagnetic case.\\\\
Setting \il{g_1 = g_2 =g >0} and integrating either of the scaling equations gives
\beq[isot]
g(D^\prime) &= \fr{g_0}{1-2g_0\ln\fr{D}{D^\prime}} \\
\implies 2g(D^\prime) &= \fr{1}{\ln \fr{D^\prime}{T_K}}
\eeq
where \il{T_K = \fr{D}{k_B} \ex{-\fr{1}{2g_0}}}. \il{D^\prime} is the running bandwidth and \il{D} is the original bandwidth. This is almost the same as the one obtained in eq.~\ref{tk}, because \il{g = N J}. The expression for \il{g_{D^\prime}} shows that perturbation theory will work only for \il{T \gg T_K}, because close to \il{T_K}, the expression becomes non-analytic.\\\\
The ferromagnetic case \il{(g<0)}, on the other hand, remains perturbative.
\beq
g(D^\prime) = \fr{g_0}{1-2g_0\ln\fr{D}{D^\prime}} = -\fr{|g_0|}{1+2|g_0|\ln\fr{D}{D^\prime}}
\eeq
At all points, the expression remains analytic, and gradually goes to zero at \il{D^\prime = 0}.

\subsubsection*{Alternate way of obtaining the scaling equations}
From eq.~\ref{hamtmat}, the interaction part can be written as
\beq
\Delta H_{ll^\prime} = \fr{1}{2}\qq{T_{ll^\prime}(E_l) + T_{ll^\prime}(E_{l^\prime})}
\eeq
where the transition matrix \il{T} is
\beq
T_{ll^\prime}(E) = \sum_H \fr{V_{lH}V_{Hl^\prime}}{E - E_H}
\eeq
Here, \il{\{H\} = \{D-\delta D, D\}} and 
\beq
V = J \vec S_d \cdot \sum_{k,k^\prime,\alpha,\alpha^\prime} c^\dagger_{k \alpha} \vec \sigma_{\alpha \alpha^\prime} c_{k^\prime \alpha^\prime}
\eeq
The first process is 
\begin{gather}
	k \alpha \xrightarrow{\quad \sigma^b \quad} q \lambda \xrightarrow{\quad \sigma^a \quad} k^\prime \beta \\
	d \sigma \xrightarrow{\quad S_d^b \quad} d\sigma^{\prime\prime} \xrightarrow{\quad S_d^a \quad} d\sigma^\prime
\end{gather}
The transition matrix element is
\beq
T_1 &= \sum_{q\in\{D-\delta D\},\lambda,\sigma^{\prime\prime}}\bra{k^\prime\beta,\sigma^\prime}V\ket{q \lambda,\sigma^{\prime\prime}}\bra{q \lambda,\sigma^{\prime\prime}}V\ket{k\alpha,\sigma}\fr{1}{E-E_q}\\
    &= J^2\sum_{\sigma^{\prime\prime}}\rr{S_d^a}_{\sigma^\prime \sigma^{\prime\prime}}\rr{S_d^b}_{ \sigma^{\prime\prime}\sigma}\sum_\lambda \rr{\sigma^a}_{\beta \lambda} \rr{\sigma^b}_{\lambda\alpha}\sum_{q\in\{D-\delta D\}}\fr{1}{E-E_q}\\
    &\approx J^2\rr{ S_d^a S_d^b}_{ \sigma^{\prime}\sigma}\rr{\sigma^a \sigma^b}_{\beta \alpha} \fr{N |\delta D|}{E -D}
\eeq
The second process is
\begin{gather}
	k \alpha \xrightarrow{\quad \quad} k \alpha \xrightarrow{\quad \sigma^a \quad} q \lambda \\
q \lambda \xrightarrow{\quad \sigma^b \quad} k^\prime \beta \xrightarrow{\quad \quad} k^\prime \beta \\
	d \sigma \xrightarrow{\quad S_d^b \quad} d\sigma^{\prime\prime} \xrightarrow{\quad S_d^a \quad} d\sigma^\prime
\end{gather}
Here the intermediate state consists of two electrons with energy \il{E_k, E_{k^\prime}} and a hole with energy \il{-E_q}. The transition matrix element is
\beq
T_2 &= \sum_{q\in\{D-|\delta D|\},\lambda,\sigma^{\prime\prime}}\bra{q \lambda,k^\prime\beta,\sigma^\prime}V\ket{k^\prime \beta,k \alpha,\sigma^{\prime\prime}}\bra{k^\prime \beta,k \alpha,\sigma^{\prime\prime}}V\ket{q\lambda,k\alpha,\sigma}\fr{1}{E-\rr{E_k + E_{k^\prime} - E_q}}\\
    &\approx -J^2\rr{ S_d^a S_d^b}_{ \sigma^{\prime}\sigma}\rr{\sigma^b\sigma^a }_{\beta \alpha} \fr{N |\delta D|}{E -D}
\eeq
Neglecting \il{E} with respect to \il{D} and adding the contributions, we get
\beq
T &= \fr{J^2 N |\delta D|}{D} \rr{S_d^a S_d^b}_{\sigma^{\prime}\sigma}\qq{\sigma^b,\sigma^a}_{\beta \alpha}\\
  &=\fr{J^2 N |\delta D|}{2D} \qq{S_d^a,S_d^b}_{\sigma^{\prime}\sigma}\qq{\sigma^b,\sigma^a}_{\beta \alpha}
\eeq
In the last step, I used \il{\cc{S^a,S^b}=0}. Now,
\beq
\qq{S_d^a,S_d^b}_{\sigma^{\prime}\sigma}\qq{\sigma^b,\sigma^a}_{\beta \alpha} &= -\qq{S_d^a,S_d^b}_{\sigma^{\prime}\sigma}\qq{\sigma^a,\sigma^b}_{\beta \alpha}\\
&= -i\epsilon_{abc}S^c_{\sigma \sigma^\prime} 2 i \epsilon_{abd}\sigma^d_{\beta \alpha}\\
&=4\delta_{cd}S^c_{\sigma \sigma^\prime} \sigma^d_{\beta \alpha}\\
&=4\vec S_{\sigma \sigma^\prime} \cdot \vec \sigma_{\beta \alpha}
\eeq
Therefore,
\beq
T = \fr{2J^2 N |\delta D|}{D}\vec S_{\sigma \sigma^\prime} \cdot \vec \sigma_{\beta \alpha}
\eeq
The correction to the coupling \il{J} can be read off:
\beq
J(D - \delta D) = J(D) - \fr{2J^2 N \delta D}{D}
\eeq
This gives the same scaling equations we found earlier.

\subsection{Universality}
Adding a higher order correction to the Poor Man's scaling gives
\beq
\pd{g}{\ln D} = -2g^2 + 2g^3
\eeq
It can be integrated from \il{g^0(D)} to \il{g(D^\prime)}:
\beq
\ln \fr{D^\prime}{D} = -\int_{g_0}^g\fr{dg}{2g^2 - 2g^3} = -\int_{g_0}^g\fr{dg}{2g^2}\rr{1+g} \\
\eeq
Defining \il{D^\prime = k_B T_K} to be the temperature where \il{g \sim 1}, we can write
\beq
\ln \fr{k_B T_K}{D} = -\int_{g_0}^1\fr{dg}{2g^2}\rr{1+g} = -\fr{1}{2g_0} + \fr{1}{2}\ln g_0 + O(1)\\
=-\fr{1}{2g_0} + \fr{1}{2}\ln 2g_0 + O(1)
\eeq
This gives a better estimate of the Kondo temperature
\beq[sol]
T_K = \fr{D}{k_B} \sqrt{2g_0}\ex{-\fr{1}{2g_0}}
\eeq
\il{T_K} can also be determined by appealing to dimensional arguments and ideas of universality. Since the energy scale in question is \il{D}, we can write
\beq[form]
k_B T_K = D y(g)
\eeq
where \il{y} is some dimensionless quantity. Since \il{T_K} is a physical quantity, it cannot change with our choice of the bandwidth \il{D}:
\beq
\dv{T_K}{D}=0
\eeq
Substituting the form of \il{T_K}, eq.~\ref{form}, in this equation gives
\beq
y(g) + D\dv{y(g)}{D} = 0\\
\implies y + D\dv{y}{g}\dv{g}{D} = 0\\
\implies y -2g^2\dv{y}{g} = 0\\
\implies y = e^{-\fr{1}{2g}}
\eeq
This gives almost the same solution as eq.~\ref{sol}:
\beq
T_K = \fr{D}{k_B}e^{-\fr{1}{2g}}
\eeq
The difference in the pre-factor arises from the extra contribution incorporated in that solution.\\\\
The fact that the scaling equations are universal can be seen by noting that from eq.~\ref{isot}, up to second order, we can write
\beq
g(D^\prime) = g_0\rr{1 + 2g_0^2\ln\fr{D}{D^\prime}}
\eeq
As we lower the temperature, the quantum processes are able to be coherent and lower energies.  At temperature \il{T}, the order of energies that is explored by the processes is \il{k_B T}. Hence we can set \il{\fr{D}{D^\prime} =\fr{T}{T_F}}. This says that the variation of the coupling from \il{g_0} to \il{g} is
\beq
g_0 \ra g = g_0\rr{1 + 2g_0 \ln \fr{T_F}{T}}
\eeq
Since \il{g \equiv N J}, we have recovered eq.~\ref{change}. Since eq.~\ref{change} was obtained as a perturbation calculation, it should have been valid only at \il{T \gg T_K}, but the scaling relation holds at all temperatures.

\subsection{Method of pseudo-fermions}
Spin operators, unlike fermionic creation and annihilation operators,  do not satisfy Wick's theorem. To remedy this, they can be factorised into fermionic operators. For example,
\beq
S^z = \fr{\sigma^z}{2} = \sum_{ij} c^\dagger_i \fr{\sigma^z_{ij}}{2} c_j = \fr{1}{2}\rr{c^\dagger_\ua c_\ua - c^\dagger_\da c_\da}
\eeq
Similarly,
\beq
S^x = \fr{1}{2}\rr{c^\dagger_\ua c_\da + c^\dagger_\da c_\ua}\\
S^y = \fr{-i}{2}\rr{c^\dagger_\ua c_\da - c^\dagger_\da c_\ua}
\eeq
Now, the state \il{\ket{\ua}} can be represented as
\beq
\ket{\ua} = c^\dagger_\ua \ket{0}
\eeq
This however means that we get two other states in the Hilbert space, \il{\ket{0}} and \il{\ket{\ua\da}}, which are not allowed physically. To remove them, we can do the following. We can modify the Hamiltonian \il{H}, by introducing a complex chemical potential
\beq
\mu = -i\fr{\pi}{2}k_B T
\eeq
The new Hamiltonian is
\beq
\wl H = H -\mu (n_d - 1)
\eeq
The new partition function is then allowed to run over the entire Hilbert space, including the unphysical states. The actual partition function for the original Hamiltonian \il{H} is
\beq
Z = \text{Tr}\qq{\ex{-\beta H}} = \sum_{\sigma_d = \ua,\da}\sum_{k}\qq{\ex{-\beta H}}
\eeq
The modified partition function is
\beq
\wl Z &= \text{Tr}\qq{\ex{-\beta \rr{H -  \mu(n_d-1)}}}\\
	     &=\text{Tr}\qq{\ex{-\beta H - i\fr{\pi}{2}(n_d -1)}}\\
	     &=\sum_{\sigma_d = \ua,\da}\sum_{k}\qq{\ex{-\beta H}} + \sum_{k}\ex{-\beta H + i\fr{\pi}{2}} + \sum_{k}\ex{-\beta H - i\fr{\pi}{2}}\\
	     &= Z\bigg\vert_{n_d = 1} + i Z\bigg\vert_{n_d = 0} - i Z\bigg\vert_{n_d = 0}
\eeq
Since the Hamiltonian involves the impurity electrons only as spin operators, and since \il{S_d(0) = 0 = S_d(\ua\da)}, we have 
\beq
Z\bigg\vert_{n_d = 0} = Z\bigg\vert_{n_d = 0}
\eeq
Hence,
\beq
\wl Z = Z
\eeq
Thus, we are able to retain the correct partition function because of the introduction of the complex chemical potential.

\subsection{Nozières' local Fermi liquid theory}
Wilson's numerical renormalization group calculation showed that the low temperature specific heat contribution from the singlet is linear in temperature
\beq
C_V = \gamma T
\eeq
This suggests that the strong-coupling limit of the Kondo model is a Fermi liquid.\\
The singlet state (\il{s=0}) has an energy
\beq
E_g = J\qq{2\vec S_e \cdot \vec S_d} = J\qq{S^2 - S_d^2 - S_e^2} = J\qq{s\rr{s+1} - \fr{3}{2}} = -\fr{3J}{2}
\eeq
Since the interaction term is spherically symmetric, it suffices to consider a one dimensional chain of conduction electrons with the impurity site coupling to the conduction electron at the origin. This electron forms a singlet with the impurity electron,
\beq
\fr{\ket{0_\ua,d\da} - \ket{0_\da,d_\ua}}{\sqrt 2}
\eeq
Considering a tight-binding model, the only electron that can hop to the zeroth site is the one on the first site. The hopping of this electron on to the zeroth site would lead to an energy of
\beq
E_1 = -\fr{3}{2}J + \fr{3}{2}J = 0
\eeq
because the new electron would have the spin opposite to the other electron on the \il{0^\text{th}} site. This means that breaking the singlet raises the energy by \il{\fr{3}{2}J}. At low temperatures and very large \il{J}, this is not possible. That being said, there can always be virtual fluctuations into excited states. For example, the impurity electron can tunnel into the conduction band (\il{n_d = 0}) or another conduction electron may scatter into the impurity site (\il{n_d = 2}). Both these states have zero energy. With further virtual excitations, it is also possible to go into the triplet state with energy \il{\fr{J}{2}}. What this means is that although the singlet is stable with respect to energy-conserving transitions, the singlet is virtually polarizable, with the help of the site 1 electron. This induces an interaction on the site 1. Since the interaction on the site 1 is just a manifestation of the polarizability of the singlet, we can either take the singlet with its polarizability and assume the conduction band to be non-interacting, or we can assume the singlet to be static and take the Fermi sea to have a localised interaction at the site 1. In the latter picture, we have a frozen singlet (which can be ignored) and an interacting Fermi sea.\\\\
The goal is to calculate the change in phase shift suffered by the conduction electrons in the presence of interactions. In the absence of interactions, the scattered wavefunction is
\beq[shift]
\psi \sim \fr{\sin\qq{kr + \delta(E_k)}}{r}
\eeq
That is, the phase shift is only a function of the energy. At the Fermi surface, this value \il{\delta(0)} is \il{\fr{\pi}{2}}, as known from the Friedel sum rule.
\beq
n = \sum_\sigma \fr{\delta}{\pi} \implies 1 = \fr{2\delta}{\pi} \implies \delta = \fr{\pi}{2}
\eeq
\il{n} is the number of conduction electrons bound in the resonance and the sum is over the possible quantum numbers (spin in this case). \il{\delta(0)} can also be obtained directly from eq.~\ref{shift}, by substituting \il{k=k_F} and noting that the isolation of the 0\uu{th} site means all wavefunctions should shift by \il{\Delta r = a}:
\beq
k_F a = \delta(0) \implies \delta(0) = \fr{\pi}{2a} 2 = \fr{\pi}{2}
\eeq
where the formula for \il{k_F} was used.\\\\
In a Fermi gas, the energy levels are separated by
\beq
\Delta \epsilon = \pd{\epsilon}{k}\Delta k
\eeq
With the condition that the wavefunction should vanish at the boundary, we have \il{\Delta k = k_n - k_{n-1} = \fr{\pi}{L}}. Hence,
\beq
\Delta \epsilon = \pd{\epsilon}{k}\fr{\pi}{L}
\eeq
However, this changes in the presence of the impurity. Because of eq.~\ref{shift}, the boundary condition becomes
\beq
k_nL + \delta(\epsilon_k) = n\pi \implies k_n = \fr{n\pi}{L} - \fr{\delta}{L} = k_n^0 -\fr{\delta(\epsilon_k)}{L}
\eeq
The energy becomes
\beq
\epsilon(k) &= \epsilon(k^0) + \pd{\epsilon}{k}\rr{k - k_0}\\
	    &= \epsilon_k - \pd{\epsilon}{k}\fr{\delta(\epsilon_k)}{L}
\eeq
In the Landau formulation of an interacting Fermi liquid, the phase shifts will depend on the quasiparticle occupation probabilities \il{n_{k\sigma}}. Hence,
\beq
\wl \epsilon_\sigma(k) = \epsilon_k - \pd{\epsilon}{k}\fr{\delta_\sigma(\epsilon_k,\{n_{q,\sigma}\})}{L}
\eeq
In bulk Fermi liquid, we expand the quasiparticle energy in the deviation of the quasiparticle distribution \il{n_k} from the ideal Fermi-Dirac distribution \il{n^0_k},
\beq[fliq]
\wl \epsilon_p = \underbrace{\epsilon_F}_{\text{Fermi gas}} &+ \overbrace{\fr{p_F^*}{m}\rr{p-p_F}}^{\text{linear contribution for \it{p} close to }p_F} \\
	    &+ \underbrace{\sum_{q\sigma}f(p,q)\rr{n_q - n^0_q}}_{\text{interacting between two quasiparticles at momenta \it{p} and \it{q}}}
\eeq
Similarly, for this local Fermi liquid, the phase shift depends on the energy of the quasiparticle \il{\wl \epsilon} and the quasiparticle occupation \il{n_{q\sigma}}. Accordingly,
\beq
\delta_\sigma(\wl \epsilon,\{n_{q,\sigma}\}) = \delta_\sigma(\wl \epsilon = \epsilon_F, n_k = n^0_k) + \alpha\rr{\wl \epsilon - \epsilon_F} + \Phi \sum_{q\sigma^\prime}\rr{n_{q\sigma^\prime} - n^0_{q\sigma^\prime}}
\eeq
This is just a Taylor expansion of \il{\delta_\sigma} around \il{\tilde \epsilon = \epsilon_F} and \il{n_q = n^0_q}. \il{\Phi} and \il{\alpha} play the same role as \il{f} and \il{\fr{p_F^*}{m}} in eq.~\ref{fliq}. Specifically, \il{\Phi} represents the onsite interaction between quasiparticles of  opposite spin and 
\beq
\alpha = \dv{\delta_\sigma}{E}
\eeq
Since \il{\Phi} acts only between quasiparticles of opposite spin, the last term can be simplified by requiring \il{\sigma^\prime = -\sigma},

\beq[phases]
\delta_\sigma(\wl \epsilon,\{n_{q,\sigma}\}) = \delta_\sigma(\wl \epsilon = \epsilon_F, n_k = n^0_k) + \alpha\rr{\wl \epsilon - \epsilon_F} + \Phi \sum_{q}\delta n_{q,-\sigma}
\eeq
Since the singlet is isolated from the Fermi liquid, any change in the chemical potential will not affect the average occupation of the impurity site \il{\avg{n_d}}, and since we know that \il{\avg{n_d} = \fr{2\delta(0)}{\pi}}, this means that \il{\delta(0)}, the phase shift at the Fermi surface, is invariant under a change of the chemical potential. This in turn means that the resonance scattering (\il{\delta = \fr{\pi}{2}}) will always be pinned to the Fermi surface. With this knowledge, let us explicitly try to calculate the change in the phase shift at Fermi surface when we change the chemical potential by \il{\Delta \mu}. Before the change in chemical potential,
\beq
\delta^0_\ua = \fr{\pi}{2} + \Phi\sum_q \delta n^0_{q\da}
\eeq
Since \il{\delta n^0 = n^0 - n^0 =0},
\beq
\delta^0_\ua = \fr{\pi}{2}
\eeq
After the change in chemical potential, \il{\epsilon_F^\prime = \epsilon_F + \Delta \mu} and 
\begin{gather}
N(\mu = 0) = N^0 \\
N(E^\prime = E+\mu) = N(E^\prime = E) + \dv{N}{E^\prime}\rr{E^\prime - E} = N^0 + \rho \Delta \mu\\
\implies \sum_q \delta n_q = N - N^0 = \rho \Delta \mu
\end{gather}
Hence, from eq.~\ref{phases},
\beq
\delta_\ua &= \fr{\pi}{2} + \alpha\rr{\epsilon_F^\prime - \epsilon_F} + \Phi\sum_q \delta n_{q\da}\\
	   &= \delta^0_\ua + \alpha\Delta\mu + \Phi \rho \Delta \mu
\eeq
Hence the change in the phase is
\beq
0 = \Delta \delta_\ua = \Delta \mu\rr{\alpha + \Phi \rho} \implies \alpha = -\Phi\rho
\eeq\\\\
This shows that the interaction term \il{\Phi} is responsible for pinning the resonance at the Fermi level; without that term in the formalism, the occupancy of the impurity site will change. This is similar to the fact that the interaction term \il{f(k,k^\prime)} in the bulk Fermi liquid is responsible for making the Landau theory invariant under Galilean transformations.\\\\
Now we can calculate the density of states. From the boundary condition, we have
\beq
n_\sigma = \fr{kL}{\pi} + \fr{\delta_\sigma(E)}{\pi} = n^0 + \fr{\delta_\sigma(E)}{\pi}
\eeq
Hence,
\beq
\rho &= \dv{n_\sigma}{E} = \rho^0 + \fr{1}{\pi}\dv{\delta_\sigma}{E} \\
\implies \rho&=\rho^0 + \fr{1}{\pi}\alpha
\eeq
\il{\rho^0} is the density of states in absence of the impurity. The low temperature specific heat of an ideal Fermi liquid can be shown to be
\beq
C_v^0 = \gamma T = \fr{\pi^2 k_B^2}{3} \mathcal{N}(0) T
\eeq
The interacting Fermi liquid is just a renormalised version of the Fermi gas, with a modified density of states \il{\fr{1}{\pi}\alpha}. Hence, the impurity contribution to the specific heat is
\beq
C_v &= \fr{\pi^2 k_B^2}{3}\rr{\rho_\ua + \rho_\da} T\\
    &=\fr{2\alpha}{\pi} \fr{\pi^2 k_B^2}{3} T
\eeq
In presence of a magnetic field \il{B}, the magnetization is 
\beq
m = \delta n \times \mu
\eeq
where \il{\mu} is the magnetic moment
\beq
\mu = -\fr{g}{2}\mu_B 
\eeq
and \il{\delta n} is the difference in number between up and down electrons
\beq
\delta n = \avg{n_\ua} - \avg{n_\da} = \fr{1}{\pi}\rr{\delta_\ua - \delta_\da}
\eeq
In the presence of the magnetic field, all energies get modified,
\beq
E^B_\sigma = E - \sigma \fr{g\mu_B}{2}B
\eeq
Hence,
\beq
\sum_k \delta n_{k\sigma} = N_\sigma(E^B_\sigma) - N(E) = \dv{N}{E^B}\rr{E^B - E} = -\rho \fr{g \mu_B}{2}\sigma B
\eeq
This modifies the phase shift at the Fermi surface,
\beq
\delta_\sigma(\epsilon_F) &= \fr{\pi}{2} + \alpha\rr{\epsilon_F - \fr{g \mu_B}{2}\sigma B - \epsilon_F} + \Phi\sum_q \delta n_{q,-\sigma}\\
			  &= \fr{\pi}{2} - \sigma \fr{g \mu_B}{2}\alpha B + \Phi \rho \fr{g \mu_B}{2}\sigma B\\
			  &= \fr{\pi}{2} - 2\alpha\fr{g \mu_B}{2}\sigma B
\eeq
Hence,
\beq
\delta n = \fr{1}{\pi}\rr{\delta_\ua - \delta_\da} = -\fr{4\alpha B}{\pi}\fr{g \mu_B}{2}
\eeq
The susceptibility is
\beq
\chi = \pd{m}{B} = \pd{}{B}\mu\delta n = \fr{4\alpha}{\pi}\rr{\fr{g \mu_B}{2}}^2
\eeq

The susceptibility for an ideal Fermi gas can be calculated similarly. The additional energy of an electron with spin \il{\sigma} in a magnetic field \il{B} is \il{-\sigma \fr{g}{2}\mu_B B}. The magnetization induced at the Fermi surface is \il{\delta n \times \mu}, where \il{\mu} is the magnetic moment
\beq
\mu = -\fr{g}{2}\mu_B 
\eeq
and \il{\delta n} is the difference in number between up and down electrons
\beq
\delta n = n_\ua(0) - n_\da(0) = n_\ua(\epsilon_F - \fr{g}{2}\mu_B B) - n_\da(\epsilon_F + \fr{g}{2}\mu_B B) = -\fr{1}{2}\mathcal{N}(0)gB \mu_B
\eeq
\il{\mathcal{N}(0) = \pd{n}{E}\bigg |_{\epsilon_F}} is the density of states at the Fermi energy and the \il{\fr{1}{2}} is because we are counting electrons of a particular spin only. Therefore,
\beq
m = \delta n \times \mu = \mathcal{N}(0)\rr{\fr{g}{2} \mu_B}^2B
\eeq
The magnetic susceptibility comes out to be 
\beq
\chi^0 = \pd{m}{B}\bigg |_{B \ra 0} = \mathcal{N}(0)\rr{\fr{g}{2} \mu_B}^2
\eeq
The Wilson ratio \il{R} can now be computed,
\beq
R = \frac{\chi/\chi_0}{C_v/C_v^0} = \frac{4\alpha/\pi\mathcal{N}(0)}{2\alpha/\pi\mathcal{N}(0)} = 2
\eeq

\subsection{Numerical renormalization group calculation}
Wilson's idea wanted to remove the limitations of the perturbative nature of Anderson's scaling method. To that end, we transformed the Hamiltonian into a one-dimensional chain, and then iteratively diagonalised chains of increasing length. The Hamiltonian we are working with is
\beq
H = \sum_k \epsilon_k n_k + J \vec S_d \cdot \vec \sigma_e
\eeq
where \il{\vec \sigma_e = \sum_{k_1,k_2,\alpha\beta}c^\dagger_{k_1\alpha}\vec \sigma_{\alpha\beta}c_{k_2,\beta}} is the conduction electron spin at the origin. This assumes that the exchange interaction \il{J(k,k^\prime} is independent of spin. To form the linear chain, we construct a new basis in which to express the conduction electron part \il{H_c}, out of the states \il{\ket{0},H_c\ket{0},H_c^2\ket{0},...}. \il{\ket{0}} is the origin site, where the impurity resides. The first member of the new basis is \il{\ket{0}}. The next member is taken to be some state in the subspace of \il{\ket{0}} and \il{H_c\ket{0}},
\beq
\ket{1} = \rr{\lambda_1 H_c\ket{0} + \lambda_2\ket{0}}
\eeq
This is a general form for any ket in the subspace spanned by \il{\ket{0}} and \il{H_c\ket{0}}. Since we want the state to be normalised , we can shift one of the parameters to the denominator:
\beq
\ket{1} = \fr{1}{\gamma_0}\rr{H_c\ket{0} + \lambda\ket{0}}
\eeq
where \il{\gamma_0} sets \il{\avg{1|1} = 1}. The remaining parameter is set by requiring \il{\avg{1|0} = 0}. That gives
\beq
\lambda = -\avg{0|H_c|0}
\eeq
Therefore,
\beq
\ket{1} = \fr{1}{\gamma_0}\rr{H_c\ket{0} -\avg{0|H_c|0}\ket{0}}
\eeq
The general state can be shown to be
\beq[chain]
\ket{n+1} = \fr{1}{\gamma_n}\rr{H_c \ket{n} - \ket{n}\avg{n|H_c|n-1}-\ket{n-1}\avg{n-1|H_c|n}}
\eeq
From eq.~\ref{chain}, by multiplying \il{\bra{n^\prime}} from left, we get
\beq
\delta_{n^\prime,n+1} = \fr{1}{\gamma_n}\qq{\rr{H_c}_{n^\prime,n} + \rr{H_c}_{n,n-1}\delta_{n^\prime,n} + \rr{H_c}_{n-1,n}\delta_{n^\prime,n-1}}
\eeq
Clearly, for \il{n^\prime < n-1} or \il{n^\prime > n+1}, we get
\beq
\rr{H_c}_{n^\prime,n} = 0
\eeq
so the only non-zero terms are for \il{n^\prime = n-1,n,n+1}. For \il{n^\prime = n+1} gives
\beq
\rr{H_c}_{n+1,n} = \gamma_n
\eeq
Taking the complex conjugate of this gives
\beq
\gamma_n^* = \rr{H_c^\dagger}_{n,n+1} = \rr{H_c}_{n,n+1}
\eeq
Defining 
\beq
\rr{H_c}_{n,n} = \epsilon_n
\eeq
we can write
\beq
H_c &= \sum_{n_1,n_2} \ket{n_1}\bra{n_1} H_c \ket{n_2}\bra{n_2}\\
    &= \sum_{n}\epsilon_n \ket{n}\bra{n} + \sum_{n}\rr{\gamma_n\ket{n} \bra{n+1}+ \gamma^*_n\ket{n+1} \bra{n}}\\
    &= \sum_{n}\epsilon_n \hat n_n + \sum_{n}\rr{\gamma_nc^\dagger_n c_{n+1}+ \gamma^*_nc^\dagger_{n+1} c_{n}}
\eeq
The diagonalization of these chains become impossible for \il{n>8}. To remedy this problem, Wilson, after diagonalization a chain of a particular length, retained only the lowest parts of the spectrum, and the Hamiltonian for the next stage was formed out of these low-lying states. This keeps the size of the Hilber space (and hence the matrices) manageable. Another problem is that as one goes on adding sites to the chain, the couplings need to die off, otherwise this process will never converge. 
\subsubsection*{Logarithmic discretization}
First, note that up to first order
\beq
\epsilon_k = \epsilon_F + (k-k_F)\pd{\epsilon_k}{k}
\eeq
By choosing \il{k_F = \epsilon_F = 0}, we get \il{\epsilon_k = k}.\\\\
Wilson divided the conduction band into patches, \il{[\Lambda^{-(n+1)},\Lambda^{-n}]}, for \il{n=1,2,3..}. The width of each interval is
\beq
d_n = \Lambda^{-n}\rr{1-\Lambda^{-1}}
\eeq
We can now define orthogonal functions in this \il{n^\text{th}} interval \il{k \in [\Lambda^{-(n+1)},\Lambda^{-n}]},
\beq
\psi_{m,n}(k) = \fr{1}{\sqrt{d_n}} \ex{\fr{2\pi i m}{d_n} k}
\eeq
They allows us to define a new set of creation operators,
\beq
a^\dagger_{m,n} = \sum_k \psi_m(k)c^\dagger_k
\eeq
Similarly functions can be defined in the negative interval \il{-k \in[\Lambda^{-(n+1)},\Lambda^{-n}]}.
\begin{gather}
	\phi_{m,n}(k) = \fr{1}{\sqrt{d_n}} \ex{-\fr{2\pi i m}{d_n} k}\\
	b^\dagger_{m,n} = \sum_k \phi_m(k)c^\dagger_k
\end{gather}
Then,
\beq
a^\dagger_{m,n} + b^\dagger_{m,n} = \fr{2}{\sqrt{d_n}}\sum_{\pm k \in []} \cos \rr{\fr{2\pi mk}{d_n}} c^\dagger_k
\eeq
Summing over \il{n} involves summing over all momenta.
\beq[zerom]
\sum_n \rr{a^\dagger_{m,n} + b^\dagger_{m,n}} &= \fr{2}{\sqrt{d_n}}\sum_{k} \cos \rr{\fr{2\pi mk}{d_n}} c^\dagger_k\\
\implies \sum_n \rr{a^\dagger_{0,n} + b^\dagger_{0,n}} &= \fr{2}{\sqrt{d_n}}\sum_{k} c^\dagger_k\\
\eeq
For the momentum-independent \il{J(k,k^\prime)}, the coupling term involves.
\beq
\sum_{k,q}c^\dagger_k c_q = \sum_{k}c^\dagger_k \sum_{q}c_q
\eeq
Looking at eq.~\ref{zerom}, we see that the impurity spin is coupled only to the \il{m=0} operators.
This is where the approximation comes in, in Wilson's scheme. All the \il{m} values other than \il{m=0} are ignored.\\\\
Wilson chose
\beq
\epsilon_n = 0, \gamma = D^\prime \Lambda^\fr{-n}{2}
\eeq
with \il{\Lambda>1}. The Hamiltonian for \il{N} sites then turns out to be
\beq[run]
H_N = D^\prime \sum_{n=0}^{N-1} \Lambda^{-\fr{n}{2}}\rr{c^\dagger_n c_{n+1}+c^\dagger_{n+1} c_{n}} + 2 J \vec S_d \cdot \vec S_e
\eeq
The next step involves adding another site to the chain. The next Hamiltonian is hence
\beq
H_{N+1} = H_N + D^\prime \Lambda^{-\fr{N}{2}}\rr{c^\dagger_N c_{N+1}+c^\dagger_{N+1} c_{N}}
\eeq
To compare the couplings, and hence the Hamiltonians, at each value of \il{N}, we need to rescale the Hamiltonians \il{H_N} so that the lowest energy scale is independent of the running index \il{N}. Looking at eq.~\ref{run}, the lowest energy scale is \il{\Gamma_N = D^\prime \Lambda^{-\fr{N-1}{2}}}. Hence, the rescaled Hamiltonian is
\beq
\ol H_N = \frac{H_N}{\Gamma_N} = \fr{\Lambda^{\fr{N-1}{2}}}{D^\prime} H_N
\eeq
The utility can be seen by noting the relation between \il{\ol H_{N+1}} and \il{\ol H_N},
\beq[rec]
\ol H_{N+1} &= \fr{\Lambda^{\fr{N}{2}}}{D^\prime}\qq{H_N + \Lambda^{\fr{-N}{2}}D^\prime\rr{c^\dagger_N c_{N+1}+c^\dagger_{N+1} c_{N}}}\\
\implies \ol H_{N+1} &= \Lambda^\fr{1}{2}\ol H_N + \rr{c^\dagger_N c_{N+1}+c^\dagger_{N+1} c_{N}}
\eeq
In the series of Hamiltonians \il{\{H_N\}}, the couplings to the extra site are all same, so the lowest energy scales are all of the same order. This allows us to construct a flow of the Hamiltonians. The real Hamiltonian is the unscaled one, so it is given by
\beq
H = \lim_{N \ra \infty} H_N = \lim_{N \ra \infty} D^\prime \Lambda^\fr{1-N}{2} \ol H_N
\eeq
Since \il{\ol H_N} is exactly diagonalised with a spectrum \il{\{E_m,\ket{m}\}}, it can be written down as
\beq
\ol H_N = \sum_m E_m \ket{m}\bra{m}
\eeq
The next Hamiltonian is then
\beq
\ol H_{N+1} = \Lambda^\fr{1}{2} \sum_m E_m \ket{m}\bra{m} + \sum_{m.m^\prime}\rr{ C(m,m^\prime) \ket{m}\bra{m^\prime} + \text{h.c.}}
\eeq
This is the same equation as eq.~\ref{rec}, with \il{\ol H_N} expressed in its eigenbasis and the creation and annihilation operators also expressed in that basis; the \il{C(m,m^\prime)} are just the matrix elements of \il{c} and \il{c^\dagger} in that basis.\\\\
To check whether the guesses about the fixed points are true, Wilson did the following. He set \il{J=0.009} and then then calculated the lowest excitations of the Hamiltonians obtained from the NRG in the limit of large \il{N}. They indeed correspond to the excitations of the Kondo hamiltonian at \il{J = \infty}, meaning that under the application of the NRG, the \il{J=0.009} Hamiltonian flowed to the fixed-point Hamiltonian \il{J = \infty}.

\subsection{Anderson model \il{\leftrightarrow} Fermi liquid}
The fixed-point Hamiltonians are found to represent interacting Fermi liquids. The effective Hamiltonian can be shown to resemble the Anderson model, but with modified parameters,
\beq
H_\text{eff} = \sum_k \epsilon_k n_k + \sum_k{V_k c^\dagger_d c_k + \text{h.c.}} + U n_{d\ua}n_{d\da}
\eeq
The parameters \il{\epsilon_k,V_k,U} are not the same as the Anderson model we start with, but I am using the same symbols for convenience. We will see that a mean-field treatment of this model goes to the Landau Fermi liquid model at \il{T \ra 0}. Under mean-field,
\beq
n_{d\ua}n_{d\da} \approx n_{d\ua}\avg{n_{d\da}} + n_{d\da}\avg{n_{d\ua}} = \sum_{k,q,\sigma} |\alpha_{q}|^2|\alpha_{k}|^2 n_{k\sigma} \avg{n_{q,-\sigma}}
\eeq
Since the \il{\alpha} are pure phases, there mod-square are unity. The effective one-particle energy becomes
\beq
\wl\epsilon_{k\sigma} = \epsilon_k + U\sum_q \avg{n_{q,-\sigma}}
\eeq
This is analogous to eq.~\ref{temp_en}, \il{U} acting as the interaction between the quasiparticles.  This quantity \il{\gamma = \fr{C_v}{T}} is
\begin{gather}
	\gamma_\text{imp} = \fr{2\pi^2 k_B^2}{3}\rho_d(0)\label{gimp}
\end{gather}
The magnetic susceptibility is
\beq
\chi_m &= -\fr{g\mu_B}{2}\pd{}{B}\qq{n(\wl\epsilon_{F\ua}-h) - n(\wl \epsilon_{F\da}+h)}\\
       &= -\fr{g\mu_B}{2}\sum_k \delta(\wl\epsilon_k - \epsilon_F)\qq{\pd{}{B}\rr{\epsilon_k + Ud_\ua - h} - \pd{}{B}\rr{\epsilon_k + Ud_\da + h}}\\
       &=-\fr{g\mu_B}{2}\sum_k \delta(\wl\epsilon_k - \epsilon_F)\qq{-g\mu_B + U\pd{d_\ua}{B} - U\pd{d_\da}{B}}
\eeq
where \il{h = \fr{g\mu_B B}{2}} and \il{d_\sigma = \sum_l \avg{\delta n_{l,-\sigma}}}. Since \il{\ua} is shifted below the Fermi surface and \il{\da} is shifted above, we have \il{d_\ua = -d_\da}. Therefore,
\beq
\chi_m &=\fr{g^2\mu_B^2}{2}\sum_k \delta(\wl\epsilon_k - \epsilon_F)\qq{1 + U\pd{d_\da}{h}}
\eeq
Since \il{\partial_h d_\da = \sum_k \delta(\epsilon_F - \wl\epsilon_k)\partial_h h} up to lowest order,
\beq
 \chi_m&=\fr{g^2\mu_B^2}{2}\rr{\rho_0(0)+\rho_{d}(0)}\qq{1+U\rho_{d}(0)}
\eeq
The impurity contribution is
\beq[mimp]
\chi_{\text{m,imp}} = \fr{g^2\mu_B^2}{2}\rho_{d}(0)\qq{1+U\rho_{d}(0)}
\eeq
Similarly, the charge susceptibility is
\begin{gather}
	\chi_\text{c,imp} = 2\rho_d(0)\rr{1-U\rho_d(0)}\label{chimp}
\end{gather}
where \il{\rho_d(E)} is the density of states in the non-interacting limit. \il{\chi_m} and \il{\chi_c} are the magnetic and charge susceptibilities. Eliminating \il{\rho_d(0)} between eqs.~\ref{gimp},\\\ref{mimp} and \ref{chimp} gives
\beq[historia]
\fr{4}{g^2\mu_ B^2}\chi_\text{m,imp} + \chi_\text{c,imp} = \fr{6}{\pi^2k_B^2}\gamma_\text{imp}
\eeq
The Wilson ratio is
\beq
R = 1+U\rho_d(0)
\eeq
From the expressions of \il{\chi_m} and \il{\chi_c}, its easy to prove
\beq
U\rho_d(0) = \fr{4\chi_m - g^2\mu_B^2\chi_c^2}{4\chi_m + g^2\mu_B^2\chi_c^2}
\eeq
Substituting that in the Wilson ratio gives
\beq
R = \fr{2}{1 + \fr{g^2\mu_B^2}{4}\fr{\chi_c}{\chi_m}}
\eeq
At the strong-coupling limit, the impurity is localised at the 0\uu{th} site, so the charge susceptibility must go to zero. This gives
\beq
\chi_c= 0
\eeq
Substituting this in the Wilson ratio gives
\beq
R = 2
\eeq

\subsection{Microscopic approach}
To do a perturbative expansion of the Hamiltonian in terms of the interaction \il{U}, it is useful to introduce the self energy \il{\Sigma(E) = \Sigma(0) + E\Sigma^\prime + \Sigma^\text{rem}(E)}. In the absence of interaction, the impurity Green's function is
\beq
G_d^0 = \fr{1}{E - \epsilon_d + i\Delta}
\eeq
Including the self energy gives
\beq[greend1]
G_d = \fr{1}{E - \epsilon_d + i\Delta - \Sigma(E)}
\eeq
Looking at eq.~\ref{greend1}, we can write down the density of states and hence the average occupancy of the impurity site
\begin{gather}
	\rho_{d\sigma} = \fr{-1}{\pi}\text{Im}\qq{G_d}\label{jaeger} \\
	\avg{n_{d\sigma}} = \int_{-\infty}^0 dE\;\rho_{d\sigma} = -\fr{1}{\pi}\text{Im}\int_{-\infty}^0 dE\;\;G_d(E)\label{allforone}
\end{gather}
Luttinger proved that 
\beq
\int_{-\infty}^0 \Sigma^\prime G dE = 0
\eeq
In order to use this, note that
\beq
		     &\ln G_d^{-1}  = \ln\rr{E - \epsilon_d +i\Delta -\Sigma}\\
		     &\implies \dv{\ln G_d^{-1}}{E} = \fr{1}{E - \epsilon_d +i\Delta -\Sigma}\rr{1 - \Sigma^\prime} = G_d - \Sigma^\prime G_d\\
		     &\implies G_d = \Sigma^\prime G_d - \dv{\ln G_d}{E}
\eeq
Substituting this expression for \il{G_d} in eq.~\ref{allforone},
\beq
\avg{n_{d\sigma}} = -\fr{1}{\pi}\text{Im}\int_{-\infty}^0\Sigma^\prime G\;dE +\fr{1}{\pi}\text{Im}\int_{-\infty}^0\dv{\ln G_d}{E}dE
\eeq
The first integral is zero, courtesy Luttinger. We get
\beq
\avg{n_{d\sigma}} &= \fr{1}{\pi}\text{Im}\qq{\ln \fr{1}{E - \epsilon_d +i\Delta -\Sigma}}_{-\infty}^0\\
&=-\fr{1}{\pi}\text{Im}\qq{\ln \rr{E - \epsilon_d +i\Delta -\Sigma}}_{-\infty}^0\\
&=-\fr{1}{\pi}\text{Im}\qq{\ln{e^{i\theta}}}_{-\infty}^0 && \qq{\tan \theta = \fr{\Delta}{E - \epsilon_d -\Sigma}}\\
&= -\fr{1}{\pi}\qq{\theta(0) - \theta(-\infty)}\\
&= \fr{1}{\pi}\tan^{-1}\fr{\Delta}{\epsilon_d+\Sigma(0)}\\
&=\fr{1}{2} - \fr{1}{\pi}\tan^{-1}\fr{\epsilon_d+\Sigma(0)}{\Delta}
\eeq
We can now relate the average occupancy with the density of states. From eq.~\ref{jaeger},
\beq
\rho_{d\sigma} = \fr{1}{\pi}\fr{\Delta}{\rr{\epsilon_d + \Sigma(0)}^2 + \Delta^2} = \fr{1}{\pi\Delta}\qq{1+\rr{\fr{\epsilon_d+\Sigma(0)}{\Delta}}^2}^{-1}\\
=\fr{1}{\pi\Delta}\qq{1+\cot^2\pi\avg{n_{d\sigma}}}^{-1}=\fr{\sin^2 \pi\avg{n_{d\sigma}}}{\pi \Delta}
\eeq
The average occupation for the non-interacting quasiparticles can be similarly written from eq.~\ref{nandemonayi}.
\beq
\avg{\wl n_d} =\fr{1}{2} - \fr{1}{\pi}\tan^{-1}\fr{\wl\epsilon_d}{\wl\Delta} = \fr{1}{2} - \fr{1}{\pi}\tan^{-1}\fr{\epsilon_d - \Sigma(0)}{\Delta} = \avg{n_d} 
\eeq
This shows that the quasiparticles are in one-one correspondence with the actual particles. For a Fermi liquid, the specific heat is given by \il{\wl C_v = \fr{2\pi^2 k_B^2}{3}\wl\rho(0)T}. Applying it to the problem at hand, we get
\beq[eren]
\gamma_\text{imp} = \fr{2\pi k_B^2}{3}\fr{\wl\Delta}{\wl \epsilon_d^2 + \wl\Delta^2}
\eeq
This matches with the value obtained from Nozières treatment. Luttinger also proved that, up to first order,
\begin{gather}
	\chi_\text{m,imp} = \fr{g^2 \mu_B^2}{2}\alpha^\prime \rho_d(0)\\\label{mikasa}
\chi_\text{c,imp} = 2\alpha^{\prime\prime}\rho_d(0)
\end{gather}
where
\begin{gather}
\alpha^\prime = \fr{1-\partial_h\Sigma_\sigma(\epsilon_F,0)}{1-\partial_E\Sigma_\sigma(\epsilon_F,0)}\\
\alpha^{\prime\prime}=\fr{1+\partial_\mu\Sigma_\sigma(\epsilon_F,0)}{1-\partial_E\Sigma_\sigma(\epsilon_F,0)}
\end{gather}

\subsection*{Perturbation expansion approach}
As shown previously in section \ref{adiab}, this can be shown to take the form
\beq[greend2]
G_d = \fr{Z}{E - \wl \epsilon_d + i\wl \Delta - \wl \Sigma(E)}
\eeq
where the \il{\large \textasciitilde} represents the renormalised quantities
\beq[renpar]
 \wl \epsilon_d &= Z(\epsilon_d +\Sigma(0))\\
 \wl \Delta &= Z\Delta\\
 \wl \Sigma &= Z\Sigma^\text{rem}(E)\\
 Z^{-1} &= 1 - \Sigma(0)^\prime\\
 \wl \Gamma_{\sigma\sigma^\prime}(E,E^\prime) &= z^2 \Gamma_{\sigma\sigma^\prime}(E,E^\prime) \\
 \wl U &= z^2 \Gamma_{\ua\da}(0,0)
\eeq
The perturbative expansion is about the bare Hamiltonian, that is, the one with \il{\wl\Sigma = 0}. The corresponding Greens function (non-interacting quasiparticle Green's function) is
\beq[nandemonayi]
G^0_d = \fr{1}{E - \wl\epsilon_d + i\wl\Delta}
\eeq
The Anderson hamiltonian
\beq
H = \epsilon_d n_d + Un_{d\ua}n_{d\da} + \sum_k \epsilon_k n_k + \sum_k \rr{V_k c^\dagger_{d\sigma}c_{k\sigma}+V_k^* c^\dagger_{k\sigma}c_{d\sigma}}
\eeq
can be written in the form
\beq[id]
H = \wl H_{qp} - \wl H_c
\eeq
\il{\wl H_{qp} = \wl H_{qp}^0 + \wl H_{qp}^I} is the total quasiparticle Hamiltonian, consisting of a non-interacting part \il{\wl H_{qp}^0} and an interaction \il{\wl H_{qp}^I}.
\begin{gather}
\wl H_{qp}^0 = \wl \epsilon_d \wl n_d + \sum_k \epsilon_k n_k + \sum_k \rr{\wl V_k \wl c^\dagger_{d\sigma}c_{k\sigma}+\wl V_k^* c^\dagger_{k\sigma}\wl c_{d\sigma}}\\
\wl H_{qp}^I = \wl U \wl n_{d\ua}\wl n_{d\da}
\end{gather}
The renormalised parameters are defined in eq.~\ref{renpar}. The renormalised operators are
\begin{gather}
\wl c^\dagger_d = \sqrt z c^\dagger_d\\
\wl c_d = \sqrt z c_d
\end{gather}
The \il{\wl H_c} that satisfies eq.~\ref{id} is
\beq
\wl H_c = \lambda_1 \wl n_d + \lambda_2 n_{d\ua}n_{d\da}
\eeq
where
\begin{gather}
\lambda_1 = z\Sigma(0,0)\\
\lambda_2 = z^2 \qq{\Gamma_{\ua\da}(0,0) - U}
\end{gather}
Close to the Fermi surface, we want the renormalised self-energy \il{\wl \Sigma(E)} to vary as \il{E^2}. That gives two constraints
\beq
\wl \Sigma(0) &= 0\\
\wl \Sigma^\prime(0) &= 0
\eeq
Close to the Fermi surface, we also have
\begin{gather}
\wl\Gamma_{\ua\da}(0) = \wl U\\
\wl\Gamma_{\sigma\sigma}(0) = 0\\
\implies\Gamma_{\sigma\sigma^\prime}(0) = \wl U(1-\delta_{\sigma\sigma^\prime})
\end{gather}
This is the third constraint. The perturbation expansion is in powers of the renormalised interaction \il{\wl U}. The parameters that are determined by the expansion are \il{\lambda_1,\lambda_2,z}. Hence, they should be expanded in powers of \il{\wl U}.
\begin{gather}
\lambda_i =  \sum_n \lambda_i^{(n)}\wl U^n\\
z = \sum_n z^{(n)}\wl U^n\\
\end{gather}
The expansion is about the non-interacting quasiparticle Hamiltonian. The corresponding Green's function is
\beq
G^0 = \fr{1}{E -\wl\epsilon_d +i\wl \Delta}
\eeq
By considering diagrams for the self energy, we get
\beq
\pd{\Sigma_\sigma(0)}{h} - \pd{\Sigma_\sigma(0)}{\mu} = 2\pd{\Sigma_\sigma(0)}{E}
\eeq
This can be used to prove the mean-field result eq.~\ref{historia}.


\end{document}
